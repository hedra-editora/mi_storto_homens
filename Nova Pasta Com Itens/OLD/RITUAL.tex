\chapter{Osiip, o ritual de iniciação masculina}

\letra{A}{ história} abaixo conta do Osiip, um ritual de iniciação masculina que
não é mais realizado pelos Karitiana. A geração atual de homens mais
velhos, porém, vivenciou o ritual. O menino, antes de começar a caçar e
poder se casar, precisava perfurar os ninhos de diferentes tipos de
vespas e tomar banhos de plantas importantes para o grupo. Nesse
período, o iniciado precisava também fazer uma reclusão alimentar e
manter um comportamento reservado, tudo sob supervisão de um homem mais
velho, geralmente seu pai.

A narrativa de Osiip foi contada por Cizino Karitiana, o último pajé do
povo até o presente momento, para sua família estendida e Luciana
Storto, que a transcreveu e traduziu com Nelson Karitiana, João
Karitiana, Luiz Karitiana e Inácio Karitiana. Quando, na narrativa,
Cizino se dirige a um interlocutor, está falando a seu filho mais velho
que ainda não se casara, para explicar que o ritual teria sido
necessário no passado quando ele fosse se casar. Para esta publicação, o
material foi editado por Luciana Storto e Ivan Karitiana.

Uma primeira descrição do ritual e análise linguística do uso da arte
verbal nesta narrativa foi publicada em 2019 por Luciana Storto em
\emph{Línguas Indígenas: tradução, universais e diversidade}, através da
editora Mercado de Letras. A versão publicada aqui tem um número
diferente de sentenças, pois o critério usado nesta é prosódico e
naquela era sintático, mas a divisão da narrativa em partes permanece a
mesma.

Uma análise antropológica do ritual, de autoria de Felipe Vander Velden,
é feita em um manuscrito inédito que comentaremos aqui, pois esclarece
aspectos do significado do ritual e da relação dos Karitiana com seres
não-humanos, como as vespas. Vander Velden sugere que os Karitiana usam
picadas de vespas consideradas caçadoras para através do ritual adquirir
esta característica das mesmas. Neste manuscrito inédito, intitulado
``As vespas que caçam com seus dentes: artefatos multiespécies, ritual e
relações entre humanos e não-humanos entre os Karitiana (Rondônia)'', o
autor procura identificar algumas espécies de plantas e vespas citadas
na narrativa, entre elas a planta denominada \emph{sojoty} , cuja seiva
leitosa é usada como remédio no ritual, a saber:

\emph{``Sojoty} (`é como batata do mato', ou `folha que arde',
dizem; certamente se trata de uma \emph{Araceae}, talvez a que chamamos
de `comigo-ninguém-pode', possivelmente \emph{Dieffenbachia
seguine})''.

\chapter*{Osiip}
\begingroup\parindent=0em

\section{Parte I}

1. Meu pai me disse, antigamente

2. Quando temos dez anos de idade, nos fazem receber o \emph{Osiip}

3. Aos dez verões é a salvação do mau caçador do nosso povo

4. Nós perfuramos o vespeiro, nos fazem passar a planta \emph{sojoty}

~

5. Nossos pais nos aconselham quando estamos prestes a nos casar

6. Quando estamos prestes a receber uma mulher

7. Quando nós, de fato, recebemos uma mulher, nós perfuramos o vespeiro

8. No meu caso, eu não perfurei muitos vespeiros

\section{Parte II}

9. Há verdadeiros perfuradores de vespas

10. Perfuradores de vespas por dez vezes, perfuradores de vespas por cinco vezes

11. Quanto a mim, eu perfurei vespeiro apenas quatro vezes

12. Quanto a mim, quatro vezes apenas eu perfurei vespeiro

~

13. Logo após receber as vespas, não se pode comer

14. Três dias depois das vespas, já se pode comer

15. É muito difícil, por causa da fome e da sede

16. Deita-se, não pode haver nenhuma conversa, não pode haver nenhum
barulho

17. Deve-se caminhar em direção à serenidade

18. Quando se está sereno, o espírito do \emph{Osiip} trabalha

19. Nós matamos caça

~

20. Depois de três dias, nós comemos um mingau forte

21. Mingau forte, sementes de milho torradas, espigas de milho assadas
com as folhas, milho assado na espiga é o que nós comemos

22. Não se bebe água

23. Chicha é para ser bebida

24. Não se bebe água mesmo

25. De acordo com a palavra dos nossos pais

\section{Parte III}

26. Com a palavra do meu pai dirigida a mim, eu fiz o \emph{Osiip}

27. Por causa disso eu mato um pouco de caça assim até hoje

28. Macaco apenas

29. No meu caso, eu não matei caça grande

~

30. O \emph{Osiip} é bom

31. Não se mata caça de graça

32. Não se come caça de graça

33. Não se comia caça de graça, antigamente

34. Homens sem o \emph{Osiip} não matavam caça, antigamente

35. O homem que não soubesse matar não era presenteado com caça,
antigamente

36. Então, por causa disso, costumava-se passar pelo \emph{Osiip}
antigamente

~

37. O meu pai falou comigo

38. ``Receba o \emph{Osiipo}'', o meu pai disse para mim

39. ``Você não vai matar caça para a sua esposa se você não receber o
\emph{Osiip}''

40. O meu pai disse para mim

~

41. Quanto a mim, eu recebi o \emph{Osiip} por isso

42. E fui então tirar as plantas da mistura para o \emph{Osiip}, as
plantas para as vespas: \emph{sojoty,} \emph{ewoket},
\emph{gosonderepo}, \emph{Osiip tepy}, e \emph{pasỹ}

43. O Osiip é recebido com a mistura

44. Sem isso, o espírito do \emph{Osiip} não trabalha

~

45. A pessoa não deve comer escondido, não deve se masturbar; isso não é
para ser feito

46. Não é para se comer mamão, óleo não é para ser comido, coisas
gordurosas não são para ser comidas, não se conversa com mulher,
mulheres não podem se aproximar enquanto nós estamos passando pelo
\emph{Osiip}

~

47. Então, meu pai disse para mim: ``nós ainda não terminamos''

48. Então, quanto a mim, eu recebi o \emph{Osiip}

49. Então, quanto a mim, eu perfurei o vespeiro

50. O meu pai cantava, enquanto eu perfurava o vespeiro

51. ``Cantar, cantar, cantar'', o meu pai fazia assim enquanto eu perfurava o vespeiro

~

52. Então, no meu caso, eu perfurei o ninho das vespas \emph{gop miem}

53. As \emph{gop miem} são dolorosas e rápidas

54. Ao meio-dia, a dor da \emph{gop miem} diminui

55. Às duas horas, nós estamos rindo

56. ``Eu vou conseguir sarar'', você diz então

57. Não era a minha hora de morrer

~

58. ``Você sarou?'', o nosso pai nos pergunta então

59. ``Sim'', nós dizemos ao nosso pai

60. ``Muito bom'', o nosso pai diz então

61. ``É muito bom, você viu?'', diz então o nosso pai

62. ``Você não morreu mesmo, viu?'', o meu pai me disse

~

63. Então, quanto a mim, cinco dias se passaram

64. Depois de cinco dias, a primeira coisa que eu matei para sarar foi
passarinho

65. O pequeno pássaro \emph{morondek} está entre os primeiros que se come
para sarar

66. O pássaro \emph{piisomo} está entre os primeiros que se come, sem
espalhar as sobras, para sarar

67. Depois de matar os primeiros passarinhos que se come para curar, a
pessoa come outros: \emph{yrypano}, \emph{pityjo}, \emph{hanhano},
\emph{yt'yto}

~

68. Naquele momento, caça de grande porte ainda não seria comida

69. ``Você ainda não comerá caça de grande porte'', o meu pai me
disse``Ou você salgará'', disse o meu pai

70. ``Você não vai comer caça grande, ou você vai reter líquido''

71. Por causa disso, eu não comia aquilo; aquilo não era comido lá

72. Então, depois de dez dias, depois de dormir dez dias, a pessoa come
caça grande

73. Come-se porco selvagem, macaco. Quanto a macaco-aranha, não se comeria ainda

74. Depois de vinte dias, a pessoa já come macaco-aranha

75. Então já não é mais ruim

~

76. A pessoa ainda não deve se banhar

77. Vai se banhar com a planta \emph{Osiip tepy}

78. O \emph{Osiip tepy} é nosso instrumento de banho

79. O \emph{Osiip mynan}, também, o '\emph{ewoket} também

~

80. Então, um mês depois, a pessoa come peixe

81. O peixe jatuarana grelhado é um dos primeiros que se come

82. O peixe-cachorro é um dos primeiros que se come

83. Traíra, um peixe caçador, é um dos primeiros que se come

84. Então, depois de um mês, come-se jatuarana

~

85. Então, aos dois meses, caça realmente mansa começa a se aproximar

86. Veado, caça, o \emph{Osiip} amansa, amansa

87. Caça, o \emph{Osiip} amansa

88,Porco selvagem, o \emph{Osiip} amansa

89. Caça, o \emph{Osiip} debilita

90. Caça, o \emph{Osiip} desnorteia

91. Então, no caso do nosso povo, nós gostamos que a caça se aproxime de nós

~

92. Nós sabemos como matar a caça, nós atiramos bem

93. Nós não trememos mais

94. Então nós matamos a caça com arco e flecha

95. Não se matava caça com as armas do homem branco, antigamente

96. Grandes flechas matavam a caça, antigamente

97. Com flechas de ganchos, a pessoa fazia a caça morrer gritando, antigamente

~

98. O meu pai falava, antigamente

99. Eu fiquei com o meu pai até amadurecer como atirador

100. Com a sorte que eu tinha, o meu pai matava muito

101. Dez macacos, o meu pai matava

102. Quanto a mim, eu só matei três

\section{Parte IV}

103. Era assim

104. Então em três meses, o \emph{Osiip} acaba

105. Se não se casa, não acaba

106. Se a gente não casa, o \emph{Osiip} será repetido depois de três meses

107. Então já não é mais ruim

108. Nós não ficamos loucos, nós ficamos espertos

~

109. Quando acaba, em quatro meses, a pessoa faz o \emph{Osiip} de novo

110. A pessoa pega novamente \emph{gop sõwõrã}

111. O vespeiro \emph{gop miem} não é perfurado novamente

112. Vespeiros de \emph{gop sõwõrã} são perfurados novamente

~

113. As vespas vermelhas me fizeram desmaiar

114. Vespas vermelhas são muito dolorosas

115. Só quando cai a noite as picadas de vespas vermelhas aliviam

116. Quando dormimos, melhora

~

117. Depois que melhora, a pessoa perfura novamente

118. ``Bate, bate, fura'', a pessoa faz um buraco novamente no vespeiro

119. Então a mão entra, a larva dele é novamente removida

120. Retira novamente a larva

121. O nosso braço inteiro é coberto de vespas novamente

122. A dor fica conosco até o romper do dia

123. Aquilo fica até o romper do dia

~

124. Então é assim que éa o \emph{Osiip}

125. Quanto a mim, eu peguei as vespas quatro vezes

126. As vespas \emph{gop sõrõwã,} as \emph{gop miemo}s as \emph{gop miemo} duas vezes, as \emph{gop sõwõrã} duas vezes também

127. As \emph{gop bisõwõrã} eu não peguei

128. As \emph{gop bikip} eu não peguei

129. Foi assim

\section{Parte V}

130. Então quanto a mim, eu matei caça

131. Macacos, eu eliminei muitos macacos

132. Quanto a mim, eu matei muitos macacos

133. Quanto a mim, eu carreguei muitos cestos de caça

~

134. Por isso, eu casei

135. Quando eu matei caça, o meu falecido pai me liberou para uma mulher

136. ``Case-se'', ele disse

137. Por isso, eu casei

138. ``Casar'' eu fiz

139. Eu fiquei com esposa

140.  Minha esposa nunca passou fome comigo

~

141. No meu caso, eu matei caça

142. No meu caso, eu alimento a minha esposa assim até hoje

143. Com a arma do homem branco, eu alimento a minha esposa assim até hoje

144. Desde que eu fiz o \emph{Osiip}, eu sou um bom caçador, e assim permaneço

~

145. Eu não sou um bom caçador, eu sou um caçador mais ou menos até hoje em dia

146. Permaneço até hoje

147. Antigamente, eu não era desse jeito

148. Eu não precisava sair antigamente, não mesmo

~

149. Na fossa, então, havia macacos

150. Macacos apareciam de repente

151. Nós não íamos longe

152. Cotia é a primeira caça que aparecia sorrateiramente

153. Nós buscávamos as nossas armas e logo, havia macacos lá

154. É isso que a gente matava

155. Não havia falta de caça

156. Nambús, quando fazemos o \emph{Osiip}, aparecem muitos nambus

~

157. Quando você fizer o \emph{Osiip}, você também vai ser assim

158. Você não vai mais ser como você é agora

159. Você não vai mais ser caçador ruim

160. Você não vai mais ser um mata-nada

161. Vocês vão sentir o peso da sua caça

~

162. Plantas parecidas com cipós precisam ser passadas em nós

163. O timbó precisa ser aplicado no nosso rosto

164. Aplicado aqui no rosto

165. Aplicado aqui e aqui

166. A erva \emph{pasỹ} tem que ser aplicada aqui e aqui

~

167. O \emph{Osiip} faz a gente ficar com tornozeleiras e um cinto, que
cheiram bem

168. Então a caça definitivamente não desaparece

169. O cipó do \emph{Osiip} é feito para ser colocado nos nossos pênis

170. A nossa urina o joga fora

171. A nossa urina faz ``tchoo'', pra cima

172. Nós não tocamos no nosso pênis

173. Então nós matamos a caça

~

174. Então os nossos velhos dizem

175. ``Agora atire para cima''

176. Em quatro meses, os nossos velhos nos dizem ``tente aquela em
seguida''`

177. Então nós atiramos, bem envergonhados

178. Não achamos que vamos acertar, mas acertamos

~

179. A flecha vai sozinha nela

180. ``Voa, acerta'' no meio dela

181. A flecha cai, bonita, junto da caça

182. ''ai, perfura''

183. Nós achamos que acertamos sem querer

~

184. Então nós fazemos novamente

185. Nós não achamos que vamos acertá-la, mas, mesmo assim, de qualquer
forma, nós acertamos novamente

186. ``Voa, acerta'' novamente

187. Nossa flecha nunca está vazia

~

188. ``Ah, eu estou assim'', nós dizemos

189. Nós ficamos muito bons

190. A nossa flecha fica como a arma 22 do homem branco

191. Nossa flecha genuína

192. Cai apenas na caça

193. Mesmo que pensemos que não vamos acertar, acertamos

194. A flecha encontra sozinha o seu caminho até a caça

195. Eu fiz o pai do Rogério ficar meio ansioso por causa de um jacamim
voando

196. Era assim realmente

~

197. O meu pai cantava assim

198. Há muitas músicas do \emph{Osiip}

\begin{quote}
\forceindent\emph{199. Cesto de caça até o topo, até o topo, até o topo}

\emph{200. E eu o estou carregando, estou }

\emph{201. Cesto de macacos até o topo, até o topo}

\emph{202. Cesto de macacos-aranha está cheio até o topo}

\emph{203. E eu o estou carregando, estou}
\end{quote}

204. Esta é a música do \emph{Osiip}

\begin{quote}
\forceindent\emph{205. Bate, bate, bate, bate, a madeira que se move está chorando}
\end{quote}

206. Agora é isso, acabou

~

207. Quando nós estamos começando a entrar no ninho das vespas

208. O pássaro \emph{teõwãt,} olha para nós. O pássaro '\emph{eet'eet},
olha para nós. O pássaro \emph{hĩrã}, olha para nós

209. ``Vá'', eles nos dizem

210. Então nós não ficamos mais parados

~

211. Os nossos pais pegam a gente

212. No peito deles os nossos pais nos levam

213. Empurrar a gente quando está perto das vespas, os nossos pais fazem

214. Então os nossos pais correm um pouco

~

215. Cair em um abraço, nós fazemos, no topo do ninho das vespas

216. O nosso medo não existe mais

217. ``Susto'', ele nos pega com isso

218. Nós então nos sentamos abaixo dele, nessa altura, na árvore

219. Nós nos sentamos de frente para ele

~

220. Então as vespas fazem ``ffffff''

221. A mão vai para dentro e para fora do ninho

222. Nós limpamos o agrupamento delas

223. Depois disso, várias vespas picam o nosso braço esquerdo

224. Então nós pegamos a sua larva

225. E a colocamos aqui no nosso peito

226. Nós não respiramos mais

~

227. Eles dizem que isso é para que o nosso cesto de caça fique tão cheio
que mal possamos carregá-lo

228. Eles dizem que nós ficamos curvados

229. Então nós caímos muito perto, exatamente lá

~

230. Então nós aplicamos a planta \emph{sojoty}

231. Tiram-se os ferrões

232. Nós aplicamos a planta \emph{sojoty} nos ferimentos deixados pelos
ferrões

233. Nós fazemos uma massagem usando o \emph{sojoty}

234. Também há um tipo diferente de \emph{sojoty} cru

235. ``Você vai aplicar? \emph{Sojoty} cru?'' os nossos pais nos dizem

236. ``Aplique-o'', nós dizemos

237. Nós achamos que a dor vai embora

238. O nosso pai aplica isso em nós

239. Passa, passa, passa no nosso ânus, no nosso pênis

240. Não há mais

241. Então a dificuldade de caçar vai embora

242. Então não há mais dificuldade de caçar

243. Com essa aplicação, nós sentimos a dor do \emph{sojoty}

244. O \emph{sojoty} dói de um jeito diferente

~

245. Foi assim que eu fiz

246. Então o meu pai riu de mim

247. Então para a minha surpresa, de tarde eu já estava curado

248. O meu pai ficou alegre

249. ``Agora, você vai matar caça, meu filho''

250. ``Você vai matar para a sua esposa''

251. ``Você vai matar para os seus familiares''

252. O meu pai disse pra mim

~

253. Então eu já não era ruim de caça

254. Aí eu me tornei um verdadeiro caçador

255. Um verdadeiro matador de caça

256. Um verdadeiro matador de mutum

257. Por causa disso, você está bem alimentado hoje

\section{Parte VI}

258. Agora eu só mato com a arma do homem branco

259. Eu não costumava matar com as armas do homem branco antigamente

260. Quanto a mim, é com flechas que eu costumava matar antigamente

261. Fazer tiro ao alvo

~

262. Depois disso, eu perfurei de novo

263. O \emph{gop miem}, eu perfurei novamente

264. O \emph{gop miem} que tinha deixado o ninho

265. Esses não me picaram

266. Esses não picam muito

267. Então eles só picaram o meu peito

~

268. Depois disso, eu não tinha mais o meu pai

269. O meu pai morreu

270. Quando eu estava quase concluindo o \emph{Osiip}, o meu pai morreu

271. Depois que meu pai faleceu, eu não perfurei mais vespas

~

272. Foi aqui que eu perfurei novamente um vespeiro

273. O irmão do meu pai me fez colocar a mão no vespeiro \emph{gop
sõwõra}, o irmão mais novo dele, meu cunhado, meu sogro, me fizeram
colocar a mão em um vespeiro

274. Desta vez, a vespa \emph{gop sõwõra} me picou

275. A vespa picou até as minhas orelhas

~

276. No começo, não se come no \emph{Osiip}

277. Nós dormimos três dias

278. Fica-se muito fraco

279. Quando dez dias passavam, os homens daqui costumavam matar a
primeira coisa para comer

280. Em cinco dias, o meu pai me deixou comer passarinho

~

281. Então nós vivemos como bons caçadores até os dias de hoje

282. Waldemar, Garcia, nós estamos matando um pouco até os dias de hoje

283. Antigamente, nós nos guiávamos pela vivência dos nossos anciões



