\chapter{Minibiografias}

\paragraph{Barabadá Karitiana} foi um pajé Karitiana pertencente ao grupo Joari
(também conhecido como Capivari). Ele narrou o encontro, vivenciado por
ele, entre dois grupos de Karitiana que viviam em aldeias separadas, os
Joari (ou Capivari) e os Karitiana. Estes grupos passaram a viver juntos
na Terra Indígena Karitiana.

\paragraph{Cizino Karitiana} é o atual pajé e cacique Karitiana. Ele narrou o ritual
de iniciação masculina, intitulado \emph{Osiip}, pelo qual passou várias
vezes.

\paragraph{Garcia Karitiana} foi um cacique Karitiana. Ele narrou os mitos de origem
do sol e da lua.

\paragraph{Inácio Karitiana} é licenciado em Educação Básica Intercultural pela
Universidade Federal de Rondônia e professor da Escola Indígena Estadual
de Ensino Fundamental Kity Pypydnipa.

\paragraph{Ivan Rocha} é doutor em Linguística pela Universidade de São Paulo e
pesquisador visitante do Museu Paraense Emílio Goeldi, com bolsa do
Programa de Capacitação Institucional (\textsc{pci}) do Ministério da Ciência,
Tecnologia e Inovação (\textsc{mcti}/\textsc{cnp}q).

\paragraph{João Karitiana} é licenciado em Educação Básica Intercultural pela
Universidade Federal de Rondônia e professor da Escola Indígena Estadual
de Ensino Fundamental e Médio Kyõwã.

\paragraph{Luiz Karitiana} é licenciado em Educação Básica Intercultural pela
Universidade Federal de Rondônia e professor da Escola Indígena Estadual
de Ensino Fundamental e Médio Kyõwã.

\paragraph{Nelson Karitiana} é licenciado em Educação Básica Intercultural pela
Universidade Federal de Rondônia e professor da Escola Indígena Estadual
de Ensino Fundamental e Médio Kyõwã.

\paragraph{Valdomiro Karitiana} é filho de Barabadá Karitiana. Ele acompanhou a
linguista durante a gravação da história do encontro entre os dois
grupos locais e auxiliou na transcrição e tradução.

\begin{comment}
(Quarta-capa)

Então ele (Moraes) morreu, a doença do branco o matou

Então meu finado pai se juntou

Quando o Moraes morreu, nós ficamos juntos

Nunca mais nos separamos

Assim nós ficamos, assim

Não havia mais homens quando ele morreu

Os homens ficaram com o meu pai

O meu finado pai trouxe todos para cá

Então o meu pai morreu aqui

Então eles não foram mais na aldeia deles

Aqui meu finado pai morreu

Assim o pessoal ficou junto

Não existe mais o que contar, esse é o final
\end{comment}


