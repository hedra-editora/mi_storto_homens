\chapter{Apresentação}
%\chapter{Quem são os Karitiana}

Os Karitiana são um grupo indígena ainda pouco conhecido no Brasil. Vivem no atual estado de Rondônia, considerado o lugar de origem da
língua-mãe de todas as línguas tupi. Falam a língua de
mesmo nome, que é a única remanescente da família linguística arikém, o
que lhes confere uma importância central para os estudos comparativos
das línguas tupi e, consequentemente, das línguas indígenas como um
todo.

Se aproximaram dos não indígenas durante o ciclo da
borracha. Tanto a memória do grupo quanto os documentos não indígenas dão
destaque para esses vínculos de trabalho caracterizados pela violência
dos patrões. Nesse período, os não indígenas disseminaram entre eles
diversas doenças, como a gripe e o sarampo. Por isso, em meados do
século \textsc{xx}, os Karitiana sofreram um grande declínio populacional 
chegando a apenas 64 pessoas na década de 1970. Para que continuassem a
existir, dois grupos locais decidiram viver juntos, casando-se entre si,
e procuraram o Serviço de Proteção aos Índios, para que seus direitos
como povo indígena fossem garantidos.

No censo realizado pelo linguista Ivan Rocha em 2017, os Karitiana
contavam com 397 pessoas.

Atualmente eles habitam sete aldeias, sendo cinco na Terra Indígena
Karitiana, demarcada em 1986, e duas fora dela, em áreas que são parte
do seu território tradicional. Algumas famílias também moram nas cidades
rondonienses de Porto Velho, a capital, e em Cacoal.

Além de trabalharem em atividades agrícolas, no manejo dos recursos
florestais e na produção de artesanato, os Karitiana também são
profissionais das áreas de saúde e educação. O grupo luta historicamente
pela garantia de seus direitos, como a ampliação da terra indígena e o
fortalecimento da educação e da saúde indígena.