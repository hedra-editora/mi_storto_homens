\part{Não havia mais homens}

\chapter{A história de Gokyp, o Sol}

\letra{A}{ história} do Sol é uma das narrativas que compõem o repertório de mitos
dos Karitiana. Nascido como uma criança muito quente, que parecia
febril, ninguém conseguia chegar muito perto do Sol sem se queimar. As
pessoas, preocupadas com o perigo que ele significava para a comunidade,
pensaram em matá-lo. Cada vez mais quente, o Sol subiu pelo esteio de
uma casa e decidiu ir para o céu, onde vive até hoje.

A narrativa aqui publicada foi contada por Garcia Karitiana para Luciana
Storto, que a transcreveu e traduziu com Nelson Karitiana. Para esta
publicação, o material foi editado por Íris Morais Araújo e Karin
Vivanco.

Uma primeira transcrição, glosagem e tradução desta narrativa foi
publicada por Storto na \emph{Revista Linguíʃtica} 15 (2019).

\chapter*{O Sol}
\begingroup\parindent=0em

1. Dizem que o Sol vivia antigamente

2. Dizem que o Sol começou sua existência como uma criança

3. Dizem que o Sol vivia

~

4. Então os homens disseram ``O que é isso?''

5. Dizem que a vida começou como uma doença para ele

6. Dizem que a criança ficava cada vez mais quente

~

7. ``Esse aí está doente?'', falava o seu pessoal

8. Seria semelhante a uma doença

9. Mas ele não estava doente realmente

10. Ele nunca esteve doente

~

11. Então dizem que o calor dele ficava cada vez mais intenso

12. O Sol era meio quente e foi ficando mais quente

13. Naquele momento, ele se tornaria o Sol

~

14. Então aconteceu

15. ``Ah, o que é isso?'', diziam os homens

16. Aí, ele não existia mais

17. Então ele se tornou tão grande que não podia mais viver aqui

18. Ele se tornou enorme

~

19. Aí, dizem que o Sol não queimava mais só um pouco

20. Dizem que ele se tornou incandescente

21. Quando sua incandescência ficou insustentável, dizem que os homens
queriam matá-lo

22. Porque ele não era mais um ser humano

~

23. Os homens tinham um desejo de matar

24. Então não o fariam

25. Aí, os homens pensaram

26. ``O Sol não pode ser morto''

~

27. Eles pensaram que era assim que fariam

28. Eles pensavam que ele era humano

29. Mas ele estava ficando diferente, meio fraco

~

30. Por causa disso, os homens sentiam vontade de matá-lo

31. Matá-lo é o que queriam fazer

32. Aí, ele continuou a viver

~

33. Dizem que ele saiu pelo esteio central do telhado da casa

34. Porque dizem que os homens tinham o desejo de matá-lo

35. Sabia-se que queriam matá-lo

~

36. Na frente deles, então, ele saiu

37. Pelo esteio central da casa ele sairia

38. Aí, vieram os homens com bordunas para matá-lo, sorrateiramente

39. Mas eles não podiam mais se aproximar dele

40. Ele estava muito incandescente mesmo

~

41. Foi então que, dizem, ele saiu pelo esteio central da casa

42. Aí, o Sol cantou antes de sair

\begin{quote}
\forceindent\emph{43. Tragam-me para dentro, mulheres}

\emph{44. Tragam-me para dentro, mulheres}

\emph{45. Tragam-me para dentro, tragam-me para dentro, mulheres}

\emph{46. Pois o crânio partido está me matando}
\end{quote}

47. Assim disse o Sol enquanto ele subia

48. Então dizem que o Sol subiu; ``Pro alto'' ele foi

49. Então dizem que quem queria matá-lo morreu enquanto ele estava subindo

50. ``Caíram esparramados'', os homens

~

51. O Sol subiu para as alturas

52. Então o Sol ficou incandescente, incandescente de verdade

53. Assim, dizem, é que a história do Sol deve ser contada



