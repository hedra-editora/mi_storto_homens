\chapter[Para ler as palavras karitiana]{Para ler as palavras\break karitiana}

\textls[-10]{Neste livro, foi adotada a ortografia elaborada pela linguista Luciana Storto, que coordenou um programa de alfabetização da língua karitiana desenvolvido junto à comunidade na década de 1990 e aprovado por ela em 1996, quando as convenções ortográficas foram registradas no material de apoio ao aprendizado da ortografia karitiana, que tem sido usado desde então no ensino de sua língua materna. Atualmente, o grupo vem discutindo a reformulação de algumas dessas convenções ortográficas.}

\section{Vogais}

\begingroup
\begin{tabular}{rl}
/a/ & como \textit{a} em \textit{até}\\
/e/ & como \textit{e} em \textit{mesa}\\
/i/ & como \textit{i} em \textit{idoso}\\
/o/ & como \textit{o} em \textit{hoje}\\
/y/ & como um som intermediário entre \textit{i} e \textit{u}\protect\footnotemark\\
\end{tabular}

\footnotetext{\textls[-10]{Esse som não possui um equivalente no português do Brasil. Para pronunciá-lo, se pode falar um \textit{i} e, gradualmente, mover a língua em direção a um \textit{u}. Quando a língua estiver em uma posição entre \textit{i} e \textit{u}, esta será a pronúncia do \textit{y}. Os linguistas classificam esse som como uma vogal ``central alta'' e, a partir de um inventário internacional convencional de símbolos, o Alfabeto Fonético Internacional, transcrevem-no como um ``i'' tachado, o símbolo ``ɨ''.}}
\endgroup

\section{Consoantes}

\begingroup
\begin{tabular}{rl}
/b/ & como \textit{b} em \textit{boto}\\
/d/ & como \textit{d} em \textit{dedo}\\
/g/ & como \textit{g} em \textit{gato}\\
/h/ & como \textit{r} em \textit{rato}\\
/j/ & como \textit{dj} no início da palavra; no meio \textit{i} como em \textit{saia}\\
/k/ & como \textit{c} em \textit{casa}\\
/m/ & como \textit{m} em \textit{mulher}\\
/n/ & como \textit{n} em \textit{nariz}\\
/p/ & como \textit{p} em \textit{pé}\\
/r/ & como \textit{r} em \textit{arara}\\
/s/ & como \textit{s} em \textit{sapo}\\
/t/ & como \textit{t} em \textit{tatu}\\
/w/ & como \textit{u} em \textit{água}\\
/x/ & como \textit{tch} em \textit{tchau}\\
/`/ & \textls[-15]{uma pausa, como quando dizemos \textit{ã--ã} com o sentido de \textit{não}}\protect\footnotemark
\end{tabular}\\

\footnotetext{\textls[10]{Corresponde a uma breve pausa entre as duas sílabas, que equivale a uma obstrução, na região das cordas vocais, do fluxo de ar que vem do pulmão, chamada de consoante oclusiva glotal no alfabeto fonético.}} 
\endgroup