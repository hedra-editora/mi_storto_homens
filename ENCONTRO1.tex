\chapter{O encontro de dois grupos locais}

\letra{A}{ história} abaixo, contada por Barabadá, trata de um evento crucial para
os Karitiana, que ele vivenciou quando era jovem: a reunião de dois
grupos locais, que até então eram autônomos: os Joari (ou Capivari),
grupo de Barabadá, e os Karitiana, grupo de Moraes Karitiana. Moraes
tinha se casado com sete mulheres para assegurar a continuidade de seu
grupo e os Joari tinham apenas uma mulher naquele momento. A reunião
dessas pessoas na atual aldeia central da Área Indígena Karitiana e a
realização de casamentos entre elas foi fundamental para que se
revertesse o forte declínio populacional vivenciado por ambos os grupos,
que poderia resultar em sua extinção.

A narrativa foi contada por Barabadá em sua casa, sua esposa Augusta e
seu filho Valdomiro, gravado por Luciana Storto, que a transcreveu e
traduziu com a ajuda de Valdomiro Karitiana e Nelson Karitiana. Para
esta publicação, o material foi editado por Luciana Storto e Ivan Rocha.

\chapter*{O encontro de dois grupos locais}\parindent=0em

1. Quando minha mãe morreu, eu fui andar sozinho

2. Eu fui andar lá

3. Depois eu voltei de novo aqui

4. Eu fiquei aqui por um tempo

~

5. Depois eu fui de novo lá na aldeia

6. Então eu vi o acampamento dos homens

7. Onde dormiram, onde fizeram fogo

8. Eu vi o local da fogueira deles de perto

9. O acampamento abandonado deles

10. ``Tinha gente aqui, né?'', pensei ``Tinha gente\footnote{Pessoas da
  mesma etnia.  Os Karitiana e Joari pertenciam a um mesmo grupo étnico
  que havia se separado.} aqui?''

~

11. Meu finado pai secava taquara

12. Então eles chegaram ali

13. ``Dormiram aqui?'', eu disse

14. Então meu finado pai e os outros chegaram

15. Chegaram

16. ``De quem é este acampamento? Este acampamento é do inimigo?''\footnote{Inimigo
  nesta passagem é entendido como não indígena, ou não Karitiana.}

17. ``Aqui tinha gente'', meu finado pai falou

18. ``Vamos ver o lugar onde o inimigo dormiu'', eu disse

19. ``Vamos lá''

~

20. O caminho tinha acabado

21. Não tinha picada {[}aberta na mata{]}

22. Então nós retornamos

23. Então eu fui pelo caminho, eu continuei sozinho

24. Fui, então chegou meu finado pai atrás de mim

25. ``Vamos ver, você não vai sozinho. Você não deve continuar sozinho'',
meu finado pai falou

~

26. Eu fui

27. Andamos, andamos, andamos, chegamos

28. Então eu fiquei pensando, com saudade de quando eu estava na beira do
rio

29. ``Eu não vou cruzar o rio por aqui'', eu falei para mim mesmo

30. Então eu vi a outra margem

31. Cruzei pela água

~

32. Eu olhei, eu continuei de novo, nós continuamos, chegamos na aldeia

33. Havia pegadas de homens

34. ``Homens estiveram aqui!'', disse meu finado pai

35. Então fomos por ali

36. Eu fui na frente

37. Andamos, andamos, andamos, chegamos na casa do inimigo

38. Quando chegamos na casa deles, o galo cantou

39. ``Aqui tem gente!''

~

40. Andamos, andamos, andamos, chegamos

41. Nós ficamos em pé perto deles

42. Meu finado pai ficou com medo

43. ``É o inimigo'', disse meu finado pai

~

44. Nós voltamos e fomos de novo

45. Então, dizem que os homens chegaram de novo

46. Homens muito valentes

47. ``Ah, vamos matar os inimigos!''

48. ``Vamos ver! Assim será''

49. ``Já que você é valente, você vai entrar''

50. Estávamos doentes, com a doença do branco\footnote{A expressão
  ``doença do branco'' pode se referir a diferentes doenças levadas
  paras as aldeias por agentes públicos, missionários, garimpeiros e
  madeireiros, por exemplo, malária, sarampo, gripe, tuberculose etc.}

51. Então nós não entramos

52. Eles vieram atrás de nós

~

53. Então, quando eles vieram, apareceram outros homens

54. Eles esperavam caça no pé de tucumã\footnote{\emph{Astrocaryum
  aculeatum}.}

55. Enquanto eles esperavam no pé de tucumã,

56. Foram buscar o finado pai de Eremby Byyt\footnote{O finado pai de
  Augusta, esposa do narrador Barabadá.}

57. ``Eu matei mutum'', ele disse

58. ``Então, eu vim'', eu disse

59. Eu fui

60. ``Eu vou esperar no local onde o mutum come, amigo''

61. ``Ah'', disse meu finado pai

~

62. Eu fui, andei, andei, andei, sentei

63. Diz que eles imitavam o barulho do nambu

64. ``\emph{Hong}'', ele fez. ``\emph{Hong}, aqui tem alguém'', disse o homem

65. ``\emph{Hong}, \emph{hong}, \emph{hong}'', fizeram

66. Dizem que chegaram na tocaia

67. ``Olha aqui, olha aqui, aqui tem alguém'', disseram

68. ``Você matou cotia, amigo?'',\footnote{Amigo ou querido(a), com a função
  de um vocativo.} ele falou. ``Amigo?'' (silêncio)

69. ``Amigo? Aqui tem alguém, porque tem um buraco na tocaia''

70. Dizem que falou assim o finado pai de Eremby Byyt (Augusta)

71. Olhou, abriu, fez assim por cima da tocaia

72. Olhou, viu a arma dele com um desenho em vermelho e branco

73. Ele ficou com medo

~

74. Olhou, andou, andou, parou

75. ``É você, amigo?'', falou para o amigo dele. ``É você, amigo?'', chegou

76. ``Aqui tem.  Vai lá com ele''

77. Dizem que falou assim o finado pai de Eremby Byyt

78. Dizem que foi lá com ele

79. Olhou, abriu

80. ``Olha aqui!''

81. ``Ah! Você me deu medo, amigo!''

82. ``Sai, amigo'', disse ele

83. ``Você não vai me matar né, amigo?'', disse o amigo dele

84. ``Não vou matar não, amigo. Sou gente (da mesma etnia que você), amigo''

85. ``Você me deu medo, amigo. Você me causou muito medo. Quando você me
estranhou, eu fiquei assim''

86. ``Eu não estranho você não. Vem para cá''

87. ``Você não vai me matar não, amigo?''

88. ``Vou nada! Vou nada! Vem para cá''

~

89. Diz que ele saiu

90. Andou, andou, saiu

91. ``É ele mesmo'', disse o homem ``Você está sozinho?''

92. ``Estou com companhia. Meu amigo está longe'', ele disse

93. Diz que ele ficou em pé e quase caiu o adorno do pênis dele

94. ``O adorno do seu pênis está caindo, meu amigo'', diz que falou o
sobrinho dele

95. ``Não cai, não, amigo. Não vai cair não, amigo! Quando você me
estranhou, eu fiquei assim, amigo''

96. ``Ah, não vai ser fácil, meu amigo''

97. ``Você me deu medo, amigo''

98. ``Qual é o seu nome?''

99. ``Esse é meu nome'', disse ele

100. ``Este é o nome do meu tio'', dizem que falou o finado pai
dela.\footnote{O narrador se refere aqui à sua esposa, Augusta Karitiana.}
``Eu não sabia que você era meu tio''

~

101. ``E o seu amigo?''

102. ``Está lá''

103. ``Não temos mais mulher. Só há uma mulher'' (na aldeia do grupo
Joari/Capivari)

104. ``Vamos lá, vamos tomar chicha e comer carne na casa dele''

105. ``Tem carne na mão dele''

106. ``Tem homem caçador aqui''

107. ``Quem matou macaco-aranha?''

108. Aquele que matou disse para mim

109. ``Ah, vamos ver a caça''

110. Fomos indo em zigue-zague, zigue-zague, zigue-zague, chegamos

111. Enquanto eu comecei a fazer um jirau

112. ``Ah, vamos, meu tio. Vamos tomar chicha, meu tio''

~

113. Fomos, andamos, andamos, chegamos

114. Eu estava lá

115. ``Ah, nós chegamos'', disseram os homens. ``Ah, nós chegamos.
Seguindo o seu cheiro, nós chegamos aqui''

116. ``Ah, vocês são como nós também''

117. ``Nós somos vocês. Nós fomos feitos da mesma parte do cabelo do
Byjyty\footnote{Byjyty é o neto do demiurgo Botyj̃;  de mechas de seu
  cabelo, foram criados os seres humanos de todas as etnias indígenas.}

118. ``Deita aqui, deita aqui comigo''

119. Deitaram, deitaram comigo

120. ``Ah'', eu falei para ele, ``Cadê a sua família?''

121. ``Eu não tenho família. Meu pai foi lá para o Igarapé Preto'', disse
meu irmão sobre meu finado pai

~

122. Conversei, conversei, falei

123. O meu finado pai veio para cá

124. Andaram, andaram, chegaram.

125. ``Devagar os homens vieram'', falou meu finado pai

126. ``Chegaram? Quem? É o Taaty?''

127. ``Não é não, é o pai dele''

128. ``Está bem'', e foram

129. ``Sim'', o amigo dele disse, entrou

130. ``Você está aqui, meu irmão?'', ele disse

131. ``Estou aqui, meu irmão. Você está deitado?''

132. ``Estou aqui, meu irmão'', disse meu finado pai

133. ``Eu estou atrás de você. Você ficou com medo de mim, meu irmão?''

134. ``Ah, eu sei, meu irmão, eu sei mesmo, meu irmão''

135. ``Eu sei, meu irmão, só que eu não sabia. Pensei que você fosse branco''

136. ``Não sou branco não, meu irmão. Sou eu!''

137. ``Eu sei que você é gente, por isso vim me juntar a você''

138. ``Está bem!'' Os homens começaram a ficar juntos, ficar juntos, {[}e então{]} foram embora.

~

139. Meu finado pai levou o finado pai de Nyorem.\footnote{O narrador se
  refere ao pai da Lourdes Karitiana (Nyorem).}

140. ``Onde é o rio \emph{Bisỹ Herednemo}, meu avô?''

141. ``É para cá'', falou meu finado pai, e foi

142. Foi o finado meu irmão, e ficou com medo do finado pai da Augusta,
que foi atrás dele

143. Saíram, foram embora, e um da família dele também foi com ele

144. Depois, ele disse para o nosso pessoal: ``Onde está o lugar dos
macacos- aranha?''

145. ``Vou caçar'', ele disse

146. ``Vamos caçar'', disse o finado pai do Epitácio. O Pereira era pequeno.

147. Então foram. Eu fiquei sozinho

~

148. ``Você sabe onde os macacos-aranha ficam, meu tio?'', falou o finado pai do Antônio Paulo

149. ``Lá, onde fica o meu caminho de caça, tem macaco-aranha''

150. ``Vou lá''

151. ``Vamos lá, meu tio''

152. Então correu até não dar mais para alcançar, dizem que não dava; ele fugiu

153. Andou, andou, andou

154. Eu fiquei com eles, sozinho, então comemos batata, comida

155. ``Ah'', ele foi falando, ``você sabe onde está a caça, meu cunhado?''

156. ``Eu vi mesmo, sobrinho. No lago aqui perto tem muito macaco-aranha''

157. ``Vamos ver lá''

158. ``Então vamos'', eu disse

159. Nós não caçamos, não, nós ficamos sentados, conversando

~

160. Fomos, andamos, continuamos, andamos, andamos

161. ``Aqui tem mutum'', olhamos, chegamos no lago, não tinha mutum

162. Tinha macaco-aranha, ``\emph{poo}, \emph{poo}'', faziam os
macacos-aranha. 

163. ``Aqui tem, vamos matar este macaco-aranha''

164. ``Eu vou matá-lo'', ele disse

165. ``Mata!''

166. Eu não estava agachado, eu estava em pé

167. Foram, atiraram lá longe no macaco-aranha e no quati, atiramos cinco
vezes

168. Nós nos reunimos e andamos

~

169. ``Eu só matei bicho ruim!''

170. Eu pensei que era mucura\footnote{Mucura é de uso regional. O animal
  é conhecido em outras regiões do Brasil como gambá. Nome científico:
  \emph{Didelphis virginiana}}

171. ``Em qual você atirou?'', eu disse

172. ``Eu só matei bicho ruim. Eu não atirei em bicho bom não''

173. ``O que foi que você matou? O quê? Irara\footnote{Nome científico:
  \emph{Eira barbara}}?''

174. ``Não é irara não, é quati''\footnote{Nome científico: \emph{Nasua
  nasua}}

175. ``Por que você não traz?'', eu disse

176. ``Você come quati?''

177. ``Eu como quati'', eu disse

178. ``Está bem, eu também como''

179. ``Pega'', eu disse

180. Então foi deixar a espingarda dele de novo

~

181. Andaram, andaram, pegaram

182. Trouxeram de novo, andaram, reuniram-se aqui

183. Nós nos sentamos em cima do tronco, então não andamos mais, só
ficamos falando

184. Nós nos sentamos.

185. ``Já pegamos caça, e agora vamos esperar os homens aqui''

186. ``Eles vêm vindo lá. Nós íamos deixá-los''

187. Contou, contou o que deram para ele, como viviam naquele tempo

188. Contou como pegou a mulher do outro, contou o que deram para ele,
falou, falou

189. Falaram, falaram, terminou, falaram, falaram, terminou

190. Falaram, falaram da mulher namoradeira, acabou

~

191. Chegou o irmão mais novo dele

192. ``Vocês estão chegando?''

193. Chegaram os amigos dele

194. ``E meu tio?''

195. ``Meu tio não está aqui não, ele já foi, não dá mais para
alcançá-lo, tio. Eu acho que ele foi em direção ao Igarapé Preto''

196. ``Para você ele não foi lá não, meu sobrinho?''

197. ``Não, nós estávamos juntos''

198. ``Vamos, que está escurecendo''

199. Então foram embora, foram

~

200. Andaram, andaram, andaram.

201. ``Mais para lá tem mutum'', eu disse

202. Andamos, andamos, andamos mais um pouquinho, o mutum voou

203. ``\emph{Pij̃ bok bok xak}'', e atiramos no mutum

204. Então o amigo dele atirou e eu fiquei pensando

205. ``Não se pode matar caça do outro''

206. Atiraram, atiraram, aquele que o matou se alegrou, o amigo dele

207. ``Eu o matei, meu amigo'', ele disse fingindo

208. ``Vá pegá-lo, meu amigo'', ele disse

209. Eu fiquei pensando que aquele que estava comigo não o matou

210. Aquele que estava comigo matou só quati

211. Estava um cheiro ruim,\footnote{Referências a quatis que é uma das
  caças com cheiro forte, na região amazônica esse cheiro ruim é também
  conhecido como pitiú.} não dava para lavar, não dava para comer

212. Mutum, macaco, quati, macaco-aranha, nambu azul, todos os homens
foram até a caça, acabou

213. Só eu mesmo fiquei por aqui, os homens foram lá

214. Então os homens foram

~

215. Nós fomos, fomos

216. ``Leva a gente lá, meu primo'', eles disseram

217. ``Leve-me, meu irmão'', disse o finado pai desta (Augusta).  Vamos,
meu cunhado'', ele disse para mim

218. ``Eu vou ficar aqui'', eu disse, quando ele falou assim para mim

219. ``Ah, eu vou''

220. ``Ah, então vai'', eu disse

~

221. Depois de cinco dias, vieram de novo para cá

222. Aquele que o inimigo matou estava doente

223. Minha família estava doente

224. Quando o filho sarou, o pai dele estava lá, à espera do filho

225. Dormimos cinco dias, então voltamos para cá

226. Então ele voltou para onde eu estava, apesar de eu não acreditar que
ele fosse voltar

~

227. No caminho, eu o encontrei

228. Encontrei-o na lagoa, o Jere'op Ambyjywyp

229. No caminho ele ficou doente

230. Eu tinha mandado o outro encontrá-lo

231. Chegaram de novo

232. Depois me levaram sozinho

233. Como este daqui

234. Eu fiquei realmente sozinho

235. Lá na casa dele

236. Eu fui de novo sozinho, assim

237. Eu não tinha nada

238. Do meu grupo só tinha eu

~

239. Andou, andou, andou. Disse o meu finado pai:

240. ``Não mexe com o pessoal, não mexe com as crianças'', disse meu finado pai

241. ``Depois, vocês vão sair'', disse o meu finado pai

242. Muitas mulheres, muitas crianças, havia lá

243. Já que ele falou comigo, ``eu estou indo lá'', eu disse

244. ``Vamos. Se você não me convidasse, eu não iria''

~

245. Andamos, andamos, continuamos só com ele

246. ``Eu vou levar seu filho'' --- mas se ele não tivesse falado com o meu
finado pai, eu não iria --- ``vou levar o seu filho, meu irmão''

247. Como ele me falou, eu fui

~

248. Andamos, andamos, continuamos

249. Andamos, andamos, chegamos

250. Eu entrei sozinho

251. ``Cheguei!''

252. ``Vem'', disseram os homens

253. Enquanto nós estávamos rindo, as mulheres disseram: ``O macaco-aranha gritou''

254. ``O macaco-aranha gritou, vão caçar! Mata macaco-aranha para nós!''

255. ``Ah! Dizem que o macaco-aranha gritou'', nós dissemos. ``Vamos atirar!''.

256. Fomos caçar.

257. Então fomos.

258. ``Aqui está a munição do seu tio'', eu disse

259. Pegaram umas cinco

260. Eu peguei e fui embora

~

261. Fomos, fomos, fomos

262. ``A anta vai ser minha. A anta vai ser minha, você vai ver''

263. ``Nós vamos matar anta!''

264. ``Vamos! Vamos matar antas, meu cunhado''

265. Assim disse o finado pai de Eremby Byyt

266. Fomos, andamos, andamos, ouvimos o barulho do macaco-aranha e do
macaco-prego

267. ``Aqui tem caça!''. Os macacos gritaram, ele correu na direção deles

268. Ele correu, parou. Ele atirou, acertou um

~

269. Então chegou o gavião \emph{pãram}\footnote{Espécie de gavião que
  sinaliza a presença de anta naquela localidade, nomeado ``gavião da
  anta''.}

270. Gritou, passou por cima do amigo dele, gritou, gritou

271. Chegou de novo outro gavião logo atrás desse

272. Gritou, gritou e a anta respondeu

273. ``\emph{Rinh}'' disse a anta

274. ``\emph{Rinh}'', enquanto ele estava lá

275. ``Ah, a anta! Eu vou matar esta anta!''

276. ``Eu não vou gritar não''

277. Eu corri

278. Corri, andei, andei, andei

279. Quando eu estava em pé, falaram:

280. ``Onde está, meu cunhado?''

281. ``Onde está, meu cunhado?''. ``Aqui tem anta'', eu disse

282. ``Vamos matá-la''

283. ``Vamos, sim!''

~

284. Andamos, andamos, olhamos, olhamos cuidadosamente. Havia vários jacamins

285. Os jacamins estavam parados e fazendo muito barulho

286. ``Parece que aqui tem jacamins''.

287. Cercaram ao redor deles, mas os jacamins não voaram

288. Andamos, andamos, andamos chegamos

289. Dando a volta na subida do tronco caído

290. Andei, andei, dei a volta por cima deles

291. Eu olhei para cá, por cima do tronco

292. Elas estavam lá

293. Eu olhei e as vi imediatamente

294. Abaixei os olhos e logo a vi, embaixo

295. ``Olha aqui''

296. Mirei a arma na direção da anta

297. ``Quem vai começar a atirar?''

298. ``Eu vou atirar primeiro'', eu falei

299. ``Se você não atirar, eu vou atirar, meu cunhado'', ele disse

300. ``Ah'', ele disse quando apertei o gatilho

301. Eu atirei, ele também, várias vezes

302. ``Matamos a anta!''

303. ``Vamos! Aqui também tem caça''

304. ``Vamos, meu cunhado'', eu disse

~

305. Andamos, andamos, chegamos no caminho

306. ``No que você atirou?'', disseram suas mulheres

307. ``Nós não matamos, minha tia'', eu disse

308. ``Nós colocamos os macacos-aranha para correr''

309. ``Acabou minha munição''

310. ``Eu estou envergonhado de não ter conseguido'', eu disse

311. ``Ah! Como você atirou, meu sobrinho, para onde?'', ela disse

312. ``Não matou mesmo?'', disse de novo para o marido dela

313. ``Nós matamos apenas tamanduá magro. Nós matamos''

~

314. Ah, eu não pensei que as mulheres estivessem assim tão agitadas

315. ``Ah! Vocês mataram anta, não mataram? Vocês mataram anta para vocês?''

316. ``Matamos mesmo, minha avó''

317. ``Tudo bem. Foi anta que vocês mataram.''

318. Meu pai foi atrás do irmão dele

319. ``O meu irmão estava lá, endireitando uma peça de madeira."

320. ``Chame-o!''. Foi lá com ele. Foi

321. Andou, chegou. ``Os homens mataram a anta'', ele disse

322. ``Ah! Mataram anta! Vamos! Vamos cortar a anta'', ele disse

323. Foram

324. ``Eu cheguei, você matou a anta''

325. ``Eu matei mesmo''

326. ``Lá só tinha anta magra''

327. ``Para mim não é ruim''

328. ``A anta está magra, mas está boa'', ele disse

~

329. Então eles vieram

330. Fomos, então foram todos, as mulheres limparam a casa

331. As mocinhas e as crianças também foram

332. ``Olha aqui, olha'', nós dissemos!

333. ``Ah! Vocês mataram a anta boa'', disseram os homens para nós

334. Ele a cortou, viu a gordura branca

335. ``Não há carne na nuca, mas há muita gordura. Só sai gordura''

336. Ele a repartiu, repartiu, e então acabou

337. Nós fomos embora e alcançamos o caminho

338. Nós assamos a anta, assamos, assamos, assamos, terminou

339. Os homens ficaram todos loucos

340. Os homens cantavam e enlouqueciam por causa da gordura da minha anta

341. Eu não participei, porque eu sou dono da anta

~

342. Dois dias se passaram

343. ``Daqui a três dias eu vou embora'', eu disse

344. ``Vai lá, você vai sozinho?

345. Vá, você vai sozinho''

346. ``Eu vou sozinho''

347. ``Vai atrás dele, vai'', falaram para o finado pai do Sebastião

348. ``Vai atrás dele'', falaram para o finado pai do Epitácio quando o
Pereira era pequeno

349. Nós viemos em três

350. Voltamos, chegamos na aldeia

~

351. Voltamos, eles tinham matado queixadas

352. Atiraram, flecharam, atiraram quando nós estávamos vindo para cá

353. Andamos, andamos, chegamos, deixamos a caça no chão

354. ``Cheguei, meu pai. Eu estou com a anta, que está muito gorda. Eu matei esta anta, meu pai''

355. ``É a parte de trás dela?''

356. ``Não é a parte de trás não, é que a costela da anta é gorda. Está gorda''

357. O meu finado pai pensou: ``Será que ele vai me dar a parte do umbigo?''

358. Nós chegamos

359. Os homens foram embora

~

360. Assim nós aparecemos

361. Quando estávamos prestes a sair, eles chegaram de novo sozinhos

362. No local onde nos pegaram, os homens se reuniram

363. Depois que eles foram, outros chegaram também

364. Os velhos ficaram lá mesmo

365. Um homem chegou e o outro presenteou a filha para ele

366. O meu finado pai pedia uma mulher

367. ``Consegue uma para mim também'', meu irmão disse

368. Assim é que foi

369. Nós fomos de novo

370. Chegamos de novo aqui

371. Então nós ficamos aqui com eles

372. Nós ficamos e não voltamos para lá

373. Acabou aqui mesmo

~

374. Nós viemos para cá, a outra parte voltou, mas não para lá

375. Não havia mais gente aqui, ficamos lá com eles

376. O finado pai desta aqui ficou lá

377. O meu finado pai ficou aqui sozinho

378. Pojepap ficou aqui sozinho

379. Nós todos ficamos lá

380. Nós ficamos sozinhos por muito tempo

381. Ficamos por muito tempo

~

382. Então ele {[}Moraes{]} morreu, a doença do branco o matou

383. Então meu finado pai se juntou

384. Quando o Moraes morreu, nós ficamos juntos

385. Nunca mais nos separamos

386. Assim nós ficamos, assim

387. Não havia mais homens quando ele morreu

388. Os homens ficaram com o meu pai

389. O meu finado pai trouxe todos para cá

390. Então o meu pai morreu aqui

391. Então eles não foram mais na aldeia deles

392. Aqui meu finado pai morreu

393. Assim o pessoal ficou junto

394. Não há mais o que contar, esse é o final



