\chapter{Como foi feito este livro}

\begin{flushright}
\textsc{íris morais araújo}\\
\textsc{karin vivanco}
\end{flushright}

\noindent{}Em 1992, a linguista Luciana Storto
iniciou sua pesquisa sobre a língua karitiana. Para poder estudar o
idioma, ela gravou, neste e nos cinco anos seguintes, histórias
tradicionais do povo Karitiana --- mitos de origem, rituais e narrativas
históricas. As histórias foram narradas por Pereira Karitiana, Barabadá
Karitiana, Garcia Karitiana, Antonio Paulo Karitiana, Cizino Karitiana,
Joana Karitiana e Nazaré Karitiana, alguns dos homens e mulheres mais
velhos de então, tidos como conhecedores da arte verbal. Com o
importante apoio de interlocutores indígenas mais jovens, os atuais
professores Nelson Karitiana, João Karitiana, Luiz Karitiana e Inácio
Karitiana, bem como de vários outros falantes da língua, foram feitas as
primeiras transcrições e traduções do material.

Os linguistas trabalham transcrevendo e traduzindo as narrativas
sentença a sentença. A passagem da fala para a escrita é o primeiro
desafio colocado, já que em qualquer língua existem diferenças entre
como se fala e como se escreve. Para chegar à escrita da fala, a maneira
escolhida pela linguista foi ouvir cada sentença conjuntamente com os
jovens karitiana com os quais trabalhou na transcrição e tradução,
pausar o áudio, e ir decidindo o que permaneceria no texto transcrito e
o que seria deixado de fora da transcrição. Este foi um modo de manter o
conteúdo da narrativa e sua estrutura prosódica e artística, sem incluir
os erros, hesitações e repetições não intencionais, naturais da fala
ocorridas enquanto o falante busca na memória pelo próximo assunto a ser
narrado.

\section{a tradução}

As frases incluíram mais de uma sentença quando pronunciadas com única entoação. Os linguistas fazem esses
registros com muitos detalhes. Para eles, é importante saber como
funciona cada parte de uma única palavra, chamada de \textit{morfema}, e
cada palavra em uma frase. Para que essas informações estejam
disponíveis para outros estudiosos, as sentenças são registradas em três
linhas: 

\begin{enumerate}
\item A linha do original na língua indígena, com um hífen
separando cada morfema dentro das palavras;\footnote{\textit{I-a-oky padni Gokyp}.}

\item A linha chamada
\textit{glosa}, na qual se faz uma tradução para o português do
significado de cada morfema;\footnote{\textit{Terceira pessoa-passiva-morrer não Sol}.}

\item A linha contendo tradução
aproximada da sentença inteira para o português; neste caso, o linguista
por vezes precisa fazer escolhas entre uma tradução literal ou mais natural da sentença.\footnote{\textit{Não se mata o sol/\,O Sol não pode ser morto}.}

\end{enumerate}

As transcrições apresentadas neste livro foram editadas para uma leitura
confortável, mas se buscou preservar certas características da narrativa
oral, como a repetição poética (repetição da sentença anterior com uma
modificação, a fim de criar um efeito poético na forma ou no
significado) e estruturas sintáticas comuns na língua karitiana: um
exemplo são as inversões na ordem de palavras.\footnote{Por exemplo, \textit{Caça, o
Osiip desnorteia} em vez de \textit{O Osiip desnorteia a caça.}}

\section{quem participou}

\paragraph{Barabadá Karitiana} foi um pajé Karitiana pertencente ao grupo Joari
 (também conhecido como Capivari). Ele narrou o encontro, vivenciado por
 ele, entre dois grupos de Karitiana que viviam em aldeias separadas, os
 Joari (ou Capivari) e os Karitiana. Estes grupos passaram a viver juntos
 na Terra Indígena Karitiana.

 \paragraph{Cizino Karitiana} é o atual pajé e cacique Karitiana. Ele narrou o ritual
 de iniciação masculina, intitulado \textit{Osiip}, pelo qual passou várias
 vezes.

 \paragraph{Garcia Karitiana} foi um cacique Karitiana. Ele narrou os mitos de origem
 do sol e da lua.

 \paragraph{Inácio Karitiana} é licenciado em Educação Básica Intercultural pela
 Universidade Federal de Rondônia e professor da Escola Indígena Estadual
 de Ensino Fundamental Kity Pypydnipa.

 \paragraph{João Karitiana} é licenciado em Educação Básica Intercultural pela
 Universidade Federal de Rondônia e professor da Escola Indígena Estadual
 de Ensino Fundamental e Médio Kyõwã.

 \paragraph{Luiz Karitiana} é licenciado em Educação Básica Intercultural pela
 Universidade Federal de Rondônia e professor da Escola Indígena Estadual
 de Ensino Fundamental e Médio Kyõwã.

 \paragraph{Nelson Karitiana} é licenciado em Educação Básica Intercultural pela
 Universidade Federal de Rondônia e professor da Escola Indígena Estadual
 de Ensino Fundamental e Médio Kyõwã.

 \paragraph{Valdomiro Karitiana} é filho de Barabadá Karitiana. Ele acompanhou a
 linguista durante a gravação da história do encontro entre os dois
 grupos locais e auxiliou na transcrição e tradução.

 \paragraph{Ivan Rocha} é doutor em Linguística pela Universidade de São Paulo e
 pesquisador visitante do Museu Paraense Emílio Goeldi, com bolsa do
 Programa de Capacitação Institucional do Ministério da Ciência,
 Tecnologia e Inovação.\looseness=-1 % (\textsc{mcti}/\textsc{cnp}q).

 \paragraph{Íris Morais Araújo} é doutora em Antropologia Social pela Universidade de São Paulo e professora substituta da Universidade Federal do Tocantins.% (Campus Porto Nacional).\looseness=-1

 \paragraph{Karin Vivanco} é doutora em Linguística pela Universidade de São Paulo e professora da Universidade Federal do Rio Grande do Sul.


% Nem todas as narrativas gravadas por Luciana na década de 1990 chegaram
% a ser transcritas e traduzidas, mas todas as que o foram e que não fazem
% parte deste volume serão publicadas futuramente em outro volume desta
% coleção.
