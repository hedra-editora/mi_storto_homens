\chapter{Como foi feito este livro}

\begin{flushright}
\textsc{íris morais araújo}\\
\textsc{karin vivanco}
\end{flushright}

\noindent{}\textls[-15]{Em 1992, a linguista Luciana Storto iniciou sua pesquisa sobre a língua karitiana. Para poder estudar o idioma, ela gravou, neste e nos cinco anos seguintes, histórias tradicionais do povo Karitiana --- mitos de origem, rituais e narrativas históricas. As histórias foram narradas por Pereira Karitiana, Barabadá Karitiana, Garcia Karitiana, Antonio Paulo Karitiana, Cizino Karitiana, Joana Karitiana e Nazaré Karitiana, alguns dos homens e mulheres mais velhos de então, tidos como conhecedores da arte verbal. Com o importante apoio de interlocutores indígenas mais jovens, os atuais professores Nelson Karitiana, João Karitiana, Luiz Karitiana e Inácio Karitiana, bem como de vários outros falantes da língua, foram feitas as primeiras transcrições e traduções do material.}

\textls[15]{Os linguistas trabalham transcrevendo e traduzindo as narrativas sentença a sentença. A passagem da fala para a escrita é o primeiro desafio colocado, já que em qualquer língua existem diferenças entre como se fala e como se escreve. Para chegar à escrita da fala, a maneira escolhida pela linguista foi ouvir cada sentença conjuntamente com os jovens karitiana com os quais trabalhou na transcrição e tradução, pausar o áudio, e ir decidindo o que permaneceria no texto transcrito e o que seria deixado de fora da transcrição. Este foi um modo de manter o conteúdo da narrativa e sua estrutura prosódica e artística, sem incluir os erros, hesitações e repetições não intencionais, naturais da fala ocorridas enquanto o falante busca na memória pelo próximo assunto a ser narrado.}

\section{a tradução}

\textls[-20]{As frases incluíram mais de uma sentença quando pronunciadas com única entoação. Os linguistas fazem esses registros com muitos detalhes. Para eles, é importante saber como funciona cada parte de uma única palavra, chamada de \textit{morfema}, e cada palavra em uma frase. Para que essas informações estejam disponíveis para outros estudiosos, as sentenças são registradas em três linhas:}

\begin{enumerate}
\item A linha do original na língua indígena, com um hífen separando cada morfema dentro das palavras como, por exemplo, \textit{I-a-oky padni Gokyp};
\item A linha chamada \textit{glosa}, na qual se faz uma tradução para o português do significado de cada morfema como, por exemplo, \textit{Terceira pessoa-passiva-morrer não Sol};
\item \textls[15]{A linha contendo tradução aproximada da sentença inteira para o português; neste caso, o linguista por vezes precisa fazer escolhas entre uma tradução literal ou mais natural da sentença como, por exemplo, \textit{Não se mata o sol; O Sol não pode ser morto}. }
\end{enumerate}

\textls[15]{As transcrições apresentadas neste livro foram editadas para uma leitura confortável, mas se buscou preservar certas características da narrativa oral, como a repetição poética (repetição da sentença anterior com uma modificação, a fim de criar um efeito poético na forma ou no significado) e estruturas sintáticas comuns na língua karitiana: um exemplo são as inversões na ordem de palavras. Por exemplo, \textit{Caça, o Osiip desnorteia} em vez de \textit{O Osiip desnorteia a caça}.}

\section{quem participou do livro\protect\footnotemark}
\footnotetext{\textls[-10]{Além dos narradores e organizadora, apresentados na página 4 deste volume.}}

\paragraph{Inácio Karitiana} \textls[-10]{é licenciado em Educação Básica Intercultural pela Universidade Federal de Rondônia e professor da Escola Indígena Estadual de Ensino Fundamental Kity Pypydnipa.}
\vspace{-0.2cm}

\paragraph{João Karitiana} é licenciado em Educação Básica Intercultural pela Universidade Federal de Rondônia e professor da Escola Indígena Estadual de Ensino Fundamental e Médio Kyõwã.
\vspace{-0.2cm}

\paragraph{Luiz Karitiana} é licenciado em Educação Básica Intercultural pela Universidade Federal de Rondônia e professor da Escola Indígena Estadual de Ensino Fundamental e Médio Kyõwã.
\vspace{-0.2cm}

\paragraph{Nelson Karitiana} é licenciado em Educação Básica Intercultural pela Universidade Federal de Rondônia e professor da Escola Indígena Estadual de Ensino Fundamental e Médio Kyõwã.
\vspace{-0.2cm}

\paragraph{Valdomiro Karitiana} \textls[-25]{é filho de Barabadá Karitiana. Ele acompanhou a linguista durante a gravação da história do encontro entre os dois grupos locais e auxiliou na transcrição e tradução.} 
\vspace{-0.2cm}

\paragraph{Ivan Rocha} é doutor em Linguística pela Universidade de São Paulo e pesquisador visitante do Museu Paraense Emílio Goeldi, com bolsa do Programa de Capacitação Institucional do Ministério da Ciência, Tecnologia e Inovação. 
% (\textsc{mcti}\quad/\textsc{cnp}\quadq). 
\vspace{-0.2cm}

\paragraph{Íris Morais Araújo} é doutora em Antropologia Social pela Universidade de São Paulo e professora substituta da Universidade Federal do Tocantins.% (Campus Porto Nacional). 
\vspace{-0.2cm}

\paragraph{Karin Vivanco} é doutora em Linguística pela Universidade de São Paulo e professora da Universidade Federal do Rio Grande do Sul.

% Nem todas as narrativas gravadas por Luciana na década de 1990 chegaram a ser transcritas e traduzidas, mas todas as que o foram e que não fazem parte deste volume serão publicadas futuramente em outro volume desta coleção.