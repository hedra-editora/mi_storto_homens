\part{Não havia mais homens}

\chapter*{}
\thispagestyle{empty}

\vspace*{\fill}
\paragraph{A história de Gokyp, o Sol} A história do Sol é uma das narrativas que compõem o repertório de mitos dos Karitiana. Nascido como uma criança muito quente, que parecia febril, ninguém conseguia chegar muito perto do Sol sem se queimar. As
pessoas, preocupadas com o perigo que ele significava para a comunidade,
pensaram em matá-lo. Cada vez mais quente, o Sol subiu pelo esteio de
uma casa e decidiu ir para o céu, onde vive até hoje.\footnote{A narrativa aqui publicada foi contada por Garcia Karitiana para Luciana Storto, que a transcreveu e traduziu com Nelson Karitiana. Para esta
publicação, o material foi editado por Íris Morais Araújo e Karin
Vivanco. Uma primeira transcrição, glosagem e tradução desta narrativa foi publicada por Storto na \textit{Revista Linguíʃtica} 15 (2019).}
\vspace*{\fill}

\chapter{O Sol}
%\begingroup\parindent=0em

\begin{enumerate}
\item Dizem que o Sol vivia antigamente
\item Dizem que o Sol começou sua existência como uma criança
\item Dizem que o Sol vivia

\begin{center}\adforn{68}\end{center}

\item Então os homens disseram \textit{O que é isso?}
\item Dizem que a vida começou como uma doença para ele
\item Dizem que a criança ficava cada vez mais quente

\begin{center}\adforn{68}\end{center}

\item \textit{Esse aí está doente?}, falava o seu pessoal
\item Seria semelhante a uma doença
\item Mas ele não estava doente realmente
\item Ele nunca esteve doente

\begin{center}\adforn{68}\end{center}

\item Então dizem que o calor dele ficava cada vez mais intenso
\item O Sol era meio quente e foi ficando mais quente
\item Naquele momento, ele se tornaria o Sol

\begin{center}\adforn{68}\end{center}

\item Então aconteceu
\item \textit{Ah, o que é isso?}, diziam os homens
\item Aí, ele não existia mais
\item Então ele se tornou tão grande que não podia mais viver aqui
\item Ele se tornou enorme

\begin{center}\adforn{68}\end{center}

\item Aí, dizem que o Sol não queimava mais só um pouco
\item Dizem que ele se tornou incandescente
\item Quando sua incandescência ficou insustentável, dizem que os homens
queriam matá-lo
\item Porque ele não era mais um ser humano

\begin{center}\adforn{68}\end{center}

\item Os homens tinham um desejo de matar
\item Então não o fariam
\item Aí, os homens pensaram
\item \textit{O Sol não pode ser morto}

\begin{center}\adforn{68}\end{center}

\item Eles pensaram que era assim que fariam
\item Eles pensavam que ele era humano
\item Mas ele estava ficando diferente, meio fraco

\begin{center}\adforn{68}\end{center}

\item Por causa disso, os homens sentiam vontade de matá-lo
\item Matá-lo é o que queriam fazer
\item Aí, ele continuou a viver

\begin{center}\adforn{68}\end{center}

\item Dizem que ele saiu pelo esteio central do telhado da casa
\item Porque dizem que os homens tinham o desejo de matá-lo
\item Sabia-se que queriam matá-lo

\begin{center}\adforn{68}\end{center}

\item Na frente deles, então, ele saiu
\item Pelo esteio central da casa ele sairia
\item Aí, vieram os homens com bordunas para matá-lo, sorrateiramente
\item Mas eles não podiam mais se aproximar dele
\item Ele estava muito incandescente mesmo

\begin{center}\adforn{68}\end{center}

\item Foi então que, dizem, ele saiu pelo esteio central da casa
\item Aí, o Sol cantou antes de sair
%\begin{quote}
\item \textit{Tragam-me para dentro, mulheres}
\item \textit{Tragam-me para dentro, mulheres}
\item \textit{Tragam-me para dentro, tragam-me para dentro, mulheres}
\item \textit{Pois o crânio partido está me matando}
%\end{quote}
\item Assim disse o Sol enquanto ele subia
\item Então dizem que o Sol subiu; \textit{Pro alto} ele foi
\item Então dizem que quem queria matá-lo morreu enquanto ele estava subindo
\item \textit{Caíram esparramados}, os homens

\begin{center}\adforn{68}\end{center}

\item O Sol subiu para as alturas
\item Então o Sol ficou incandescente, incandescente de verdade
\item Assim, dizem, é que a história do Sol deve ser contada
\end{enumerate}

\chapter{Gokyp}

\begin{enumerate}
\item Pyry'a sarytyn keerep Gokyp
\item Õwã horot taka'oot saryt Gokyp
\item Taaka andyk saryt Gokyp

\begin{center}\adforn{68}\end{center}
\item Masõng \textit{ti'a hỹ?}, iri'aj taso
\item Kinda oti horot taka'oot saryt ihot iaka
\item Okywyra okywyra okywyra taaka saryt õwã

\begin{center}\adforn{68}\end{center}
\item \textit{A kinda otidna hỹ?}, iri'a andyki ijiriso
\item Kinda oti horot iakaj
\item Ikinda otidni
\item Takinda otidna sogng iaki

\begin{center}\adforn{68}\end{center}

\item Masong naakat okyp okyp ywytiyty tat, iri'aj
\item Ty'in taakat iokyp pywytiyri Gokyp
\item Gokyp pasagngam iakabman

\begin{center}\adforn{68}\end{center}

\item Masong naka'a andyk
\item \textit{Ãh ti'a tyka hỹ?}, iri'aj taso
\item Masong imbodnoko
\item Atykiri iaka padnoko hak tatyyt tykiri
\item Tyyty tat, iri'aj

\begin{center}\adforn{68}\end{center}

\item Atykiri ipikyp owogoko saryty padni Gokyp
\item Napikybm saryt
\item I pikywyt tykiri, iatakipawyt tykiri napyting saryt iokyty taso
\item Masong iaki padnoko

\begin{center}\adforn{68}\end{center}

\item Napymbowak ity taso
\item Masong iaoky padnaty
\item Masong nakakãrãt taso
\item Iaoky padni Gokyp

\begin{center}\adforn{68}\end{center}

\item Masong kahyt i'at, irikãraj͂
\item Yjxa pitat i'at tykat, irikãrãj
\item Myn hodno yjsararaj sat iakiip

\begin{center}\adforn{68}\end{center}

\item Masong napymbowak saryt ity taso
\item Iokyty napyting
\item Masong ta'a tyka siit

\begin{center}\adforn{68}\end{center}

\item Namboryt saryt ambiity sopakat
\item Masong napymbowak saryt it taso
\item Ta'ãty ipymbowak tyso

\begin{center}\adforn{68}\end{center}

\item Asonderep masong namboryt
\item Ambi sopakat imboryri
\item Masong naymbykyjy'oom taso iokyp meresõ tyyt
\item Ipynotam padnoko I
\item Piikyp piikyp harara I

\begin{center}\adforn{68}\end{center}

\item Masong namboryt saryt i ambi sopakat
\item Masong nakahyryj͂a tatat tysypy'oot Gokyp
%\begin{quote}
\item \textit{Ajymymewã j͂onso}
\item \textit{Ajymymewã j͂onso}
\item \textit{Ajymymewã ymymewã j͂onso}
\item \textit{Opa pyka yoky tyki}
%\end{quote}
\item Masong naka'a saryt Gokyp taambo tyki'oot
\item Masong naambo saryt Gokyp; atoop iri'aj hoop
\item Masong ta'ãty ipymbowak atapopi saryt tatat tysypy'oot
\item \textit{Syyryp} iri'aj taso

\begin{center}\adforn{68}\end{center}

\item Naambot ohyn Gokyp
\item Masong naakat piikyp piikyp harara Gokyp
\item Naka'a saryt kahyt Gokyp pynhadna
\end{enumerate}

%\endgroup

%\chapter{A história do Lua, Oti}

\chapter*{}
\thispagestyle{empty}

\vspace*{\fill}
\paragraph{A história do Lua, Oti} A história do Lua, um homem chamado Oti, é uma importante narrativa que também compõe o \textit{corpus} mítico dos Karitiana. Quando vivia entre
os humanos, Oti teve relações sexuais não consentidas com a mãe e a
irmã. Esta última descobriu que o irmão a visitava à noite, no escuro,
após seguir a sugestão de seu namorado de passar tinta de jenipapo no
intruso durante uma dessas relações sexuais. A mãe, por sua vez, foi
forçada por Lua a ceder a seus desejos quando foram para a floresta
buscar os frutos da palmeira patauá.\footnote{\textit{Oenocarpus bataua}, também 
conhecida por patuá ou patoá.} Após esses eventos, Lua subiu
nessa palmeira e foi viver no céu. Contudo, antes disso, cortou suas
próprias pernas, para que fossem sepultadas junto com seus objetos e
decidiu que todas as mulheres, a partir de então, passariam a menstruar.\footnote{A narrativa de Oti foi contada por Garcia Karitiana para Luciana Storto,
que a transcreveu e fez uma tradução preliminar. Para esta publicação, o
material foi traduzido e revisado por Inácio Karitiana, Nelson
Karitiana, Íris Morais Araújo, Karin Vivanco e Luciana Storto.}
\vspace*{\fill}


\chapter*{O Lua}
%\begingroup\parindent=0em

\begin{enumerate}
\item Assim, dizem, é que fez o Lua
\item O Lua vivia
\item O Lua era homem, também
\item Homem adulto

\begin{center}\adforn{68}\end{center}

\item Então dizem que o Lua ainda vivia entre nós
\item Ainda estava vivo
\item Dizem que o Lua transou com sua irmã menor
\item Aí, ela pensou que ele era seu namorado
\item E ela permitiu que o Lua transasse com ela, a irmã do Lua

\begin{center}\adforn{68}\end{center}

\item Então dizem que o namorado dela mexeu com ela
\item Transou com ela
\item \textit{Ah! Quem será?}
\item Ela disse assim
\item Aí, o namorado veio na direção dela
\item \textit{Vem}, disse o namorado dela
\item \textit{Você veio de novo?}, ela disse

\begin{center}\adforn{68}\end{center}

\item Ah! O seu namorado falou: \textit{O que é isso?}
\item \textit{Ah! Não era você que veio antes?}, ela disse
\item \textit{Eu ainda não vim com você}. \textit{Por quê?}, disse
\item \textit{Ele mexeu várias vezes comigo. Quantas vezes você já veio aqui comigo?}
\item \textit{Só agora mesmo eu vim aqui com você}
\item Disse o namorado dela

\begin{center}\adforn{68}\end{center}

\item \textit{Não é você que vem transando comigo?}
\item \textit{Qual homem será?}
\item \textit{Qual homem será que vem transando comigo?}
\item \textit{Ah!}, falou o namorado
\item \textit{Eu achei que fosse você}, disse a irmã do Lua
\item Então foi assim

\begin{center}\adforn{68}\end{center}

\item Aí o namorado disse \textit{Hum\ldots{}}
\item \textit{Passa jenipapo no corpo dele}, o namorado falou
\item \textit{Você vai passar jenipapo amanhã}
\item \textit{Amanhã você passa jenipapo no rosto dele e depois disso você vai
\item rosto dele}
\item \textit{Você olha o rosto de todos os homens}
\item \textit{Então você vai ver}
\item \textit{É você mesmo?}
\item \textit{Então você vai falar}
\item \textit{Quando você souber, você vai falar assim}
\item Então ela nunca pensou que seria seu irmão
\item \textit{Hã}, disse
\item \textit{Você passa jenipapo sorrateiramente em você}, disse
\item \textit{Você vai esconder o jenipapo embaixo da rede}, disse
\item \textit{Você vai deixar o jenipapo embaixo para quando ele vier}
\item \textit{Hoje ele vem de novo}
\item \textit{Ele não demora a vir de novo}
\item \textit{Você não vai se atrapalhar}

\begin{center}\adforn{68}\end{center}

\item Então ela ficou preparada
\item Ela se preparou para a chegada dele
\item Passou jenipapo nela, ficou preta
\item Aí sobrou
\item Então dormiu
\item A mulher dormiu com todos enfeites e pinturas
\item A mulher estava bonita

\begin{center}\adforn{68}\end{center}

\item Então no mesmo dia o Lua, o irmão dela, chegou de novo
\item Era o irmão dela
\item Ela não sabia
\item Dizem que o Lua transava com a irmã. Ela foi a primeira filha da mãe dele, foi a primeira dela
\item Então foi dormir

\begin{center}\adforn{68}\end{center}

\item Aí, escureceu o dia
\item Antes de escurecer, ele viria
\item Veio
\item Então ela empurrou o jenipapo embaixo da rede dela
\item Ela deixou o suco de jenipapo preparado
\item Então dizem que o Lua foi na direção dela
\item Aí veio de novo e mexeu

\begin{center}\adforn{68}\end{center}

\item \textit{Ah! Você chegou?}, ela disse
\item \textit{Cheguei}, disse
\item \textit{Você chegou}, disse
\item \textit{Eu vim pra você}, disse
\item \textit{Por que você sempre vem aqui comigo? É você mesmo?}, disse
\item \textit{Sou eu mesmo}, disse ele
\item \textit{Calma}, disse
\item \textit{O que é isso?}, disse o Lua.

\begin{center}\adforn{68}\end{center}

\item Então dizem que pegou nela
\item Pegou várias vezes. Enquanto pegava, ela pôs a mão no jenipapo
\item O sumo do jenipapo por cima
\item Passou várias vezes
\item Ela passou o sumo do jenipapo nele
\item Muito, ela passou
\item Então dizem que passou em cima do rosto dele; ela passou
\item Passou em cima do rosto dele
\item \textit{Ah!}, disse, enquanto estava mexendo com ela
\item Então com o sumo do jenipapo preto
\item Passou o sumo do jenipapo no rosto dele
\item Dizem que foi assim
\item Depois que transou com a irmã
\item Assim foi

\begin{center}\adforn{68}\end{center}

\item Então amanheceu
\item No dia seguinte, ela, a irmã, o procurou
\item O jenipapo tinha sujado o rosto dele
\item Ficou preto
\item Então dizem que ele lavou
\item Lavou várias vezes. Ele ficou lavando o rosto por muito tempo
\item Mas o jenipapo não saiu do rosto dele
\item Ficou preto
\item Manchou

\begin{center}\adforn{68}\end{center}

\item Aí, desconfiada, a irmã dele foi ver os homens
\item Olhou, olhou, olhou os homens
\item Nisso, dizem que saiu
\item Com vontade de revelar
\item Também dizem que o Lua estava tirando o jenipapo

\begin{center}\adforn{68}\end{center}

\item \textit{Ah!}, disse
\item \textit{Você estava premeditando para cima de mim, meu irmão!}, ela disse
\item \textit{Você esperou por mim dormindo, meu irmão}, ela disse
\item \textit{Eu dormindo e você esperou}, ela disse
\item \textit{Ah!}, dizem que ela fez assim
\item \textit{Descobri}, ela disse
\item Disse assim, sozinha, a irmã do Lua

\begin{center}\adforn{68}\end{center}

\item Então foi assim
\item E ele fez tudo de novo
\item Ele continuou fazendo
\item Ficou fazendo

\begin{center}\adforn{68}\end{center}

\item Aí, dizem que ele falou com sua mãe, sorrateiro
\item \textit{Eu quero patuá, minha mãe}, ele disse
\item \textit{Ah! Nós vamos pegar, meu filho}, disse a mãe
\item \textit{Vamos providenciar, meu filho}.
\item \textit{Tem patuá aqui, minha mãe}. \textit{Vamos pegar}, disse
\item Então dizem que ele andou bastante com a mãe dele, enganando-a
\item Fez ela andar
\item Então diz que ele andou com a mãe dele
\item Andou, andou, andou. \textit{Chegamos}, disse.
\item \textit{Lá está o patuá}, disse
\item \textit{Tem aquele patuá lá}, disse
\item Então esperou sua mãe

\begin{center}\adforn{68}\end{center}

\item \textit{Minha mãe}, disse
\item \textit{Tem muito micuim em mim, minha mãe}, disse ele
\item \textit{Estou com micuim, minha mãe}
\item \textit{Estou com carrapatos, minha mãe}, disse ele
\item \textit{Você quer que eu cate, meu filho?}, disse a mãe dele
\item Nisso, ela começou a catá-los

\begin{center}\adforn{68}\end{center}

\item A mãe dele era inocente
\item \textit{Então vou fazer assim}, ele pensou
\item \textit{Ele deve querer fazer alguma coisa ruim comigo}, a mãe dele pensou
\item Então catou micuim enquanto estavam no mato
\item Então dizem que catou muito micuim
\item Dizem que catou, catou e o Lua teve uma ereção
\item A mãe dele ficou constrangida
\item Sem graça, ela ficou

\begin{center}\adforn{68}\end{center}

\item Então dizem que derrubou a sua mãe à força
\item Ele a pegou à força e ficou em cima dela
\item \textit{Ah! Você está fazendo uma coisa ruim, meu filho!}, disse a mãe dele
\item \textit{Você está fazendo uma coisa que você realmente não deveria fazer, meu filho}, disse
\item \textit{Não existe mais}, ela disse
\item \textit{Nós nos deitamos}, ele disse
\item \textit{Você se lembra}, disse sua mãe
\item Dizem que ele tinha ouvido sua mãe
\item \textit{Solta, solta}, disse
\item \textit{Das coisas que você realmente não deveria ter feito, você lembra}, disse sua mãe
\item \textit{Você me enganou}, ela disse
\item \textit{Estou indo dormir}, ela disse

\begin{center}\adforn{68}\end{center}

\item \textit{Então espere aí, eu vou pegar}, ele disse
\item \textit{Eu vou pegar patuá}, disse
\item \textit{Eu vou pegar o patuá}
\item \textit{Tira o patuá, tira!}, disse a mãe dele
\item A mãe ficou com raiva dele
\item A mãe ficou com uma enorme raiva dele
\item O que ele fez não se pode consertar nunca

\begin{center}\adforn{68}\end{center}

\item Então assim vive o Lua
\item Até hoje o Lua não conhece as nossas regras
\item Foi assim que surgiu o Lua
\item Assim é que foi
\item Para o mundo ser assim ele fez isso

\begin{center}\adforn{68}\end{center}

\item Então subiu para o alto, andando
\item Então subiu no alto da palmeira do patuá
\item Então ele subiu, subiu até o caule do patuá
\item \textit{Aqui tem patuá, minha mãe}, disse fingindo
\item \textit{Você pega patuá, mãe}, disse
\item A mãe não falou mais nada depois que ele transou com ela
\item Sua mãe ficou brava

\begin{center}\adforn{68}\end{center}

\item Então cortou, fingindo
\item Cortou, cortou. E o cacho de patuá caiu
\item Assim fez
\item Depois ele subiu na copa do patuá
\item Então dizem que ele subiu para o alto
\item Puxou o olho do patuá para subir

\begin{center}\adforn{68}\end{center}

\item Aí ele disse: \textit{Lá vai minha coxa, mãe!}
\item \textit{Lá vai minha coxa, minha mãe!}
\item Ah! A mãe nunca pensou que ele faria isso
\item A mãe pensava que ele estava brincando
\item Então, dizem que cortou a coxa aqui
\item Serrou, serrou, serrou suas coxas
\item Então dizem que jogou as pernas; caíram
\item Ele soltou as pernas e elas caíram
\item As pernas dele caíram
\item \textit{Você fez uma coisa que realmente não deveria ter feito}, disse a mãe dele
\item Aí disse: \textit{Ah!}

\begin{center}\adforn{68}\end{center}

\item Então dizem que o Lua subiu para o alto
\item \textit{Eu vou}, ele disse
\item \textit{Com isso, você vai juntar minhas coisas, mãe}
\item Então dizem que o Lua foi para o alto
\item Ele foi
\item Subiu dentro do patuá
\item Dentro do patuá, ele subiu
\item Então ele puxou o olho do patuá
\item Quando o Lua puxou olho do patuá, houve um estrondo
\item Parece que arrebentou o olho do patuá
\item Quando ele ia embora

\begin{center}\adforn{68}\end{center}

\item Então o Lua foi embora
\item Então dizem que o Lua não viveu mais
\item Dizem que foi
\item Assim o Lua ficou lá em cima
\item É assim até hoje
\item Por isso o Lua está como é agora

\begin{center}\adforn{68}\end{center}

\item Então por isso as mulheres todas vivem assim
\item Até hoje fez as mulheres viverem assim
\item É por isso que surgiu menstruação
\item Foi assim que começou a menstruação das mulheres
\item Não foi a vontade do Lua, mas sim de Botyj̃
\item Mulher não pode viver sem menstruação
\item A mulher vive como tem que viver
\item É por isso que Botyj̃ fez isso
\item É por isso que o Lua ficou assim
\item Para nós vivermos, o Lua fez isso
\end{enumerate}

% \chapter*{Oti}

% 1. Naka'a saryt Oti

% 2. Taaka andyk saryt Oti

% 3. Taso tyym naakat Otit

% 4. Taso sota

% ~

% 5. Masong naaka andyk saryt Oti tyym

% 6. Naaka andyk

% 7. Masong tapan'in ataso'y saryt Oti

% 8. Masong taoj̃ombakap akat takãrãt

% 9. Tampyso saryt Oti, Oti pan'in

% ~

% 10. Masong taoj̃ombakap akat takãrãt napyso pysodn andyk saryt

% 11. Tik tik, iri'a andyki isok

% 12. \textit{Ãh! Morã iaka akadna hỹ?}

% 13. Masong naka'at

% 14. Masong nayryt ioj̃ombakap pita ikyn

% 15. \textit{Yrydn}, iri’aj ioj̃ombakap

% 16. ``Ãh! ayryt oko hỹ?'', iri’aj

% ~

% 17.Ãh! Iri’aj ioj̃ombaka pita, ``ti’a hỹ?''

% 18. ``Ãh! An aka mini iyryt ykyn?'', iri'aj

% 19. ``Ãh! Yryty padni yn akyn yn''. ``Ti’a hỹ?'', iri’aj

% 20. ``Pyso pyso ka'at ysok. Tikat ayryt ahop aka ykyn?''

% 21. ``Ho y’asot myrỹ’int ytayryt yn akyn yn''

% 22. Iri’aj ioj̃ombakap

% ~

% 23. ``Ãh!'', iri'aj. ``An aka mini ysok ipyso tykat?'', iri'aj

% 24. ``Mõrã taso akamon hỹ?'', iri’aj

% 25. ``Morã taso akamon? Ysok ipysok pysok tykadn?''

% 26. ``Ãh!'', iri’aj ioj̃ombakap

% 27. ``An akat ytakakãrãt yn'' iri'aj Oti pan'in, iri'aj

% 28. Masong naaka andyk

% ~

% 29. Masong ``myna'' iri’aj ioj̃ombakap

% 30. ``Im'y kinda pasojo isok'', iri'aj

% 31. ``Dibm kinda pasojo an nam'y tykiit''

% 32. ``Dibm kinda pasoj, pymbak iasop a, ambyygn atasombaki dibm iasooty''

% 33. ``Taso'oot atasombaki taso asooty''

% 34. ``Masong ataso'oori''

% 35. ``An nakahygng my'an!''

% 36. ``Masong ataka'aj''

% 37. ``Ity asondyp tykiri kahyt ataka'aj''

% 38. Masong tasyky akakit taso’ootot irikãraj̃

% 39. ``Hỹ!'', iri’aj

% 40. ``Im'y'oma kinda pasojo asok'', iri'aj

% 41. ``Kinda pasojot a'atidnan atakasywi'', iri'aj

% 42. ``Kinda pasojot a’atidnan atakasywi iaj̃ot''

% 43. ``Kiit tayryrydnaj'', iri'aj

% 44. ``Iyryt pahoto padni'', iri'aj

% 45. ``Ogngom'oman atakasywi'', iri'aj

% ~

% 46. Masong kasypy 'om saryt mynat

% 47. Iyrytyt sokynyn irisywi

% 48. Masong kam’yt kinda pasoj tasok j̃ong j̃ong eem tat, iri’aj

% 49. Masong pi'idna tat iri'aj

% 50. Masong nakakat

% 51. Masong nakakat ej̃ep hãraj̃ ’oman iakajt j̃onso tao’it tykiri

% 52. Se’a ’omant iakaj j̃onso

% ~

% 53. Masong nayryt oko'oom, kiit ikyn Oti, isyky

% 54. Isyky akabm my'an

% 55. Isondyp andyky 'i

% 56. Tapan 'in pita ataso'y saryt Oti, tati'et, iri'a oori 'aty

% 57. Masong naakat tẽẽ, iri’aj

% ~

% 58. Tẽẽ, moj̃ iri’aj go

% 59. Moj̃ hã hã hã tee, kiit pymyrat iyryri i

% 60. Nayryt

% 61. Masong taeremby opi atip jykyt iri'aj kinda pasojoty

% 62. Tapymbangã pydn tyym i taj̃oj̃ kinda pasojo se

% 63. Masong nayryt saryt Oti ikyynt

% 64. Nayryt okotyn, pymbak iri'aj ipyp

% ~

% 65. ``Ãh! Ayryt hỹ?'', iri’aj i

% 66. ``Yryt yry'', iri'aj

% 67. ``Apyryryt my'anan'', iri'aj

% 68. ``Akyyn ytayryt yn'', iri'aj

% 69. ``Masong yryt pa’in pitat masong aka hỹ? An pita mon jo hỹ?'', iri’aj

% 70. ``Yn naakat'', iri’a omaj̃ i

% 71. ``Jo’a, ko’ãj̃ty'', iri’aj

% 72. ``Masong ti’ahỹ?'', iri’a’om andyki Oti

% ~

% 73. Masong napyso saryt isok tik, iri'aj

% 74. Napysot isok, tasok ipyso tysypy'oot pymbak

% 75. Kinda pasojo se okyp

% 76. Pymbak pymbak ko’ãj̃ty

% 77. Apip pymbak kinda pasojo sety

% 78. Pitat, iri'aj

% 79. Masong napymbak saryt iaso okyp; pymbak, iri'aj

% 80. Pymbak iaso okyp taambyyk

% 81. ``Ãh!'', iri'aj. Atykiri napyso andyk isok pysodn, iri'aj

% 82. Masong kinda pasojo se eem tyyt

% 83. Ihõroni padnoko iasop tapymbagng tykiri kinda pasojo se

% 84. Atykiri nakatata'om andyk saryt

% 85. Tapan'in so'y byyk

% 86. Masong nakatat

% ~

% 87. Haabm iri'aj go

% 88. Atykiri napikarant i dibm, i pan'in

% 89. Masong nakaeem kinda pasoj iasop

% 90. Eem tat, iri'aj

% 91. Masong nakamhoron'om andyk saryt i

% 92. Horon horon horon i’a’omaj̃

% 93. Ihorodni padni iasop kinda pasoj

% 94. Pyry'eemen

% 95. Masong eem iriakaj

% ~

% 96. Masong napimboop, taso pojongot taso’ootot irikãraj̃ ipan’in

% 97. Sombakat sombakatat sombakatat, iri'aj tasoty

% 98. Apip namboryt saryt i

% 99. Atop iri'iwak

% 100. Atyym pyry'a tykadn kinda pasoj pyrorat tykadn sarytyn Oti

% ~

% 101. ``Ãh!'', iri'aj.

% 102. ``An nakahyk my'an kat yjxa yi´a tykat ysyky!'', iri'aj

% 103. ``An nakahyk my'an kat yjxaty i´a tyka kat ysyky'', iri'aj

% 104. ``Kat atakahyk my'an an'', iri'aj

% 105. ``Ãh!'', iri'aj kahyt iri'aj saryri

% 106. ``Naka'oot'', iri'aj

% 107. Iri'aj saryri kahyt Oti tapan'in tyyt iaka

% 108. Naakat andyk kahyt

% 109. Masong naka'a okotyn

% 110. Naka'a okotyn mynhodnop i aka

% 111. Masong naaka andyk

% 112. Masong nakahadna'om andyk saryt tati tyyt

% 113. ``Ewyty ytasiki'y yti'', iri'aj

% 114. ``Ãh! Yjso'oot y'et'', iri'aj iti

% 115. ``Yjso'oot y'et yjxa i'ot ewy'', iri'aj iti

% 116. ``Ewy kasot hak yti''.  ``Yjso'oot yti'', iri'aj

% 117. Atykiri kahoto'om saryt tati tyyt

% 118. Tamtarakat iritari

% 119. Masong kahot saryt tati tyyt

% 120. Terek terek terek, tong, iri'aj

% 121. ``Hodn naakat ho ewy yti'', iri'aj

% 122. ``Ewy kat ho'', iri'aj

% 123. Masong naso'akyn tatity

% ~

% 124. ``Yti'', iri'aj

% 125. ``Syyt ako naakat ysok yti'', iri'aj

% 126. ``Hirã andyka ysyytyt yti''

% 127. ``Ororojoty yti'', iri'aj

% 128. ``Masong napimbopo’om, hỹ y’et?'', iri’a andyki iti

% 129. Apip napimbopo'oot i

% 130. Iti isikinim iti

% 131. ``Kahyt tam’a tykat'', irikãraj̃

% 132. ``Tapynsoatyk tykaty'', irikãraj̃ iti

% 133. Masong napimbopo'op gaat tanakymbity

% 134. Masong napimbopo'op saryt pimboop, pimboop, pimboop, iri'aj

% 135. Pimboop pimboop apip nasoewat Oti

% 136. Kary tati kyry kyri

% 137. Tati kyry kotat, iri'aj tatity

% ~

% 138. Masong naõko saryt tati

% 139. Pak pygng ojdn iri´aj tati okyp

% 140. ``Ãh! Sarawak pitat 'a tyka ano y'i'', iri'aj iti

% 141. ``Kinda am'aki pita an nam'a tyka ano y'i'', iri'aj iti

% 142. ``Mo ari'aj myn i'', iri'aj i

% 143. ``Nakakat yjxa kaki daki yjxa'', iri'aj

% 144. ``Asikina my'ana ano'', iri'aj iti

% 145. Apirip napyso saryt tatisok

% 146. ``Ajyk ajyk pak'', iri'aj

% 147. ``Kinda am'aki pita na am'a tyka ano asikina my'ana ano'', iri'aj iti

% 148. ``Ari'aj, i 'a'omaj mo ari'aj'', iri'aj i

% 149. ``Kat yta aka daki yn'', iri'aj

% ~

% 150. ``Atykiri ko’ãj̃ty, yn naka’ot andyki yn mo'', iri’aj

% 151. ``Yn nakaora andyki ewy yn'', iri'aj

% 152. ``Yn naka'ori ewy yn''

% 153. ``Ioto'oma asikirip pihota'', iri'aj iti

% 154. Ipyhowoko'omi ity iti

% 155. Pa'ira tat ia'omaj ity iti

% 156. Iahãraj̃txi padni padni ihot iaka

% 157. Masong kahyt iaka akaj Oti

% 158. Kata'atyka tyym nasoyn horo yjxat

% 159. Kahyt naka'ora kat Oti

% 160. Kat iakabman an

% 161. Kata aka aj̃on iri’aj kahyt

% ~

% 162. Masong naambot ohyn, tamtarakat

% 163. Masong naambot ohyn ewy ohyn

% 164. Masong katat parajak parajak parajak ambodn iri'aj

% 165. ``Masong kahy ewy iti'', iri’a ’omaj̃

% 166. ``An mara ora ewy yti'', iri'aj

% 167. Apip irihynoko iti taso'y tykiri

% 168. Napa'irat iti

% ~

% 169. Masong nakaoto'om saryt

% 170. Sang sang kip kyrapygng iri'aj ewy

% 171. Kahy iri'aj

% 172. Apip naambo saryt ewy kypip tamambo akat

% 173. Masong naambot saryt ohyn i

% 174. Naatej saryt ewy

% ~

% 175. Masong ``kahy yj̃ymbo yti'', naka’a saryt

% 176. ``Kahy yj̃ymbo yti, yj̃ymbo okyp an namboji y’a kinda yti'', naka’a saryt

% 177. Ãh! Kahyt i’aty irikãraj̃ andyki iti

% 178. Masong pyt’oot pyt’oot i’aty irikãraj̃

% 179. Masong naopĩt saryt aj̃ymbo hypypip

% 180. Kiip kiip kiip iri’aj taj̃ymboty

% 181. Masong naatik saryt tasa'ep, 'ot 'ot

% 182. Pak iri'aj kyrapygng

% 183. Iri'aj isa'ep 'ot 'ot

% 184. ``Kinda am´aki pita an nam'a tyka ano y'ii'', iri'aj iti

% 185. Atykiri ``Ãh'', iri'aj

% ~

% 186. Atykiri naambot saryt ohyn Oti

% 187. ``Atykiri ypyrytat andyki yn'', iri'aj

% 188. ``Kahy okyp anamboji akinda yti'', iri'aj

% 189. Atykiri nakatat saryt Oti ohyn

% 190. Katarantyn 'a

% 191. Naambot ewy kypip

% 192. Ewy kypip taambot tykiri

% 193. Ewy ombet ata atej saryt

% 194. Masong natej ewy ombet tyryyt taadn yjasat iri'aj

% 195. Kyyj yjasat iri'aj ewy ombetety

% 196. Katat tysypy'oot

% ~

% 197. Masong nakatat Oti

% 198. Atykiri imbodnoko saryty padni Oti

% 199. Pyrytat sarytyn

% 200. Atykiri naakat ohyn Oti

% 201. Kata´a tyka tyym

% 202. Atykiri naka'a tyka kahorot Oti

% ~

% 203. Atykiri mahorot naka’a gidn j̃onso tyym

% 204. Mata agngi tyym naka’at kahorot tatyym kat j̃onso byki akaty

% 205. Atykiri naka’ooto ’oot j̃onso sara, j̃onso som

% 206. Nakakii’oot kahyt j̃onso

% 207. Kahyt koro’op, ikoro’op araki Botyj̃ koro’op naakat

% 208. Ikiipadna padni j̃onso kahyt kiikit

% 209. Takipi tyym tyym kakiit j̃onso

% 210. Atykiri nakam’at kat Botyj̃

% 211. Atykiri naakat kahyt Oti

% 212. Kahyt j̃onso bykiipat, kahyt yjtakiipat masong nakam’oot kahyt Oti

% \endgroup

% \chapter{Osiip, o ritual de iniciação masculina}

% \letra{A}{ história} abaixo conta do Osiip, um ritual de iniciação masculina que
% não é mais realizado pelos Karitiana. A geração atual de homens mais
% velhos, porém, vivenciou o ritual. O menino, antes de começar a caçar e
% poder se casar, precisava perfurar os ninhos de diferentes tipos de
% vespas e tomar banhos de plantas importantes para o grupo. Nesse
% período, o iniciado precisava também fazer uma reclusão alimentar e
% manter um comportamento reservado, tudo sob supervisão de um homem mais
% velho, geralmente seu pai.

% A narrativa de Osiip foi contada por Cizino Karitiana, o último pajé do
% povo até o presente momento, para sua família estendida e Luciana
% Storto, que a transcreveu e traduziu com Nelson Karitiana, João
% Karitiana, Luiz Karitiana e Inácio Karitiana. Quando, na narrativa,
% Cizino se dirige a um interlocutor, está falando a seu filho mais velho
% que ainda não se casara, para explicar que o ritual teria sido
% necessário no passado quando ele fosse se casar. Para esta publicação, o
% material foi editado por Luciana Storto e Ivan Karitiana.

% Uma primeira descrição do ritual e análise linguística do uso da arte
% verbal nesta narrativa foi publicada em 2019 por Luciana Storto em
% \textit{Línguas Indígenas: tradução, universais e diversidade}, através da
% editora Mercado de Letras. A versão publicada aqui tem um número
% diferente de sentenças, pois o critério usado nesta é prosódico e
% naquela era sintático, mas a divisão da narrativa em partes permanece a
% mesma.

% Uma análise antropológica do ritual, de autoria de Felipe Vander Velden,
% é feita em um manuscrito inédito que comentaremos aqui, pois esclarece
% aspectos do significado do ritual e da relação dos Karitiana com seres
% não-humanos, como as vespas. Vander Velden sugere que os Karitiana usam
% picadas de vespas consideradas caçadoras para através do ritual adquirir
% esta característica das mesmas. Neste manuscrito inédito, intitulado
% ``As vespas que caçam com seus dentes: artefatos multiespécies, ritual e
% relações entre humanos e não-humanos entre os Karitiana (Rondônia)'', o
% autor procura identificar algumas espécies de plantas e vespas citadas
% na narrativa, entre elas a planta denominada \textit{sojoty} , cuja seiva
% leitosa é usada como remédio no ritual, a saber:

% \textit{``Sojoty} (`é como batata do mato', ou `folha que arde',
% dizem; certamente se trata de uma \textit{Araceae}, talvez a que chamamos
% de `comigo-ninguém-pode', possivelmente \textit{Dieffenbachia
% seguine})''.

% \chapter*{Osiip}
% \begingroup\parindent=0em

% \section{Parte I}

% 1. Meu pai me disse, antigamente

% 2. Quando temos dez anos de idade, nos fazem receber o \textit{Osiip}

% 3. Aos dez verões é a salvação do mau caçador do nosso povo

% 4. Nós perfuramos o vespeiro, nos fazem passar a planta \textit{sojoty}

% ~

% 5. Nossos pais nos aconselham quando estamos prestes a nos casar

% 6. Quando estamos prestes a receber uma mulher

% 7. Quando nós, de fato, recebemos uma mulher, nós perfuramos o vespeiro

% 8. No meu caso, eu não perfurei muitos vespeiros

% \section{Parte II}

% 9. Há verdadeiros perfuradores de vespas

% 10. Perfuradores de vespas por dez vezes, perfuradores de vespas por cinco vezes

% 11. Quanto a mim, eu perfurei vespeiro apenas quatro vezes

% 12. Quanto a mim, quatro vezes apenas eu perfurei vespeiro

% ~

% 13. Logo após receber as vespas, não se pode comer

% 14. Três dias depois das vespas, já se pode comer

% 15. É muito difícil, por causa da fome e da sede

% 16. Deita-se, não pode haver nenhuma conversa, não pode haver nenhum
% barulho

% 17. Deve-se caminhar em direção à serenidade

% 18. Quando se está sereno, o espírito do \textit{Osiip} trabalha

% 19. Nós matamos caça

% ~

% 20. Depois de três dias, nós comemos um mingau forte

% 21. Mingau forte, sementes de milho torradas, espigas de milho assadas
% com as folhas, milho assado na espiga é o que nós comemos

% 22. Não se bebe água

% 23. Chicha é para ser bebida

% 24. Não se bebe água mesmo

% 25. De acordo com a palavra dos nossos pais

% \section{Parte III}

% 26. Com a palavra do meu pai dirigida a mim, eu fiz o \textit{Osiip}

% 27. Por causa disso eu mato um pouco de caça assim até hoje

% 28. Macaco apenas

% 29. No meu caso, eu não matei caça grande

% ~

% 30. O \textit{Osiip} é bom

% 31. Não se mata caça de graça

% 32. Não se come caça de graça

% 33. Não se comia caça de graça, antigamente

% 34. Homens sem o \textit{Osiip} não matavam caça, antigamente

% 35. O homem que não soubesse matar não era presenteado com caça,
% antigamente

% 36. Então, por causa disso, costumava-se passar pelo \textit{Osiip}
% antigamente

% ~

% 37. O meu pai falou comigo

% 38. ``Receba o \textit{Osiipo}'', o meu pai disse para mim

% 39. ``Você não vai matar caça para a sua esposa se você não receber o
% \textit{Osiip}''

% 40. O meu pai disse para mim

% ~

% 41. Quanto a mim, eu recebi o \textit{Osiip} por isso

% 42. E fui então tirar as plantas da mistura para o \textit{Osiip}, as
% plantas para as vespas: \textit{sojoty,} \textit{ewoket},
% \textit{gosonderepo}, \textit{Osiip tepy}, e \textit{pasỹ}

% 43. O Osiip é recebido com a mistura

% 44. Sem isso, o espírito do \textit{Osiip} não trabalha

% ~

% 45. A pessoa não deve comer escondido, não deve se masturbar; isso não é
% para ser feito

% 46. Não é para se comer mamão, óleo não é para ser comido, coisas
% gordurosas não são para ser comidas, não se conversa com mulher,
% mulheres não podem se aproximar enquanto nós estamos passando pelo
% \textit{Osiip}

% ~

% 47. Então, meu pai disse para mim: ``nós ainda não terminamos''

% 48. Então, quanto a mim, eu recebi o \textit{Osiip}

% 49. Então, quanto a mim, eu perfurei o vespeiro

% 50. O meu pai cantava, enquanto eu perfurava o vespeiro

% 51. ``Cantar, cantar, cantar'', o meu pai fazia assim enquanto eu perfurava o vespeiro

% ~

% 52. Então, no meu caso, eu perfurei o ninho das vespas \textit{gop miem}

% 53. As \textit{gop miem} são dolorosas e rápidas

% 54. Ao meio-dia, a dor da \textit{gop miem} diminui

% 55. Às duas horas, nós estamos rindo

% 56. ``Eu vou conseguir sarar'', você diz então

% 57. Não era a minha hora de morrer

% ~

% 58. ``Você sarou?'', o nosso pai nos pergunta então

% 59. ``Sim'', nós dizemos ao nosso pai

% 60. ``Muito bom'', o nosso pai diz então

% 61. ``É muito bom, você viu?'', diz então o nosso pai

% 62. ``Você não morreu mesmo, viu?'', o meu pai me disse

% ~

% 63. Então, quanto a mim, cinco dias se passaram

% 64. Depois de cinco dias, a primeira coisa que eu matei para sarar foi
% passarinho

% 65. O pequeno pássaro \textit{morondek} está entre os primeiros que se come
% para sarar

% 66. O pássaro \textit{piisomo} está entre os primeiros que se come, sem
% espalhar as sobras, para sarar

% 67. Depois de matar os primeiros passarinhos que se come para curar, a
% pessoa come outros: \textit{yrypano}, \textit{pityjo}, \textit{hanhano},
% \textit{yt'yto}

% ~

% 68. Naquele momento, caça de grande porte ainda não seria comida

% 69. ``Você ainda não comerá caça de grande porte'', o meu pai me
% disse``Ou você salgará'', disse o meu pai

% 70. ``Você não vai comer caça grande, ou você vai reter líquido''

% 71. Por causa disso, eu não comia aquilo; aquilo não era comido lá

% 72. Então, depois de dez dias, depois de dormir dez dias, a pessoa come
% caça grande

% 73. Come-se porco selvagem, macaco. Quanto a macaco-aranha, não se comeria ainda

% 74. Depois de vinte dias, a pessoa já come macaco-aranha

% 75. Então já não é mais ruim

% ~

% 76. A pessoa ainda não deve se banhar

% 77. Vai se banhar com a planta \textit{Osiip tepy}

% 78. O \textit{Osiip tepy} é nosso instrumento de banho

% 79. O \textit{Osiip mynan}, também, o '\textit{ewoket} também

% ~

% 80. Então, um mês depois, a pessoa come peixe

% 81. O peixe jatuarana grelhado é um dos primeiros que se come

% 82. O peixe-cachorro é um dos primeiros que se come

% 83. Traíra, um peixe caçador, é um dos primeiros que se come

% 84. Então, depois de um mês, come-se jatuarana

% ~

% 85. Então, aos dois meses, caça realmente mansa começa a se aproximar

% 86. Veado, caça, o \textit{Osiip} amansa, amansa

% 87. Caça, o \textit{Osiip} amansa

% 88,Porco selvagem, o \textit{Osiip} amansa

% 89. Caça, o \textit{Osiip} debilita

% 90. Caça, o \textit{Osiip} desnorteia

% 91. Então, no caso do nosso povo, nós gostamos que a caça se aproxime de nós

% ~

% 92. Nós sabemos como matar a caça, nós atiramos bem

% 93. Nós não trememos mais

% 94. Então nós matamos a caça com arco e flecha

% 95. Não se matava caça com as armas do homem branco, antigamente

% 96. Grandes flechas matavam a caça, antigamente

% 97. Com flechas de ganchos, a pessoa fazia a caça morrer gritando, antigamente

% ~

% 98. O meu pai falava, antigamente

% 99. Eu fiquei com o meu pai até amadurecer como atirador

% 100. Com a sorte que eu tinha, o meu pai matava muito

% 101. Dez macacos, o meu pai matava

% 102. Quanto a mim, eu só matei três

% \section{Parte IV}

% 103. Era assim

% 104. Então em três meses, o \textit{Osiip} acaba

% 105. Se não se casa, não acaba

% 106. Se a gente não casa, o \textit{Osiip} será repetido depois de três meses

% 107. Então já não é mais ruim

% 108. Nós não ficamos loucos, nós ficamos espertos

% ~

% 109. Quando acaba, em quatro meses, a pessoa faz o \textit{Osiip} de novo

% 110. A pessoa pega novamente \textit{gop sõwõrã}

% 111. O vespeiro \textit{gop miem} não é perfurado novamente

% 112. Vespeiros de \textit{gop sõwõrã} são perfurados novamente

% ~

% 113. As vespas vermelhas me fizeram desmaiar

% 114. Vespas vermelhas são muito dolorosas

% 115. Só quando cai a noite as picadas de vespas vermelhas aliviam

% 116. Quando dormimos, melhora

% ~

% 117. Depois que melhora, a pessoa perfura novamente

% 118. ``Bate, bate, fura'', a pessoa faz um buraco novamente no vespeiro

% 119. Então a mão entra, a larva dele é novamente removida

% 120. Retira novamente a larva

% 121. O nosso braço inteiro é coberto de vespas novamente

% 122. A dor fica conosco até o romper do dia

% 123. Aquilo fica até o romper do dia

% ~

% 124. Então é assim que éa o \textit{Osiip}

% 125. Quanto a mim, eu peguei as vespas quatro vezes

% 126. As vespas \textit{gop sõrõwã,} as \textit{gop miemo}s as \textit{gop miemo} duas vezes, as \textit{gop sõwõrã} duas vezes também

% 127. As \textit{gop bisõwõrã} eu não peguei

% 128. As \textit{gop bikip} eu não peguei

% 129. Foi assim

% \section{Parte V}

% 130. Então quanto a mim, eu matei caça

% 131. Macacos, eu eliminei muitos macacos

% 132. Quanto a mim, eu matei muitos macacos

% 133. Quanto a mim, eu carreguei muitos cestos de caça

% ~

% 134. Por isso, eu casei

% 135. Quando eu matei caça, o meu falecido pai me liberou para uma mulher

% 136. ``Case-se'', ele disse

% 137. Por isso, eu casei

% 138. ``Casar'' eu fiz

% 139. Eu fiquei com esposa

% 140.  Minha esposa nunca passou fome comigo

% ~

% 141. No meu caso, eu matei caça

% 142. No meu caso, eu alimento a minha esposa assim até hoje

% 143. Com a arma do homem branco, eu alimento a minha esposa assim até hoje

% 144. Desde que eu fiz o \textit{Osiip}, eu sou um bom caçador, e assim permaneço

% ~

% 145. Eu não sou um bom caçador, eu sou um caçador mais ou menos até hoje em dia

% 146. Permaneço até hoje

% 147. Antigamente, eu não era desse jeito

% 148. Eu não precisava sair antigamente, não mesmo

% ~

% 149. Na fossa, então, havia macacos

% 150. Macacos apareciam de repente

% 151. Nós não íamos longe

% 152. Cotia é a primeira caça que aparecia sorrateiramente

% 153. Nós buscávamos as nossas armas e logo, havia macacos lá

% 154. É isso que a gente matava

% 155. Não havia falta de caça

% 156. Nambús, quando fazemos o \textit{Osiip}, aparecem muitos nambus

% ~

% 157. Quando você fizer o \textit{Osiip}, você também vai ser assim

% 158. Você não vai mais ser como você é agora

% 159. Você não vai mais ser caçador ruim

% 160. Você não vai mais ser um mata-nada

% 161. Vocês vão sentir o peso da sua caça

% ~

% 162. Plantas parecidas com cipós precisam ser passadas em nós

% 163. O timbó precisa ser aplicado no nosso rosto

% 164. Aplicado aqui no rosto

% 165. Aplicado aqui e aqui

% 166. A erva \textit{pasỹ} tem que ser aplicada aqui e aqui

% ~

% 167. O \textit{Osiip} faz a gente ficar com tornozeleiras e um cinto, que
% cheiram bem

% 168. Então a caça definitivamente não desaparece

% 169. O cipó do \textit{Osiip} é feito para ser colocado nos nossos pênis

% 170. A nossa urina o joga fora

% 171. A nossa urina faz ``tchoo'', pra cima

% 172. Nós não tocamos no nosso pênis

% 173. Então nós matamos a caça

% ~

% 174. Então os nossos velhos dizem

% 175. ``Agora atire para cima''

% 176. Em quatro meses, os nossos velhos nos dizem ``tente aquela em
% seguida''`

% 177. Então nós atiramos, bem envergonhados

% 178. Não achamos que vamos acertar, mas acertamos

% ~

% 179. A flecha vai sozinha nela

% 180. ``Voa, acerta'' no meio dela

% 181. A flecha cai, bonita, junto da caça

% 182. ''ai, perfura''

% 183. Nós achamos que acertamos sem querer

% ~

% 184. Então nós fazemos novamente

% 185. Nós não achamos que vamos acertá-la, mas, mesmo assim, de qualquer
% forma, nós acertamos novamente

% 186. ``Voa, acerta'' novamente

% 187. Nossa flecha nunca está vazia

% ~

% 188. ``Ah, eu estou assim'', nós dizemos

% 189. Nós ficamos muito bons

% 190. A nossa flecha fica como a arma 22 do homem branco

% 191. Nossa flecha genuína

% 192. Cai apenas na caça

% 193. Mesmo que pensemos que não vamos acertar, acertamos

% 194. A flecha encontra sozinha o seu caminho até a caça

% 195. Eu fiz o pai do Rogério ficar meio ansioso por causa de um jacamim
% voando

% 196. Era assim realmente

% ~

% 197. O meu pai cantava assim

% 198. Há muitas músicas do \textit{Osiip}

% \begin{quote}
% \forceindent\textit{199. Cesto de caça até o topo, até o topo, até o topo}

% \textit{200. E eu o estou carregando, estou }

% \textit{201. Cesto de macacos até o topo, até o topo}

% \textit{202. Cesto de macacos-aranha está cheio até o topo}

% \textit{203. E eu o estou carregando, estou}
% \end{quote}

% 204. Esta é a música do \textit{Osiip}

% \begin{quote}
% \forceindent\textit{205. Bate, bate, bate, bate, a madeira que se move está chorando}
% \end{quote}

% 206. Agora é isso, acabou

% ~

% 207. Quando nós estamos começando a entrar no ninho das vespas

% 208. O pássaro \textit{teõwãt,} olha para nós. O pássaro '\textit{eet'eet},
% olha para nós. O pássaro \textit{hĩrã}, olha para nós

% 209. ``Vá'', eles nos dizem

% 210. Então nós não ficamos mais parados

% ~

% 211. Os nossos pais pegam a gente

% 212. No peito deles os nossos pais nos levam

% 213. Empurrar a gente quando está perto das vespas, os nossos pais fazem

% 214. Então os nossos pais correm um pouco

% ~

% 215. Cair em um abraço, nós fazemos, no topo do ninho das vespas

% 216. O nosso medo não existe mais

% 217. ``Susto'', ele nos pega com isso

% 218. Nós então nos sentamos abaixo dele, nessa altura, na árvore

% 219. Nós nos sentamos de frente para ele

% ~

% 220. Então as vespas fazem ``ffffff''

% 221. A mão vai para dentro e para fora do ninho

% 222. Nós limpamos o agrupamento delas

% 223. Depois disso, várias vespas picam o nosso braço esquerdo

% 224. Então nós pegamos a sua larva

% 225. E a colocamos aqui no nosso peito

% 226. Nós não respiramos mais

% ~

% 227. Eles dizem que isso é para que o nosso cesto de caça fique tão cheio
% que mal possamos carregá-lo

% 228. Eles dizem que nós ficamos curvados

% 229. Então nós caímos muito perto, exatamente lá

% ~

% 230. Então nós aplicamos a planta \textit{sojoty}

% 231. Tiram-se os ferrões

% 232. Nós aplicamos a planta \textit{sojoty} nos ferimentos deixados pelos
% ferrões

% 233. Nós fazemos uma massagem usando o \textit{sojoty}

% 234. Também há um tipo diferente de \textit{sojoty} cru

% 235. ``Você vai aplicar? \textit{Sojoty} cru?'' os nossos pais nos dizem

% 236. ``Aplique-o'', nós dizemos

% 237. Nós achamos que a dor vai embora

% 238. O nosso pai aplica isso em nós

% 239. Passa, passa, passa no nosso ânus, no nosso pênis

% 240. Não há mais

% 241. Então a dificuldade de caçar vai embora

% 242. Então não há mais dificuldade de caçar

% 243. Com essa aplicação, nós sentimos a dor do \textit{sojoty}

% 244. O \textit{sojoty} dói de um jeito diferente

% ~

% 245. Foi assim que eu fiz

% 246. Então o meu pai riu de mim

% 247. Então para a minha surpresa, de tarde eu já estava curado

% 248. O meu pai ficou alegre

% 249. ``Agora, você vai matar caça, meu filho''

% 250. ``Você vai matar para a sua esposa''

% 251. ``Você vai matar para os seus familiares''

% 252. O meu pai disse pra mim

% ~

% 253. Então eu já não era ruim de caça

% 254. Aí eu me tornei um verdadeiro caçador

% 255. Um verdadeiro matador de caça

% 256. Um verdadeiro matador de mutum

% 257. Por causa disso, você está bem alimentado hoje

% \section{Parte VI}

% 258. Agora eu só mato com a arma do homem branco

% 259. Eu não costumava matar com as armas do homem branco antigamente

% 260. Quanto a mim, é com flechas que eu costumava matar antigamente

% 261. Fazer tiro ao alvo

% ~

% 262. Depois disso, eu perfurei de novo

% 263. O \textit{gop miem}, eu perfurei novamente

% 264. O \textit{gop miem} que tinha deixado o ninho

% 265. Esses não me picaram

% 266. Esses não picam muito

% 267. Então eles só picaram o meu peito

% ~

% 268. Depois disso, eu não tinha mais o meu pai

% 269. O meu pai morreu

% 270. Quando eu estava quase concluindo o \textit{Osiip}, o meu pai morreu

% 271. Depois que meu pai faleceu, eu não perfurei mais vespas

% ~

% 272. Foi aqui que eu perfurei novamente um vespeiro

% 273. O irmão do meu pai me fez colocar a mão no vespeiro \textit{gop
% sõwõra}, o irmão mais novo dele, meu cunhado, meu sogro, me fizeram
% colocar a mão em um vespeiro

% 274. Desta vez, a vespa \textit{gop sõwõra} me picou

% 275. A vespa picou até as minhas orelhas

% ~

% 276. No começo, não se come no \textit{Osiip}

% 277. Nós dormimos três dias

% 278. Fica-se muito fraco

% 279. Quando dez dias passavam, os homens daqui costumavam matar a
% primeira coisa para comer

% 280. Em cinco dias, o meu pai me deixou comer passarinho

% ~

% 281. Então nós vivemos como bons caçadores até os dias de hoje

% 282. Waldemar, Garcia, nós estamos matando um pouco até os dias de hoje

% 283. Antigamente, nós nos guiávamos pela vivência dos nossos anciões



% \chapter*{Osiip}

% 1. Pyry'a ta'ãn y'it keerep

% 2. Dez anos yjakat yjxa nakam'yt Osiip

% 3. Dez ngogorongãt yjxa osiit

% 4. Yjxa naka'obm gopo, yjxa nakam'yt sojoty

% ~

% 5. Nakahadn yjxat yj'it yjsooj pasagngam tyki'oot

% 6. J̃onsot yjamy pasagng tyki’oot

% 7. J̃onsot yjamy tykiri, yjxa naka’obm gop

% 8. Yn i’obm pitani yn gop

% \section{Parte II}

% 9. Pyrykiidn taso gopo'obmon tyym

% 10. Dez gopo 'obmono, cinco gopo 'obmono

% 11. Quatro myry'in yn naka'obm yn gop

% 12. Otadnamyn yjpyoot myry'in naka'obm yn gop

% ~

% 13. Atykiri ipynpyt'y adyky gop by'y

% 14. Myj̃ymp yjkat napynpyt’yt gop

% 15. Asara'idna pitat opipytyty seakaty

% 16. Neng nengãt, ipynhadni ipynorooroni

% 17. Pongyp napyntarakat

% 18. Pongyp yjakat tykiri, napysemem Osiip

% 19. Yjxa naokyt him

% ~

% 20. Tres dias yjakat yjxa naka'yt sojsara

% 21. Sojsara, j̃om hopo, j̃om pyka, j̃om porojo yjxa ti’yt

% 22. Ia'y ese

% 23. Kytop ia'yt

% 24. Ia'y padni ese

% 25. Yj'iti hadna tyym

% \section{Parte III}

% 26. Ynty y'iti hadna tyym yn nakam'at ta'ãt Osiip

% 27. Atykiri yn naoky pymbyrat kinda ka y'a tykat tyym

% 28. Pikom myry'in

% 29. Yn ipopi padni him ondyt yn

% ~

% 30. Osiip naakat ise'at

% 31. Kydnym iaoky him

% 32. Kydnym ia'y him

% 33. Kydnym ia'y keerep him

% 34. Taso osiiki ioky padni keerep him

% 35. Iahiti padni him okyykit keerep himty

% 36. Atykiri nam'yt keerep Osiip

% ~

% 37. Nakahadn ta'ãt ynty y'it

% 38. ``Im'y Osiipo'' naka'at ta'ãt ynty y'it

% ~

% 39. ``An ihot oky asooj Osiip an nam'yki tykiri''

% 40. Naka'at ta'ãt ynty y'it

% 41. Atykiri yn nakam'yt ta'ãt Osiip yn

% 42. Yn nakako andyk Osiip pyj͂ongo, gop pyj͂ongo, sojoty, ewoketo, gosonderepo, Osiip tepy, pasỹ

% 43. Ipyj͂ong pyj͂ongãt, nam'yt Osiip

% 44. Aki tykiri ipysemem padni Osiip

% ~

% 45. Ipynpyt'y saramynt, ipynpyso'y, iam'a padni 'a

% 46. Ia’y padni byyty, ia’y padni oleo, ia’y padni kinda oroja, ia’y padni asyryty, iamhadni j̃onso, ipyrõtamy j̃onso yjosiit tyki’oot

% ~

% 47. Atykiri ``yjotamki tykiri'' kahyt naka'at ta'ãt y'it ynty

% 48. Atykiri yn nakam'yt Osiip yn

% 49. Atykiri yn naka'obm ta'ãt yn gop

% 50. Nakahyryj̃ ta’ãt y’it gopo’obm tyso’oot

% 51. Hyryj̃ hyryj̃ hyryj̃ naka’at ta’ãt y’it gopo’obm tyso’oota

% ~

% 52. Atykiri yn naka'obm yn gop miem yn

% 53. Oti okyn taakat gop miem

% 54. Omenda tapihogon gop miem oti

% 55. Gokaradn yjtaandyjyt

% 56. Atykiri ypynoydn my'an an daki

% 57. Ypopopa akam'ani gop yjxa ka'a nayt

% ~

% 58. Atykiri ``aoyt hỹ?'' naka’at yjxat yj’it

% 59. ``Yoyt ỹrỹ yj’it'', yjtaka’at yjxa

% 60. ``Se'a pitat'', masong ka'at yj'it

% 61. ``Se’a pitat aso’oora hỹ?'' masong naka’at yj’it

% 62. ``Popoki pitat an 'a'', naka'at ta'ãt ynty y'it

% ~

% 63. Atykiri ytakakat yn yjpyt yn

% 64. Yjpyt ykat ytaohyj͂ym okyt yn iij͂

% 65. Iij͂ 'ina morondek taakat ohyj͂ym

% 66. Piisomo pyrohyj͂ym oro'oroni

% 67. Ohyj͂ym pita yjnaoky byyg na’yt iij̃, yrypano, pityjo, hanhano, yt’yto

% ~

% 68. Ia'y andyky him ondyty tyym

% 69. ``An i’y andyky him ondyt'' naka’at ta’ãt ynty y’it, ``an takasiibmaj̃'', naka’at ta’ãt ynty y’it

% 70. ``An i’y padni him ondyt, an tapese tygngaj̃''

% 71. Atykiri yn naka'ykiit ta'ãt; a yn ia'y a

% 72. Masong dez dias yjakat yjpyotatytap yjkat tykiri, naa'yt him ondyt

% 73. Naa'yt sojxa, pikomo, 'õrom ia'y andyky 'õrom

% 74. Yj myhin pipyyk yjkat naa'yt 'õrom

% 75. Atykiri isara'it oko

% ~


% 76. Ipynoty andyky

% 77. Osiip tepyt napynoty andyk

% 78. Osiip tepyt nam'at yjotyypat

% 79. Osiip mynam tyym otyypat, 'ewokete tyym

% ~

% 80. Atykiri myhin otidna na'yt 'ip

% 81. Pojpoko byhibmina ia'y'oot

% 82. `Am naakat ia'y'oot

% 83. Biira, 'ip pykop naakat ia'y'oot

% 84. Atykiri myhin otidnat na'yt pojpok

% ~

% 85. Ambygng sypom otidnat yjakat takaheredna'oot him sikirip pitat

% 86. De, him ataompong ompong Osiip

% 87. Him ataompong Osiip

% 88. Sojxa ataompong Osiip

% 89. Him amsikini padni Osiip

% 90. Him atampa'irat Osiip

% 91. Atykiri yjtaso’oot hãraj̃ yjxa him herednat yjxa

% ~

% 92. Yjtapypyydn him okyyp yjtakapon hãraj̃ yjxa

% 93. Yj̃orosowot oko

% 94. Atykiri yjxa naokyt bypan pitapip tyym him

% 95. Opok bypan pip iaoky padni keerep him

% 96. Bokore naokyt keerep him

% 97. Napisỹ pip imkikĩ popit nam’at keerep him

% ~

% 98. Naka'at ta'ãt y'it keerep

% 99. Y'itityyt ytaaka andyk ta'ãt ypyokõrõngyt tyki'oot yn

% 100. Ysondak atapopit ta'ãt y'it

% 101. Yjpy otatytap napopit pikom y'it

% 102. Myj̃ymp myrỹ’ĩt yn napopit ta’ãt yn

% \section{Parte IV}

% 103. Naka'at ta'ãt kahyt

% 104. Atykiri myj̃ymp otidna namondet Osiip

% 105. Yjsooj pasagngam aki tykiri, iamondete padni

% 106. Yjsooj pasagngam aki tykiri, namondet Osiip myj̃ymp otidnat

% 107. Atykiri isara'it oko

% 108. Yj̃oropopobmi, yjtaakat se’at

% ~

% 109. Yjxa imondet byyk otadnamyn otidnan nam'y okoot Osiip

% 110. Nam'y okoot gop sõwõrã

% 111. Ia'obm oko gop miem

% 112. Gop sõwõrã ia'obm okoot

% ~

% 113. Gop sowõrã ytaoky 'it ta'ãt yn

% 114. Oti pitat gop sõwõrã

% 115. Go moj̃ tapihogngon gop sõwõrã

% 116. Yjkat ta'oot napihogngon

% ~

% 117. Masong ipihogngon byyk na'obm okoot

% 118. Tyng tyng pỹk nam’a okoot gopo ’op

% 119. Pytat masong nam'a okoot i'it ikyn

% 120. Naapykyj okoot i'it

% 121. Yjtaotõ okoot yjj̃ongo akatyym

% 122. A naakat ipyt haap yjxat

% 123. A naakat ipyt haap

% ~

% 124. Atykiri naka'at Osiip kahyt

% 125. Otadnamyn yn nakam'yt yn gop yn

% 126. Gop sõwõrã, gop miemo, gop miemo sypom, gop sowõrã sypom tyym

% 127. Gop bisõwõrã yn tim'ykit

% 128. Gop bikiip yn tim'ykit

% 129. Naka'at kahyt

% \section{Parte V}

% 130. Atykiri yn naoky ta'ãt him yn

% 131. Pikomo, ytapyriit pikomty yn

% 132. Yn napopit pikomo yno

% 133. Yn natayt serepam yno

% ~

% 134. Atykiri ytakasooj ta'ãt yn

% 135. Him yn naoky tykiri, napyhit ta’ãt ynty ymbykit j̃onsoty

% 136. ``Asodj̃a ano'', naka’at ta’ãt

% 137. Atykiri ytakasooj ta'ãt

% 138. ``Soojdn'' ytaka'at ta'ãt yn

% 139. Ytaaka ta’ãt j̃onso tyyt yn

% 140. Iopipydni yhot ysooj

% ~

% 141. Yn napopit himo yno

% 142. Yn napyt'yt yno ysooj ka y'a tykat tyym

% 143. Opok bypan pip yn napyt'yt horo ysoj ka y'a tykat

% 144. Yosii tykiri, sondagng pitat tyym ka y'a tykat

% ~

% 145. Ysondagõko tyka soro aty, mynda sondagngan ytaka'a tykat

% 146. Ka y'a tykat

% 147. Ahorot yaka padni keerep

% 148. Ymboryty padni padni keerep

% ~

% 149. Gyj̃ipa pip tyym naakat pikom

% 150. Taasootap'oot ojdn'oot ojdn, naka'at pikom

% 151. Opap yjtata padni

% 152. Myndo horot naaka'ooto'oom him

% 153. Teteet kej yjpanty yjxa a'oot naakat pikom

% 154. A naakat yjxa tiokyt

% 155. Piharap iki padni him

% 156. Pomo, ipynboryty padni pom, yjosiit tykiri

% ~

% 157. A horot tyym ajxa ka'aj ajxa ajosiit tykiri

% 158. Mahorot aj̃oroki oko

% 159. Aj̃oro sondakap oko

% 160. Pont horop aj̃oroki oko

% 161. Ajpyrym'ewep daki him

% ~

% 162. Tepa horot iki naakat iam'yt yjsok

% 163. Ting iam'yt yjasoop

% 164. Hyp a’a pasỹ

% 165. Pasỹ nam’yt ma, nam’yt ma

% 166. Tiing tiing nam’at mapip pasỹ

% ~

% 167. Kysembyk sokotyt, atyyt napỹrãkat yjj̃ing hãraj̃xat napỹrãkat Osiip

% 168. Atykiri imbodn oko padni padni him sikina

% 169. Yjopop bik nam'at Osiip tepy

% 170. Yjsi kapydn 'a

% 171. Xooo kin, naka'at yjsi

% 172. Pynpyso'y yjopo

% 173. Atykiri yjxa naokyt him

% ~

% 174. Atykiri pyryhadnan yjxaty yjhyko

% 175. Apõrã kabman ohyn goop

% 176. Otadnamyn otidnat ``a pasagngã kabma j̃aty'' naka’at yjxaty yjsota

% 177. Atykiri yjxa kapon, yjxa yjombyky pibm pibm

% 178. Isok yj’aty yjkãrã sogng isok yj̃oro’i

% ~

% 179. Ta'asotap katat bypan ikyyn

% 180. Pyrrrr sok isendap

% 181. Naka'ot se'a pitat yjpan him tyyt

% 182. Kyra pyygng

% 183. Isok a’ojdn yj’at yjparakãrãrĩ

% ~

% 184. Masong yjxa ka'a okowak

% 185. Isok yj’at yjkãrasogng isok yj̃oro atĩ

% 186. Pyrrr sok andagngan

% 187. Ipi'ywypy padni yjpon

% ~

% 188. Kat ytaka'a tyka akabm an, yjxa ka'at atykiri

% 189. Se'a pitat yjta'akat

% 190. Naki horo opok bypan vinte e dois apykop akat naakat yjpan

% 191. Bypan pita

% 192. Himsok myry'in ka'ot

% 193.Isok yj’at yjkãrãsogng isok yj̃oro’i

% 194. Taasootop nasonderep yjpan himsok

% 195. Rogério ’it naakat yn timkyrypywak, yhot syyj̃ tengãsok

% 196. Naka'at ta'ãt pita

% ~

% 197. Kahỹryj̃ ta’ãt ymbykito

% 198. Pyrykiidn Osiipi hỹryj̃a

% \begin{quote}
% \forceindent\textit{199. Serepam okywyndok okywyndok okywyndok}

% \textit{200. Yn na'a tyka ki myno myno}

% \textit{201. Pikomo serepamo okywyndok okywyndok}

% \textit{202. `Oromo serepam okywyndok}

% \textit{203. Yn na'a tyka ki myno}
% \end{quote}

% 204. Naka’at Osiipi hỹryj̃a

% \begin{quote}
% \forceindent\textit{205. Kyn kyn kyn kyn ’epe kaj̃a ihyryp tyj̃at}
% \end{quote}

% 206. Naka'at kabm, naka'at ipyykyp

% ~

% 207. Gopokyn yjmem pasagng tyso'oot

% 208. Ytakyn asomorã teõwãt. Ytakyn asomorã ’eet’eet. Ytakyn asomorã ytakyn hĩrã

% 209. ``A'aso'oora'' yjtam'at yjxa

% 210. Atykiri yj̃oroso oko

% ~

% 211. Tik naka'at yjsok yj'it

% 212. Takyryp yjta'atot yj'it

% 213. ``Bypiit gop akat bik'', naka'at yjxat yj'it

% 214. Atykiri nakapykyn pymbyra yj'it

% ~

% 215. Pak pygng, yjxa ka'at iokyp yjxa

% 216. Yjpi tysot yj̃oroso oko

% 217. Tej, i yjtakaot takyn

% 218. Dognga'ot, masong kat naohyn akat 'ep

% 219. Dognga'ot yjxa ka'at yjxa iyrypap yjxa

% ~

% 220. Masong ka'at gop ``ffffff''

% 221. Pytat pyp masong naym'at

% 222. Byj naam'at itop

% 223. Atykiri atyp iriso oko gop yj̃ongo poj̃sok

% 224. Masong naampyp i'it

% 225. Mak yjkyryp

% 226. Yjhan oko padni yjxa

% ~

% 227. Serepamaty yj'ewep tyka'ooma

% 228. Naaka saryt omem omem yjtat

% 229. Atykiri yjtaka'oot, ony horot, bypit tapitat

% ~

% 230. Atykiri yjxa nakam'yt sojoty

% 231. Naapykyj isypojo

% 232. Nam'y sojoty isypoj okyp

% 233. Piik piik piik nam'at sojoty tyyt

% 234. Tamyryta tyym naakat sojoty hipiki

% 235. ``An im’yj̃ hỹ? sojoty hipiki hỹ?'' Naka’at yjxat yj’it

% 236. ``Im'y'' yjtaka'at yjxa

% 237. Ipihogonat yjkãrãt

% 238. Nakam'yt yj'it a yjsok

% 239. J̃ong j̃ong yjjere’oworip, yjoporip

% 240. Imbodn oko

% 241. Atykiri nakahot sondakap

% 242. Atykiri imbodn oko sondakap

% 243. Aatyytap yjtapytyp andyk yjxa sojotyty

% 244. Mynhodnom tyym naotidn sojoty

% ~

% 245. Ytaka'at ta'ãt yn

% 246. Atykiri naandyj ta'ãt yhot y'it

% 247. Atykiri ytaoyt my’an gomoj̃ gop sowõrãty

% 248. Masong naosedna y'it

% 249. ``Kabm anaokyj him, y'it''

% 250. ``An nakahot okyj asooj''

% 251. ``An nakahot okyj aky'o''

% 252. Naka'at ta'ãt ynty y'it

% ~

% 253. Atykiri ambyk ysondakap oko ti’ĩ yn

% 254. Atykiri ytaakat ta'ãt sondaga pitat

% 255. Him oky pitat

% 256. Bisỹ oky pitat

% 257. Atykiri ajna'agi horo mahorot pyt'yt ajxa

% ~

% 258. Pongpan pip myry ĩt yn naokyt ka y’a tykat

% 259. Pongpan pip yn ioky him keerep

% 260. Bypan pita pip yn naokyt yn him yn keerep

% 261. Paj syk isok

% ~

% 262. Ambyk tyym yn naka'obmo okoot ta'ãt

% 263. Gop miem yn naka'obm okoot ta'ãt

% 264. Gop miem asop tysyp

% 265. Yotong tĩ’ĩ padni ’a

% 266. Yotong pitadni 'a

% 267. Atykiri ykyryp myrỹ’ĩt ytaotõ ta’ãt

% \section{Parte VI}

% 268. Ambyyk imbodnoko tĩ’ĩ y’it yhot

% 269. Nakamboop ta'ãt y'it yhot

% 270. Osiipit ypyndangĩwak nakamboop ta’ãt y’it

% 271. Y’it bowyt tykiri, yn i’obmo oko tĩ’ĩ gop

% 272. Hak yn naka'obm oko ta'ãt myhin gop

% 273. Ysyp'et ytampytat ta'ãt, iket, ysymbo, ysojo syp ytampytat ta'ãt gop sõwõrã pip

% 274. Atyym ytaotõ ta'ãt gop sõwõrã

% 275. Yopirisawarip ytaotõ ta'ãt yn gop

% ~

% 276. Ipynpyt'y adyky padni Osiip

% 277. Myj̃ymp yjkat

% 278. Pysowot pysowot harara

% 279. Yjpy otatytap taakat naohyj̃ym okyt hak taso ki

% 280. Yhot y’it naakat yjpyt kat, iij̃ ’yt

% ~

% 281. Atykiri ytaka’agĩt yta sondagan ma yta’agĩt

% 282. Pitana, Tang’ỹ kinda oky pymbyrat yjtaka’agĩt

% 283. Keerep taso kikyri iki naakat yjtat

% \endgroup

% \chapter{O encontro de dois grupos locais}

% \letra{A}{ história} abaixo, contada por Barabadá, trata de um evento crucial para
% os Karitiana, que ele vivenciou quando era jovem: a reunião de dois
% grupos locais, que até então eram autônomos: os Joari (ou Capivari),
% grupo de Barabadá, e os Karitiana, grupo de Moraes Karitiana. Moraes
% tinha se casado com sete mulheres para assegurar a continuidade de seu
% grupo e os Joari tinham apenas uma mulher naquele momento. A reunião
% dessas pessoas na atual aldeia central da Área Indígena Karitiana e a
% realização de casamentos entre elas foi fundamental para que se
% revertesse o forte declínio populacional vivenciado por ambos os grupos,
% que poderia resultar em sua extinção.

% A narrativa foi contada por Barabadá em sua casa, sua esposa Augusta e
% seu filho Valdomiro, gravado por Luciana Storto, que a transcreveu e
% traduziu com a ajuda de Valdomiro Karitiana e Nelson Karitiana. Para
% esta publicação, o material foi editado por Luciana Storto e Ivan Rocha.

% \chapter*{O encontro de dois grupos locais}

% \begingroup\parindent=0em

% 1. Quando minha mãe morreu, eu fui andar sozinho

% 2. Eu fui andar lá

% 3. Depois eu voltei de novo aqui

% 4. Eu fiquei aqui por um tempo

% ~

% 5. Depois eu fui de novo lá na aldeia

% 6. Então eu vi o acampamento dos homens

% 7. Onde dormiram, onde fizeram fogo

% 8. Eu vi o local da fogueira deles de perto

% 9. O acampamento abandonado deles

% 10. ``Tinha gente aqui, né?'', pensei ``Tinha gente\footnote{Pessoas da
%   mesma etnia.  Os Karitiana e Joari pertenciam a um mesmo grupo étnico
%   que havia se separado.} aqui?''

% ~

% 11. Meu finado pai secava taquara

% 12. Então eles chegaram ali

% 13. ``Dormiram aqui?'', eu disse

% 14. Então meu finado pai e os outros chegaram

% 15. Chegaram

% 16. ``De quem é este acampamento? Este acampamento é do inimigo?''\footnote{Inimigo
%   nesta passagem é entendido como não indígena, ou não Karitiana.}

% 17. ``Aqui tinha gente'', meu finado pai falou

% 18. ``Vamos ver o lugar onde o inimigo dormiu'', eu disse

% 19. ``Vamos lá''

% ~

% 20. O caminho tinha acabado

% 21. Não tinha picada {[}aberta na mata{]}

% 22. Então nós retornamos

% 23. Então eu fui pelo caminho, eu continuei sozinho

% 24. Fui, então chegou meu finado pai atrás de mim

% 25. ``Vamos ver, você não vai sozinho. Você não deve continuar sozinho'',
% meu finado pai falou

% ~

% 26. Eu fui

% 27. Andamos, andamos, andamos, chegamos

% 28. Então eu fiquei pensando, com saudade de quando eu estava na beira do
% rio

% 29. ``Eu não vou cruzar o rio por aqui'', eu falei para mim mesmo

% 30. Então eu vi a outra margem

% 31. Cruzei pela água

% ~

% 32. Eu olhei, eu continuei de novo, nós continuamos, chegamos na aldeia

% 33. Havia pegadas de homens

% 34. ``Homens estiveram aqui!'', disse meu finado pai

% 35. Então fomos por ali

% 36. Eu fui na frente

% 37. Andamos, andamos, andamos, chegamos na casa do inimigo

% 38. Quando chegamos na casa deles, o galo cantou

% 39. ``Aqui tem gente!''

% ~

% 40. Andamos, andamos, andamos, chegamos

% 41. Nós ficamos em pé perto deles

% 42. Meu finado pai ficou com medo

% 43. ``É o inimigo'', disse meu finado pai

% ~

% 44. Nós voltamos e fomos de novo

% 45. Então, dizem que os homens chegaram de novo

% 46. Homens muito valentes

% 47. ``Ah, vamos matar os inimigos!''

% 48. ``Vamos ver! Assim será''

% 49. ``Já que você é valente, você vai entrar''

% 50. Estávamos doentes, com a doença do branco\footnote{A expressão
%   ``doença do branco'' pode se referir a diferentes doenças levadas
%   paras as aldeias por agentes públicos, missionários, garimpeiros e
%   madeireiros, por exemplo, malária, sarampo, gripe, tuberculose etc.}

% 51. Então nós não entramos

% 52. Eles vieram atrás de nós

% ~

% 53. Então, quando eles vieram, apareceram outros homens

% 54. Eles esperavam caça no pé de tucumã\footnote{\textit{Astrocaryum
%   aculeatum}.}

% 55. Enquanto eles esperavam no pé de tucumã,

% 56. Foram buscar o finado pai de Eremby Byyt\footnote{O finado pai de
%   Augusta, esposa do narrador Barabadá.}

% 57. ``Eu matei mutum'', ele disse

% 58. ``Então, eu vim'', eu disse

% 59. Eu fui

% 60. ``Eu vou esperar no local onde o mutum come, amigo''

% 61. ``Ah'', disse meu finado pai

% ~

% 62. Eu fui, andei, andei, andei, sentei

% 63. Diz que eles imitavam o barulho do nambu

% 64. ``\textit{Hong}'', ele fez. ``\textit{Hong}, aqui tem alguém'', disse o homem

% 65. ``\textit{Hong}, \textit{hong}, \textit{hong}'', fizeram

% 66. Dizem que chegaram na tocaia

% 67. ``Olha aqui, olha aqui, aqui tem alguém'', disseram

% 68. ``Você matou cotia, amigo?'',\footnote{Amigo ou querido(a), com a função
%   de um vocativo.} ele falou. ``Amigo?'' (silêncio)

% 69. ``Amigo? Aqui tem alguém, porque tem um buraco na tocaia''

% 70. Dizem que falou assim o finado pai de Eremby Byyt (Augusta)

% 71. Olhou, abriu, fez assim por cima da tocaia

% 72. Olhou, viu a arma dele com um desenho em vermelho e branco

% 73. Ele ficou com medo

% ~

% 74. Olhou, andou, andou, parou

% 75. ``É você, amigo?'', falou para o amigo dele. ``É você, amigo?'', chegou

% 76. ``Aqui tem.  Vai lá com ele''

% 77. Dizem que falou assim o finado pai de Eremby Byyt

% 78. Dizem que foi lá com ele

% 79. Olhou, abriu

% 80. ``Olha aqui!''

% 81. ``Ah! Você me deu medo, amigo!''

% 82. ``Sai, amigo'', disse ele

% 83. ``Você não vai me matar né, amigo?'', disse o amigo dele

% 84. ``Não vou matar não, amigo. Sou gente (da mesma etnia que você), amigo''

% 85. ``Você me deu medo, amigo. Você me causou muito medo. Quando você me
% estranhou, eu fiquei assim''

% 86. ``Eu não estranho você não. Vem para cá''

% 87. ``Você não vai me matar não, amigo?''

% 88. ``Vou nada! Vou nada! Vem para cá''

% ~

% 89. Diz que ele saiu

% 90. Andou, andou, saiu

% 91. ``É ele mesmo'', disse o homem ``Você está sozinho?''

% 92. ``Estou com companhia. Meu amigo está longe'', ele disse

% 93. Diz que ele ficou em pé e quase caiu o adorno do pênis dele

% 94. ``O adorno do seu pênis está caindo, meu amigo'', diz que falou o
% sobrinho dele

% 95. ``Não cai, não, amigo. Não vai cair não, amigo! Quando você me
% estranhou, eu fiquei assim, amigo''

% 96. ``Ah, não vai ser fácil, meu amigo''

% 97. ``Você me deu medo, amigo''

% 98. ``Qual é o seu nome?''

% 99. ``Esse é meu nome'', disse ele

% 100. ``Este é o nome do meu tio'', dizem que falou o finado pai
% dela.\footnote{O narrador se refere aqui à sua esposa, Augusta Karitiana.}
% ``Eu não sabia que você era meu tio''

% ~

% 101. ``E o seu amigo?''

% 102. ``Está lá''

% 103. ``Não temos mais mulher. Só há uma mulher'' (na aldeia do grupo
% Joari/Capivari)

% 104. ``Vamos lá, vamos tomar chicha e comer carne na casa dele''

% 105. ``Tem carne na mão dele''

% 106. ``Tem homem caçador aqui''

% 107. ``Quem matou macaco-aranha?''

% 108. Aquele que matou disse para mim

% 109. ``Ah, vamos ver a caça''

% 110. Fomos indo em zigue-zague, zigue-zague, zigue-zague, chegamos

% 111. Enquanto eu comecei a fazer um jirau

% 112. ``Ah, vamos, meu tio. Vamos tomar chicha, meu tio''

% ~

% 113. Fomos, andamos, andamos, chegamos

% 114. Eu estava lá

% 115. ``Ah, nós chegamos'', disseram os homens. ``Ah, nós chegamos.
% Seguindo o seu cheiro, nós chegamos aqui''

% 116. ``Ah, vocês são como nós também''

% 117. ``Nós somos vocês. Nós fomos feitos da mesma parte do cabelo do
% Byjyty\footnote{Byjyty é o neto do demiurgo Botyj̃;  de mechas de seu
%   cabelo, foram criados os seres humanos de todas as etnias indígenas.}

% 118. ``Deita aqui, deita aqui comigo''

% 119. Deitaram, deitaram comigo

% 120. ``Ah'', eu falei para ele, ``Cadê a sua família?''

% 121. ``Eu não tenho família. Meu pai foi lá para o Igarapé Preto'', disse
% meu irmão sobre meu finado pai

% ~

% 122. Conversei, conversei, falei

% 123. O meu finado pai veio para cá

% 124. Andaram, andaram, chegaram.

% 125. ``Devagar os homens vieram'', falou meu finado pai

% 126. ``Chegaram? Quem? É o Taaty?''

% 127. ``Não é não, é o pai dele''

% 128. ``Está bem'', e foram

% 129. ``Sim'', o amigo dele disse, entrou

% 130. ``Você está aqui, meu irmão?'', ele disse

% 131. ``Estou aqui, meu irmão. Você está deitado?''

% 132. ``Estou aqui, meu irmão'', disse meu finado pai

% 133. ``Eu estou atrás de você. Você ficou com medo de mim, meu irmão?''

% 134. ``Ah, eu sei, meu irmão, eu sei mesmo, meu irmão''

% 135. ``Eu sei, meu irmão, só que eu não sabia. Pensei que você fosse branco''

% 136. ``Não sou branco não, meu irmão. Sou eu!''

% 137. ``Eu sei que você é gente, por isso vim me juntar a você''

% 138. ``Está bem!'' Os homens começaram a ficar juntos, ficar juntos, {[}e então{]} foram embora.

% ~

% 139. Meu finado pai levou o finado pai de Nyorem.\footnote{O narrador se
%   refere ao pai da Lourdes Karitiana (Nyorem).}

% 140. ``Onde é o rio \textit{Bisỹ Herednemo}, meu avô?''

% 141. ``É para cá'', falou meu finado pai, e foi

% 142. Foi o finado meu irmão, e ficou com medo do finado pai da Augusta,
% que foi atrás dele

% 143. Saíram, foram embora, e um da família dele também foi com ele

% 144. Depois, ele disse para o nosso pessoal: ``Onde está o lugar dos
% macacos- aranha?''

% 145. ``Vou caçar'', ele disse

% 146. ``Vamos caçar'', disse o finado pai do Epitácio. O Pereira era pequeno.

% 147. Então foram. Eu fiquei sozinho

% ~

% 148. ``Você sabe onde os macacos-aranha ficam, meu tio?'', falou o finado pai do Antônio Paulo

% 149. ``Lá, onde fica o meu caminho de caça, tem macaco-aranha''

% 150. ``Vou lá''

% 151. ``Vamos lá, meu tio''

% 152. Então correu até não dar mais para alcançar, dizem que não dava; ele fugiu

% 153. Andou, andou, andou

% 154. Eu fiquei com eles, sozinho, então comemos batata, comida

% 155. ``Ah'', ele foi falando, ``você sabe onde está a caça, meu cunhado?''

% 156. ``Eu vi mesmo, sobrinho. No lago aqui perto tem muito macaco-aranha''

% 157. ``Vamos ver lá''

% 158. ``Então vamos'', eu disse

% 159. Nós não caçamos, não, nós ficamos sentados, conversando

% ~

% 160. Fomos, andamos, continuamos, andamos, andamos

% 161. ``Aqui tem mutum'', olhamos, chegamos no lago, não tinha mutum

% 162. Tinha macaco-aranha, ``\textit{poo}, \textit{poo}'', faziam os
% macacos-aranha. 

% 163. ``Aqui tem, vamos matar este macaco-aranha''

% 164. ``Eu vou matá-lo'', ele disse

% 165. ``Mata!''

% 166. Eu não estava agachado, eu estava em pé

% 167. Foram, atiraram lá longe no macaco-aranha e no quati, atiramos cinco
% vezes

% 168. Nós nos reunimos e andamos

% ~

% 169. ``Eu só matei bicho ruim!''

% 170. Eu pensei que era mucura\footnote{Mucura é de uso regional. O animal
%   é conhecido em outras regiões do Brasil como gambá. Nome científico:
%   \textit{Didelphis virginiana}}

% 171. ``Em qual você atirou?'', eu disse

% 172. ``Eu só matei bicho ruim. Eu não atirei em bicho bom não''

% 173. ``O que foi que você matou? O quê? Irara\footnote{Nome científico:
%   \textit{Eira barbara}}?''

% 174. ``Não é irara não, é quati''\footnote{Nome científico: \textit{Nasua
%   nasua}}

% 175. ``Por que você não traz?'', eu disse

% 176. ``Você come quati?''

% 177. ``Eu como quati'', eu disse

% 178. ``Está bem, eu também como''

% 179. ``Pega'', eu disse

% 180. Então foi deixar a espingarda dele de novo

% ~

% 181. Andaram, andaram, pegaram

% 182. Trouxeram de novo, andaram, reuniram-se aqui

% 183. Nós nos sentamos em cima do tronco, então não andamos mais, só
% ficamos falando

% 184. Nós nos sentamos.

% 185. ``Já pegamos caça, e agora vamos esperar os homens aqui''

% 186. ``Eles vêm vindo lá. Nós íamos deixá-los''

% 187. Contou, contou o que deram para ele, como viviam naquele tempo

% 188. Contou como pegou a mulher do outro, contou o que deram para ele,
% falou, falou

% 189. Falaram, falaram, terminou, falaram, falaram, terminou

% 190. Falaram, falaram da mulher namoradeira, acabou

% ~

% 191. Chegou o irmão mais novo dele

% 192. ``Vocês estão chegando?''

% 193. Chegaram os amigos dele

% 194. ``E meu tio?''

% 195. ``Meu tio não está aqui não, ele já foi, não dá mais para
% alcançá-lo, tio. Eu acho que ele foi em direção ao Igarapé Preto''

% 196. ``Para você ele não foi lá não, meu sobrinho?''

% 197. ``Não, nós estávamos juntos''

% 198. ``Vamos, que está escurecendo''

% 199. Então foram embora, foram

% ~

% 200. Andaram, andaram, andaram.

% 201. ``Mais para lá tem mutum'', eu disse

% 202. Andamos, andamos, andamos mais um pouquinho, o mutum voou

% 203. ``\textit{Pij̃bok bok xak}'', e atiramos no mutum

% 204. Então o amigo dele atirou e eu fiquei pensando

% 205. ``Não se pode matar caça do outro''

% 206. Atiraram, atiraram, aquele que o matou se alegrou, o amigo dele

% 207. ``Eu o matei, meu amigo'', ele disse fingindo

% 208. ``Vá pegá-lo, meu amigo'', ele disse

% 209. Eu fiquei pensando que aquele que estava comigo não o matou

% 210. Aquele que estava comigo matou só quati

% 211. Estava um cheiro ruim,\footnote{Referências a quatis que é uma das
%   caças com cheiro forte, na região amazônica esse cheiro ruim é também
%   conhecido como pitiú.} não dava para lavar, não dava para comer

% 212. Mutum, macaco, quati, macaco-aranha, nambu azul, todos os homens
% foram até a caça, acabou

% 213. Só eu mesmo fiquei por aqui, os homens foram lá

% 214. Então os homens foram

% ~

% 215. Nós fomos, fomos

% 216. ``Leva a gente lá, meu primo'', eles disseram

% 217. ``Leve-me, meu irmão'', disse o finado pai desta (Augusta).  Vamos,
% meu cunhado'', ele disse para mim

% 218. ``Eu vou ficar aqui'', eu disse, quando ele falou assim para mim

% 219. ``Ah, eu vou''

% 220. ``Ah, então vai'', eu disse

% ~

% 221. Depois de cinco dias, vieram de novo para cá

% 222. Aquele que o inimigo matou estava doente

% 223. Minha família estava doente

% 224. Quando o filho sarou, o pai dele estava lá, à espera do filho

% 225. Dormimos cinco dias, então voltamos para cá

% 226. Então ele voltou para onde eu estava, apesar de eu não acreditar que
% ele fosse voltar

% ~

% 227. No caminho, eu o encontrei

% 228. Encontrei-o na lagoa, o Jere'op Ambyjywyp

% 229. No caminho ele ficou doente

% 230. Eu tinha mandado o outro encontrá-lo

% 231. Chegaram de novo

% 232. Depois me levaram sozinho

% 233. Como este daqui

% 234. Eu fiquei realmente sozinho

% 235. Lá na casa dele

% 236. Eu fui de novo sozinho, assim

% 237. Eu não tinha nada

% 238. Do meu grupo só tinha eu

% ~

% 239. Andou, andou, andou. Disse o meu finado pai:

% 240. ``Não mexe com o pessoal, não mexe com as crianças'', disse meu finado pai

% 241. ``Depois, vocês vão sair'', disse o meu finado pai

% 242. Muitas mulheres, muitas crianças, havia lá

% 243. Já que ele falou comigo, ``eu estou indo lá'', eu disse

% 244. ``Vamos. Se você não me convidasse, eu não iria''

% ~

% 245. Andamos, andamos, continuamos só com ele

% 246. ``Eu vou levar seu filho'' --- mas se ele não tivesse falado com o meu
% finado pai, eu não iria --- ``vou levar o seu filho, meu irmão''

% 247. Como ele me falou, eu fui

% ~

% 248. Andamos, andamos, continuamos

% 249. Andamos, andamos, chegamos

% 250. Eu entrei sozinho

% 251. ``Cheguei!''

% 252. ``Vem'', disseram os homens

% 253. Enquanto nós estávamos rindo, as mulheres disseram: ``O macaco-aranha gritou''

% 254. ``O macaco-aranha gritou, vão caçar! Mata macaco-aranha para nós!''

% 255. ``Ah! Dizem que o macaco-aranha gritou'', nós dissemos. ``Vamos atirar!''.

% 256. Fomos caçar.

% 257. Então fomos.

% 258. ``Aqui está a munição do seu tio'', eu disse

% 259. Pegaram umas cinco

% 260. Eu peguei e fui embora

% ~

% 261. Fomos, fomos, fomos

% 262. ``A anta vai ser minha. A anta vai ser minha, você vai ver''

% 263. ``Nós vamos matar anta!''

% 264. ``Vamos! Vamos matar antas, meu cunhado''

% 265. Assim disse o finado pai de Eremby Byyt

% 266. Fomos, andamos, andamos, ouvimos o barulho do macaco-aranha e do
% macaco-prego

% 267. ``Aqui tem caça!''. Os macacos gritaram, ele correu na direção deles

% 268. Ele correu, parou. Ele atirou, acertou um

% ~

% 269. Então chegou o gavião \textit{pãram}\footnote{Espécie de gavião que
%   sinaliza a presença de anta naquela localidade, nomeado ``gavião da
%   anta''.}

% 270. Gritou, passou por cima do amigo dele, gritou, gritou

% 271. Chegou de novo outro gavião logo atrás desse

% 272. Gritou, gritou e a anta respondeu

% 273. ``\textit{Rinh}'' disse a anta

% 274. ``\textit{Rinh}'', enquanto ele estava lá

% 275. ``Ah, a anta! Eu vou matar esta anta!''

% 276. ``Eu não vou gritar não''

% 277. Eu corri

% 278. Corri, andei, andei, andei

% 279. Quando eu estava em pé, falaram:

% 280. ``Onde está, meu cunhado?''

% 281. ``Onde está, meu cunhado?''. ``Aqui tem anta'', eu disse

% 282. ``Vamos matá-la''

% 283. ``Vamos, sim!''

% ~

% 284. Andamos, andamos, olhamos, olhamos cuidadosamente. Havia vários jacamins

% 285. Os jacamins estavam parados e fazendo muito barulho

% 286. ``Parece que aqui tem jacamins''.

% 287. Cercaram ao redor deles, mas os jacamins não voaram

% 288. Andamos, andamos, andamos chegamos

% 289. Dando a volta na subida do tronco caído

% 290. Andei, andei, dei a volta por cima deles

% 291. Eu olhei para cá, por cima do tronco

% 292. Elas estavam lá

% 293. Eu olhei e as vi imediatamente

% 294. Abaixei os olhos e logo a vi, embaixo

% 295. ``Olha aqui''

% 296. Mirei a arma na direção da anta

% 297. ``Quem vai começar a atirar?''

% 298. ``Eu vou atirar primeiro'', eu falei

% 299. ``Se você não atirar, eu vou atirar, meu cunhado'', ele disse

% 300. ``Ah'', ele disse quando apertei o gatilho

% 301. Eu atirei, ele também, várias vezes

% 302. ``Matamos a anta!''

% 303. ``Vamos! Aqui também tem caça''

% 304. ``Vamos, meu cunhado'', eu disse

% ~

% 305. Andamos, andamos, chegamos no caminho

% 306. ``No que você atirou?'', disseram suas mulheres

% 307. ``Nós não matamos, minha tia'', eu disse

% 308. ``Nós colocamos os macacos-aranha para correr''

% 309. ``Acabou minha munição''

% 310. ``Eu estou envergonhado de não ter conseguido'', eu disse

% 311. ``Ah! Como você atirou, meu sobrinho, para onde?'', ela disse

% 312. ``Não matou mesmo?'', disse de novo para o marido dela

% 313. ``Nós matamos apenas tamanduá magro. Nós matamos''

% ~

% 314. Ah, eu não pensei que as mulheres estivessem assim tão agitadas

% 315. ``Ah! Vocês mataram anta, não mataram? Vocês mataram anta para vocês?''

% 316. ``Matamos mesmo, minha avó''

% 317. ``Tudo bem. Foi anta que vocês mataram.''

% 318. Meu pai foi atrás do irmão dele

% 319. ``O meu irmão estava lá, endireitando uma peça de madeira."

% 320. ``Chame-o!''. Foi lá com ele. Foi

% 321. Andou, chegou. ``Os homens mataram a anta'', ele disse

% 322. ``Ah! Mataram anta! Vamos! Vamos cortar a anta'', ele disse

% 323. Foram

% 324. ``Eu cheguei, você matou a anta''

% 325. ``Eu matei mesmo''

% 326. ``Lá só tinha anta magra''

% 327. ``Para mim não é ruim''

% 328. ``A anta está magra, mas está boa'', ele disse

% ~

% 329. Então eles vieram

% 330. Fomos, então foram todos, as mulheres limparam a casa

% 331. As mocinhas e as crianças também foram

% 332. ``Olha aqui, olha'', nós dissemos!

% 333. ``Ah! Vocês mataram a anta boa'', disseram os homens para nós

% 334. Ele a cortou, viu a gordura branca

% 335. ``Não há carne na nuca, mas há muita gordura. Só sai gordura''

% 336. Ele a repartiu, repartiu, e então acabou

% 337. Nós fomos embora e alcançamos o caminho

% 338. Nós assamos a anta, assamos, assamos, assamos, terminou

% 339. Os homens ficaram todos loucos

% 340. Os homens cantavam e enlouqueciam por causa da gordura da minha anta

% 341. Eu não participei, porque eu sou dono da anta

% ~

% 342. Dois dias se passaram

% 343. ``Daqui a três dias eu vou embora'', eu disse

% 344. ``Vai lá, você vai sozinho?

% 345. Vá, você vai sozinho''

% 346. ``Eu vou sozinho''

% 347. ``Vai atrás dele, vai'', falaram para o finado pai do Sebastião

% 348. ``Vai atrás dele'', falaram para o finado pai do Epitácio quando o
% Pereira era pequeno

% 349. Nós viemos em três

% 350. Voltamos, chegamos na aldeia

% ~

% 351. Voltamos, eles tinham matado queixadas

% 352. Atiraram, flecharam, atiraram quando nós estávamos vindo para cá

% 353. Andamos, andamos, chegamos, deixamos a caça no chão

% 354. ``Cheguei, meu pai. Eu estou com a anta, que está muito gorda. Eu matei esta anta, meu pai''

% 355. ``É a parte de trás dela?''

% 356. ``Não é a parte de trás não, é que a costela da anta é gorda. Está gorda''

% 357. O meu finado pai pensou: ``Será que ele vai me dar a parte do umbigo?''

% 358. Nós chegamos

% 359. Os homens foram embora

% ~

% 360. Assim nós aparecemos

% 361. Quando estávamos prestes a sair, eles chegaram de novo sozinhos

% 362. No local onde nos pegaram, os homens se reuniram

% 363. Depois que eles foram, outros chegaram também

% 364. Os velhos ficaram lá mesmo

% 365. Um homem chegou e o outro presenteou a filha para ele

% 366. O meu finado pai pedia uma mulher

% 367. ``Consegue uma para mim também'', meu irmão disse

% 368. Assim é que foi

% 369. Nós fomos de novo

% 370. Chegamos de novo aqui

% 371. Então nós ficamos aqui com eles

% 372. Nós ficamos e não voltamos para lá

% 373. Acabou aqui mesmo

% ~

% 374. Nós viemos para cá, a outra parte voltou, mas não para lá

% 375. Não havia mais gente aqui, ficamos lá com eles

% 376. O finado pai desta aqui ficou lá

% 377. O meu finado pai ficou aqui sozinho

% 378. Pojepap ficou aqui sozinho

% 379. Nós todos ficamos lá

% 380. Nós ficamos sozinhos por muito tempo

% 381. Ficamos por muito tempo

% ~

% 382. Então ele {[}Moraes{]} morreu, a doença do branco o matou

% 383. Então meu finado pai se juntou

% 384. Quando o Moraes morreu, nós ficamos juntos

% 385. Nunca mais nos separamos

% 386. Assim nós ficamos, assim

% 387. Não havia mais homens quando ele morreu

% 388. Os homens ficaram com o meu pai

% 389. O meu finado pai trouxe todos para cá

% 390. Então o meu pai morreu aqui

% 391. Então eles não foram mais na aldeia deles

% 392. Aqui meu finado pai morreu

% 393. Assim o pessoal ficou junto

% 394. Não há mais o que contar, esse é o final



% \chapter*{O Encontro de dois grupos locais}

% 1. Ytakatat andyk ta’ãt yn, yti pop tykiri, ymyrỹta

% 2. Ytakatat andyk ta'ãt yn hoop

% 3. Ambyyk ytaotyt oko ta'ãt yn hak

% 4. Ytaakat hak, ytaakat hak

% ~

% 5. Ambyyk ytakatat okot hoop, aldeiasogng

% 6. Ambyyk ytaso'oot taso katapaty

% 7. Katapaty, iiso bity

% 8. Iiso bity ho horot

% 9. Katapa hyk

% 10. ``Yjxa iaka minãt haka? Yjxamon haka?''

% ~

% 11. Nakampok ta'ãt ot'ep ymbykiit

% 12. Ambyyk, naymbykyj ta'ãt a

% 13. ``Katapo j̃a?'', ytaka’a ta’ãt yn

% 14. Ambyyk, naymbykyj ta'ãt ymbykiit

% 15. Ymbykyj̃

% 16. ``Morã katapamon j̃a? Opoko katap nakaj̃an?''

% 17. ``Yjxa naka'agit hak'', naka'at ta'ãt ymbykiit

% 18. ``Yjso'oot hoop'', ytaka'at ta'ãt yta, ``opoko katapaty''

% 19. ``Yjso'oot hoop''

% ~

% 20. Nakamboop pa

% 21. Imbodni pa

% 22. Ambyyk, ytaotyt okot ta'ãt yta

% 23. Ambyyk, ytakatat pap, ytakatat okot yn ymyrỹta tyym

% 24. Koj, nayryt yandikit ymbykiit

% 25. ``Yjso’oot amyrỹ orotaty. Amyrỹ araka oko'', iri’aj ymbykiit

% ~

% 26. Ytakatat ta'ãt yn

% 27. Terek terek terek, otam

% 28. Ambyyk ytakoro’op pasap ta’ãt yn, ese pityp yta’agngj̃i’oot

% 29. ``Yn ipypĩ yn ese haka'', ytaka’at ta’ãt yn

% 30. Ytaso'oot ta'ãt yn idep otaty yn

% 31. Jyjy kip esety

% ~

% 32. Ytaso'oot ta'ãt, ytakatat oko ta'ãt, ytakahot okot ta'ãt, otam Popat

% 33. O'anat nakakit ta'ãt taso

% 34. ``Taso naakat ma'', naka'at ta'ãt ymbykiit

% 35. Ytakahot ta'ãt atyym yta

% 36. Yno yrypat iakat

% 37. Terek terek terek otam, iri'aj iambisogng

% 38. Taambisogng toroko'o, iri'aj opok ako ma

% 39. ``I’agngi j̃a!''

% ~

% 40. Terek terek terek otam

% 41. Jygng jyk yta ipityp

% 42. Nakapi ta'an ymbykiit

% 43. ``Opok naakat'', naka'at ta'ãt ymbykiit

% ~

% 44. Yjotyyt yjhot oko yjxa, hot okodn

% 45. Ambyyk, naymbykyj oko saryt taso

% 46. Taso kywyti kywyti

% 47. ``Ãh, ipopit naakaj opok!''

% 48. ``Yjso'oot y'i! Kahyt iasogng''

% 49. ``Akywyti hap, aramemy''

% 50. Otit nakaki ta'ãt opok otity

% 51. Atykiri ytaamej̃kit ta’ãt yta

% 52. Ytaandikit naymbykyj i

% ~

% 53. Atykiit naymbykyj ta'ãt, nakaheren ta'ãt i, taso

% 54. Okõrã paraki itikydnan

% 55. Okõrãpip i’adj̃i’oot 

% 56. Naapit ta'ãt Eremby Byyt bykiit

% 57. ``Yn napopit bisỹ'', para’i

% 58. ``Ambyyk, ytayryt ta'ãt'', ypara'i

% 59. Ytakatat ta'ãt

% 60. ``Ytari ỹryn yn ikydna andyki ỹryn bisỹ pyt’yypa''

% 61. ``Ãh'', iri'aj ymbykiit

% ~

% 62. Ytakatat yn sik sik sik, dok

% 63. Napyrombop tyj̃a saryt i

% 64. ``Hong'', naka’at saryt i. Hong, j̃a i’adnj̃i'', naakat saryt taso

% 65. ``Hong, hong, hong'',

% 66. Otam, naka'adn saryt ambip

% 67. ``Koroj̃a ’a, koroj̃a ’a, i’adnj̃i j̃a'', naka’at saryt

% 68. ``An ioky myndo, õẽ?'', naka’at saryt. Õẽ?'' (pongyp)

% 69. ``Õẽ, hakatyym naka’adj̃i naono horo ’a''

% 70. Naka'a saryt Eremby Byyt bykiit

% 71. Sosyp ba, naka’at saryt i ohỹrym i

% 72. Poogn, som sõwõrã pitat impan

% 73. Nakampi i

% ~

% 74. Sosyp, iri'aj, tek tek, jyk

% 75. ``A’adj̃a, õẽ?'', iri’aj taotaty. A’adj̃a, õẽ'', otam

% 76. ``Hak naakat, aso'oora ikyn''

% 77. Naka’at saryt j̃a bykiitity

% 78. Nakatat saryry i kyn

% 79. Sosyp, ba

% 80. ``Koroj̃a ’a!''

% 81. ``Ãh, an ympi, õẽ''

% 82. ``Amboryra, õẽ''

% 83. ``An yoky kymini, õẽ?''

% 84. ``Yn aokyj, õẽ, yn aoky padni, õẽ, yjxa naakat yn, õẽ''

% 85. ``An ympi tykii, õẽ, ampi pitaty. Ynty asotidn tykiri amyrỹ’i, õẽ''

% 86. ``Ysotidna yno, amboryra kagngoop''

% 87. ``An yokyj kymini, õẽ''

% 88. ``Aoky padna! Aoky padna! Aheredna myn'i''

% ~

% 89. Namboryt saryt i

% 90. Tek, tek, atop

% 91. ``Hok ipita'', naka’at saryt taso ``Amyrỹtat a’a tyka hỹ?''

% 92. ``Akot, hoop naakat yota'', naka'a saryt i

% 93. Jyk, naka'otowak saryt iohen

% 94. ``Aohen i’ot tykat, õẽ a’a'' naka’a saryt isaka’et

% 95. ``I’oto padna, õẽ, i’oto padna myn’i, õẽ. Anty asotidn tykiri amyrỹ’i, õẽ''

% 96. ``Ãh, mo’i sombawaki õẽ iri’aj saryt iota''

% 97. ``Ajxa ympi myn’i, õẽ''

% 98. ``Morãmon y'i asat?''

% 99. ``A naka horo ysat'', naka'a saryt i

% 100. ``A ysyp’et'', naka’a saryt j̃a bykiit. Ysyp’et naaka my’anan an''

% ~

% 101. ``Aota he?''

% 102. ``Hori yota''

% 103. ``Imbodna j̃onso, myhin naka’a tykat j̃onso''

% 104. ``Aa yjso'oot, yjse'y ikytopoty, yjpyt'y him ty''

% 105. ``Him naka'a tykat ipyp''

% 106. ``Taso pykop naaka horo haka''

% 107. ``Morã ioky 'õromo?''

% 108. A naoky horo ytaam'a saryt yn

% 109. ``Ah, yjso'oot himty''

% 110. Jyryj jyryj jyryj otam

% 111. kinda pe'ep yn nam'a tyki'oot

% 112. ``Ãh yjso'oot, yjso'oot ysyp'et, yjse'y kytopoty''

% ~

% 113. Kahot terek terek otam

% 114. Ytaka’a j̃a yn

% 115. ``Ah, ytaymbykyj yta'', naka’at taso. ``Ah, ytaymbykyj yta, aj̃ing tyym yta’agngĩ yta''

% 116. ``Ah, yjxa nakam'an ajxat''

% 117. ``Yjxa ytat, yjxaa, Byyjyty 'osop aky ytat''

% 118. ``Ambo kangoop, naambot ta'ãt ypyp''

% 119. Marang marang ypyp

% 120. ``Ãh'', ytakahadn ta'an ityyta. ``Ta'i apyeso?''

% 121. ``Imbodna ypyeso, hoop semsogng itato y’ito'', iam’a ta’aj̃ ymbykiit

% ~

% 122. Hadn hadn, ytaka'at ta'ãt yn

% 123. Otam kayt ta'ãt ymbykiit kangoori

% 124. Tek tek otam

% 125. ``mynda taso iymbykyj'', naka'a ta'ãt ymbykiit

% 126. ``Iymbykyj̃? Morãmon hỹ? Taatymon hỹ?''

% 127. ``Aka padni, i 'it naakat''

% 128. ``Ojo'a'', nakahot ta'ãt

% 129. ``Hyng'', iota, tõroj̃ tõroj̃

% 130. ``A’a tyka ykeet hỹ?'', para’a ti’ĩ i

% 131. ``Y'a tyki, yhaj, ambo ano, yhaj?''

% 132. ``Y'a tyki yno, ykeet'', naka'a ta'ãt ymbykiit

% 133. ``Y'a tyka'oomi yno aandikitat. Ynty anapi horo ykeet?''

% 134. ``Ãh ysoyng horo yhaj, ysoyng horo ykeet para’a ti’ĩ i''

% 135. ``Ysoyng horo ykeet, ytasoyng pita ykeet, opok akat ytakakãran an''

% 136.``Opok aka, yhaj, yn naakat yhaj!''

% 137. ``Yn naakat isondyp ajxat, yn naakat iherednan''

% 138. ``Ojo'a!'' Nakaredn'oot taso, here here naka'at taso, nakahot taso

% ~

% 139. Ybykiit natooto'oot Nyorem bykiit.

% 140. ``Tihoori i Bisỹ Herednemo, yowoj hỹ?'', naka’a ta’ãt

% 141. ``Hot taakat para’a ti’ĩ ybykiit'', nakatat i, nakatata’ot ta’ãt i

% 142. Koj, ipi ta'aman, ytaoty, Antônio Paulo bykiit, nakahot Eremby Byyt bykiit

% 143. Atop, koj atytap tyym taso myhin tyym, ipyeso tyymo

% 144. Ambyyk, ``imbodna midni 'õrom herednipa?'', nam 'a ta'ãt yjiriso

% 145. ``Ypõrĩ yry yno'', naka’a ta’ãt a

% 146. ``Yjpon ry'i'', naka'a tã'ãt Kybygngã bykiit. Pereira sino.

% 147. Koj koj pi'ina tat yta yno, yn myry'ino ymyrytat

% ~

% 148. ``Aso'ooto mini an 'õrom herednipaty, ysyp'et?'', nam'a ta'ãt
% Antônio Paulo bykiit

% 149. ``Hoop nakakit 'õrom yponpap''

% 150. ``Yso'oori yry yno'akyn''

% 151. ``Yjso'oory'i ysyp'et''

% 152. Andikitat itat tim'an, ipykydn tin'an, i'a heredn nodoto padnoko
% saryty padni'a; pyfogidn a

% 153. Ho ho ho ho ho tim'at

% 154. Napi’idn yn myrỹ’ino kat itytat tyym, takasyp andyk yta pyt’y pyt’y pyt’y, yta ohyty, ti’yty

% 155. Hỹ hadn’a tyso tyyn i, ``aso’ooto mini an him hereni paty bosy?''

% 156. ``Ypyso'ootyn soro'a ysyp'et, ejo pip hak naka'agngit 'õrom hak
% bypiit''

% 157. ``Yjso'oot yjxa ari''

% 158. ``Yjso’oot y’i'', ypara’a ti’ĩ yn

% 159. Yta napon pon pibmki ta'ãt padni, ytambikira ytakasyp ytahadna pip

% ~

% 160. Koj koj, terek terek ytaka'a andyk, terek terek terek,

% 161. ``hak naka’agngi bisỹ'', sypo hirã otam ejosok boop bisỹ

% 162. Nakaki ta’ãt ’õrom, ``poo poo'', para’a ti’ĩ ’õrom

% 163. ``Mã'ah, mã 'õrom a, yjxa ioky''

% 164. ``Yn iokyko'', para ’a ti’ĩ

% 165. ``Ioky!''

% 166. Yokysera ti’ĩ yn jyk ytaka ’a ta’ãt yn

% 167. Ho ho ho moj̃a napynpon’a tyka tyn ’õromo, irisa peng peng peng peng peng yjpyt

% 168. Nakaheredn tek tek tek

% ~

% 169. ``Kinda sara pita myrymon yntipopiit''

% 170. Dokon ãkãran ytakakãran ta'ãt yn

% 171. ``Morã kyyn apon pona' y i?

% 172. ``Kinda sara pita myrymon yntipopiit. Kinda hãraj̃ kyyn pon pon yaka ynta''

% 173. ``Morãmon y'i an tioky tyka? Morãmon? Ombaky eemo mon?''

% 174. ``Ombaky eem aka, irisa nakat''

% 175. ``Morãsogng an i taherednaki y i'', ytaka 'a ta'ãt yn

% 176. ``An i'y irisa?''

% 177. ``Yn naka'yt irisa yn'', ytaka'a ta'ãt yn

% 178. ``Ojo'a, yn naka 'yt yn tyym''

% 179. ``Ipii'', ypara’a ti’ĩ yn

% 180. Nakatat oko ta'ãt pak tam panty

% ~

% 181. Ho ho ho ho jydn

% 182. Natambykyj oko ta'ãt, syyk here, mã'a

% 183. Dok dok ’ep okyp yta’a tykiri, ytangat oko, ytakasyp hadna myrỹ’in

% 184. Dok dok

% 185. ``Yjhodn yjxa, yjxa ikydna taso haka''

% 186. ``Pyryryt tysypyn i a, iaopyj tyka''

% 187. Hadn hadn ta'a hitit, hadn hadn keerep takity, hadn hadn tairiso
% ta'an tisoojototy

% 188. Hadn hadn ta'a hitity, hadn hadn

% 189. Hadn hadn tõ, hadn hadn tõ

% 190. J̃onso pymãra katy, hadn hadn hadn boop

% ~

% 191. Nayryt ta'an ikeet, yrydn

% 192. ``Ajymbykyj tyso?''

% 193. ỹryn iota

% 194. ``Yjsyp'eto he?''

% 195. ``Ibodna yjsyp’eto, pyrytatyn yjsyp’et, iaheren nodorak yjsyp’eto, j̃oj̃ biit omirim nakapykyn ynty''

% 196. ``Ãh, iamini y'it ongoot anty?''

% 197. ``I’a padni. Akot ytaka’agngĩt, ytaamyni yta''

% 198. ``Yjhot, tamooj igo''

% 199. Atykiri imbodn oko padni, nakahot

% ~

% 200. Terek terek terek terek

% 201. ``Onym naka’a tykat bisỹ'', ytaka’a ta’ãt yn

% 202. Terek terek terek ’it, nakatam ta’ãt bisỹ

% 203. ``Pij̃ bok bok xak'' bisỹ napynpon ta’ãt

% 204. Nakapon ta'an iota, ytakoro'op oky ta'ãt yn

% 205. ``iaj̃oni taota hot aoky'', ytakoro’op oky ta’ãt yn

% 206. Pyng pygng naosedna'oom ioky tykiiri, iota

% 207. ``Yn ioky ỹryn yota'', naka’a’oom

% 208. ``Iora pita yota'', naka'a ta'ãt i

% 209. Ytyyt iaka iokyki tykiri ypakoro’op oky ti’ĩ yn

% 210. Irisa myrỹ’in atapopiit ytyyt i’atyka, yjpyt

% 211. Hydnyn j̃ingã nam, pitat boop, amhoron byyky paap, a’y byyky paap

% 212. Bisỹ, pikomo, irisa, ’oromo, pom’eem, taso’oot nakayyt him, boop

% 213. Yn myrỹ’in nakat ijygngan hakat, taso nakat ihot ’oop

% 214. Nakahot ta'ãt taso

% ~

% 215. Ytakahori yta, yjhot

% 216. ``Ytamhora ykeet'', naka'a ta'an i

% 217. ``Ymtara, yhaj'', naka’a ta’ãt j̃a bykiit. Yjso’oot bosy'', ytam’a ta’ãt yn

% 218. ``Hak ytaaka 'iri yn'', ytaka'a ta'ãt yn kahyt i'a tykiri

% 219. ``Ãh, ytakatari yno''

% 220. ``Ãh, a’a y’i, a’a’ooma y’i'', ypara’a tyj̃i yn

% ~

% 221. Cinco taakat naymbykyj oko ta'ãt kangoop

% 222. Pyrakinda otidna andyk ta'ãt opok tioky

% 223. Pyrakinda otidna andyk ta'ãt yjiriso

% 224. Ta'it oyty'oot nakasyp andyk hooop i, ymbykiit

% 225. Cinco nakakat, ambyyk naotyt okot kangoop

% 226. Ambyyk naotyt okot ykymbawak 'yt

% ~

% 227. Pap, ynaatĩ ta’ãt

% 228. Jere’op ambyjywyp okyp yn naatĩ ta’ãt

% 229. Pap, takinda otidn iatĩ

% 230. Ypyry’a ti’ĩ yn ity

% 231. Naymbykyj oko ta'ãt

% 232. Ambyyk ytaatoot oko ta’ãt yn ymyrỹtat

% 233. J̃a horot tyym

% 234. Ymyrỹta pitat

% 235. Hoop taambip

% 236. Ytakatat okot yn ymyrỹtat ho horot

% 237. Imbodni kinda yhot

% 238. Sypom yaki yn

% ~

% 239. Tek tek tek nakahaadn ymbykiit

% 240. ``Apyso’ĩ airisosok apyso’ĩ õwãsok'', naka’a ta’ãt ymbykiit

% 241. ``Ambyyk ajxa mombykyj'', naka'at ta'ãt ymbykiit

% 242. J̃onso ako õwã ako nakakit atat

% 243. Ihadnasogng myrỹ’in ``ytakatat tysyp'', ytaka’a ta’ãt yno

% 244. ``Yjso’oot yam’aakit yj̃yntaty''

% ~

% 245. Terek terek terek, nakahot oko ta’ãt, ityyt myrỹ’in

% 246. ``Yn naatoori a’it, ymbykiitikyym iakiip yj̃yntata padni. Yn naatoori a’it yhaj''

% 247. Yam'asogng ytakatat ta'ãt

% ~

% 248. Terek terek terek ytakahot oko ta'ãt yta

% 249. Terek terek terek otam ytaka'a ta'ãt yta

% 250. Tong, ytaka’a ta’ãt yn ymyrỹtat

% 251. ``Yyryt ỹryn!''

% 252. ``Ayryra!'', naka'a ta'ãt taso

% 253. Andyj andyj andyj, ytagngĩ’oot, ``’õrom ihook'', naka’a ta’ãt j̃onso ytaty

% 254. ``'Õrom ihook, 'õrom ihook, ny, ajporã! Ioky 'õromo ytahot!''

% 255. ``Hỹ ’õrom ihook saryt'', ytapara’a ti’ĩ yta. Yjpon!''

% 256. Ytakahot ta'ãt ytapon yta

% 257. Koj koj koj

% 258. ``Ma asyp’et bypan sypo'', yparam’a tyj̃i yn

% 259. Jydn, yjpyt

% 260. Kej, ytakatat ta'ãt yn

% ~

% 261. Ho ho ho,

% 262. ``'Irip naakaj yjhot. 'Irip naakaj yjhot so'oori''

% 263. ``Yjxa ioky 'iripo yjxa''

% 264. ``By! Yjxa ioky 'irip bosy''

% 265. Para’a ti’ĩ i Eremby Byyt bykiit

% 266. Nakahot ta'ãt terek terek terek poo poo naka'a ta'ãt 'õrom pikom

% 267. ``Ma himo!''. Poo poo nakapykyna'oom isambibm i

% 268. Te teet jyk pyng myhin

% ~

% 269. Atykiri nayryt ta'ãt pãram

% 270. Pãram pãram, kookodn, pãram, pãram, kookodn iota

% 271. Nayryt oko ta'ãt iota dikit

% 272. Pãram pãram i'a naakat 'irip tikyrot ta'ãt

% 273. ``Hij̃'', i’i ’iripo

% 274. ``Hij̃'', hoot iaka’oot

% 275. ``Ny, 'iripo, yn iokyj yryn 'iripo''

% 276. ``Hyy hyy hyy yry'i yn''

% 277. Ytakapykynan yn

% 278. Te teet ho ho ho

% 279. Yjygng tysypy'oot,

% 280. ``Tihori i bosy'', naka'a ta'ãt

% 281. ``Tihoot i bosy?'' ``‘Iripo hywa'', ypara’a ti’ĩ yn

% 282. ``Yjxa ioky!''

% 283. ``Hot yjso'oot!''

% ~

% 284. Terek terek terek sombaak sombaak nakaki ta’an syyj̃ ytaj̃on

% 285. Jyk jyk hiit hiit naka’a ta’ãt syyj̃

% 286. ``Hak naaka sõwom''.

% 287. Amorã otyyt taso iomon, yta imtengĩ syyj̃

% 288. Terek terek terek otam

% 289. `Epe'ot soka pyhaap okyp, aokyp

% 290. Tagng tagng tagng yparatat yn 'ep okyp

% 291. Sombaakatat ypara’a ti’ĩ yn hoop iokybm kat

% 292. Kat isyp ta'am'an

% 293. Sombaakatat so'oot ojdn, naka'a ta'ãt it

% 294. Pikybm ojdn naka'a ta'ãt i engoopat

% 295. ``Koro hyp a'a''

% 296. Irom ypara'a'oom ti'i yn ity

% 297. ``Morã iakaj ipon'oot?''

% 298. ``Yn naakaj ipon'oot'', ytaka'a ta'ãt yn

% 299. ``Aponki tykiiri ytakaporĩ yn bosy'', ytaka’a ta’ãt yn ity

% 300. ``Hỹ? Ah!'', Iam’a ti’ĩ ypymbagng tyso’oot

% 301. Kapon ta'an yn, peng peng i tyym peng yn tyym, peng, peng peng pygng

% 302. ``'Irip yjxa tiokyt''

% 303. ``Yjhot! Him tyym naaka horo hywa''

% 304. ``Yj hot, yjhot bosy'', ypaa’a ty’ĩ yn

% ~

% 305. Tek tek tek here

% 306. ``Morãkyn ajpon pona?'', naka'at ta'ãt isoojsok

% 307. ``Yta ioky padna ysokiit'', ypara’a ti’ĩ yn

% 308. ``Yta imtat yta 'õromo''

% 309. ``Bypan sypoty ypymbyyko''

% 310. ``Yombygymsogng y'a tykii padni yn'', ytaka'a ta'ãt yn

% 311. ``Ah! Tikat apon pon aka, ysyp hỹ?'', naka’a ta’ãt I

% 312. ``Iokyki pitat aj’a hỹ?'', naka’a oko ta’ãt tamanty

% 313. ``Yta naoky’oom, ojopy ’ewet myrỹ’in yta naokyt''

% ~

% 314. ``Ãh!'' Okybm tengan ikiity ytakakãran j̃onsoty

% 315. ``Ãh! ’Irip ajxa naokybm’an, ajxa ioky ’iripo hỹ ajx

% 316. ``Ytaoky pita, yowoj''

% 317. ``Ojo'a''. 'Irip naaka my'an ajxa tiokyt''

% 318. Nakatat ta'an y'it tahajakyn

% 319. ``’Ep todna hãraj̃xapip, ysyp’eto hoori ony?''

% 320. ``Iora!''. Nakatat ta'ãt akyn, koj

% 321. Tek tek otam. ``‘Irip itiokyt taso'', para’a ti’ĩ

% 322. ``Ah! ’Irip iaokybm’an! Yjso’oot! Yjxa i’ĩrã!'', naka’a ta’ãt

% 323. Nakahot, hoodn

% 324. ``Yrydn i, ajxa ioky 'iripo''

% 325. Yn ioky ỹryn pita!''

% 326. ``Hoori 'irip 'eweto''

% 327. ``Ynty isara'iri 'iripo''

% 328. ``Ta’ewet tykiri isara’iri ’iripo'', para’a ti’ĩ i

% ~

% 329. Naymbykyj ta'an

% 330. Yjso’oot! Koj koj pybm kẽremtat ambiky pirip j̃onso

% 331. J̃onso ’in’in tyym, õwã õwã tyym

% 332. ``Hyp a’a'', ytapara’a ti’ĩ yta

% 333. ``Ah! ’Irip hãraj̃ pita ajxa naoky hỹ?'', naka’at ta’ãt taso ytaty

% 334. Kiip poogng, naka'a ta'ãt iorojaty

% 335. ``Imbodni ipisyp ’irip orojadna pita sopahiwasok. Taoroja myrỹ’in nayryt''

% 336. Pak pak sang korot, sang korot, pyygng

% 337. Yjhot, ytakahẽryn ta’an yta

% 338. Yta nakamkõrong ta'an 'iripo, yta nakamhibm ta'an 'iripo imkõrong,
% imkõrong, imkõrong, pyygng

% 339. Namorenga tygng ta'ãt taso

% 340. Hỹryj̃ hỹryj̃ hỹryj̃ tamoreng tyj̃i’oot taso, y’irip orojaty

% 341. Yn igyrot ti’ĩ yn iongydn ta

% ~

% 342. Teng para'i hot, teng para'i hot

% 343. ``Myj̃ym kat ytakatari yn'', ytaka’a ta’ãt yn

% 344. ``Atara y’i, amyrỹta atari?''

% 345. ``Aso’oora y’i, amyrỹta atari''

% 346. ``Ymyrỹta ytakatari''

% 347. ``Io’ibma a’a'', param’a ti’ĩ Kỹõroj̃a bykiit

% 348. ``Io’ibma'', param’a ti’ĩ kymbygngã ymbykiit, Pereira sin tyymo

% 349. Myj̃ym ytaymbykyj

% 350. Ytaymbyj ta'an, terek terek otam akan

% ~

% 351. Ytaymbykyj yta. Sojxaty itioky ta'an ihot

% 352. Peng pygng paj syk pygng naka'a sõwon ta'ãt. Kangoop ytaymbykyj tyso'oot

% 353. Terek terek otam pak pygng ytaymbykyj yta

% 354. ``Yyryt ỹryn y’it''. ’Irip orojadna pitat tyyt ytaka’a tykat. Yhoto ka ’irip ytiokyt yn y’it''

% 355. ``Ij̃indo aka’oomang''

% 356. ``Ij̃indo aka’ooma padni, ’irip koronda inakymbi pisyp naakat. Orojadnan naakat''

% 357. ``Ipeet anapisokora yhoot?'', para’a ti’ĩ ymbykiit

% 358. Yrydn, yta

% 359. Naaka hot okot ta'ãt taso

% ~

% 360. Kahyt ytakaheredn ta'ãt yta

% 361. Kahot okowak yta aakot, naymbykyj okot i tamyrỹtat

% 362. Ytapiip, iotyyt otyyt gadnoko taako tyj̃i’oot taso, otyyt otyyt myrỹ’in nakakit

% 363. Ambyyk nakahot ta’an, ymbykyj̃ j̃a, j̃a tyymo iyryt yrydn

% 364. Ihyk hyk nakat ijygng jygngan hopap

% 365. Yydn napy'odn py'odn ta'itity taso

% 366. Ah napyromyj̃ tyj̃an ymbykiit

% 367. ``Ynty tyym ty apara'i'', ykeet naka'a ta'ãt

% 368. Hỹ para’a ti’ĩ i, kahyt iam’ a tyka

% 369. Hot okot yta

% 370. Ymbykyj okotaj̃ kangoop

% 371. Atykiri yta jygng jygng hoopap ipityp

% 372. Jyk ytaymbykyj oko kangoop

% 373. Pyygng pitat naka'at hak

% ~

% 374. Pyygng pita yta'i yta haka, ap ytaki yta ari

% 375. Ytambodnoko hak yta, aari yta jygng jyk yta'ipityp

% 376. J̃a bykiiti tyyt, jygng jyk a

% 377. Ymbykiit naakat hak iakat tamyrỹtat

% 378. Hak iakat tamyrỹtat Pojepap

% 379. Yta naakat hot ikit ytapitik

% 381. Ki ki ki pitat

% ~

% 382. Atyka naka'y nakambop i, naokyt opok oti

% 383. Morãsong ytaako pitadn, ymbykiit

% 384. Moraes apoposogng ytaako pitadn

% 385. Yta naamyn oko ywym

% 386. Kahyt ytakaki akat kahyt

% 387. Imbodn oko taso tata’agngĩt bowy tykiri

% 388. Ymbykiitisok kahot aokot taso

% 389. Ymbykiit nataambykyj kangoop ipiiti

% 390. Morãsong nakambop ymbykiit hak j̃a pip

% 391. Morãsong kahot oko pawyyt taakan, taambip

% 392. Hak paramboop ihak j̃apip ymbykiit

% 393. Kahyt naako akat taso kahyt

% 394. Imbodnoko iakara naakat ny iondetep

% \endgroup

% \chapter{Minibiografias}

% \paragraph{Barabadá Karitiana} foi um pajé Karitiana pertencente ao grupo Joari
% (também conhecido como Capivari). Ele narrou o encontro, vivenciado por
% ele, entre dois grupos de Karitiana que viviam em aldeias separadas, os
% Joari (ou Capivari) e os Karitiana. Estes grupos passaram a viver juntos
% na Terra Indígena Karitiana.

% \paragraph{Cizino Karitiana} é o atual pajé e cacique Karitiana. Ele narrou o ritual
% de iniciação masculina, intitulado \textit{Osiip}, pelo qual passou várias
% vezes.

% \paragraph{Garcia Karitiana} foi um cacique Karitiana. Ele narrou os mitos de origem
% do sol e da lua.

% \paragraph{Inácio Karitiana} é licenciado em Educação Básica Intercultural pela
% Universidade Federal de Rondônia e professor da Escola Indígena Estadual
% de Ensino Fundamental Kity Pypydnipa.

% \paragraph{Ivan Rocha} é doutor em Linguística pela Universidade de São Paulo e
% pesquisador visitante do Museu Paraense Emílio Goeldi, com bolsa do
% Programa de Capacitação Institucional (\textsc{pci}) do Ministério da Ciência,
% Tecnologia e Inovação (\textsc{mcti}/\textsc{cnp}q).

% \paragraph{João Karitiana} é licenciado em Educação Básica Intercultural pela
% Universidade Federal de Rondônia e professor da Escola Indígena Estadual
% de Ensino Fundamental e Médio Kyõwã.

% \paragraph{Luiz Karitiana} é licenciado em Educação Básica Intercultural pela
% Universidade Federal de Rondônia e professor da Escola Indígena Estadual
% de Ensino Fundamental e Médio Kyõwã.

% \paragraph{Nelson Karitiana} é licenciado em Educação Básica Intercultural pela
% Universidade Federal de Rondônia e professor da Escola Indígena Estadual
% de Ensino Fundamental e Médio Kyõwã.

% \paragraph{Valdomiro Karitiana} é filho de Barabadá Karitiana. Ele acompanhou a
% linguista durante a gravação da história do encontro entre os dois
% grupos locais e auxiliou na transcrição e tradução.

% \begin{comment}
% (Quarta-capa)

% Então ele (Moraes) morreu, a doença do branco o matou

% Então meu finado pai se juntou

% Quando o Moraes morreu, nós ficamos juntos

% Nunca mais nos separamos

% Assim nós ficamos, assim

% Não havia mais homens quando ele morreu

% Os homens ficaram com o meu pai

% O meu finado pai trouxe todos para cá

% Então o meu pai morreu aqui

% Então eles não foram mais na aldeia deles

% Aqui meu finado pai morreu

% Assim o pessoal ficou junto

% Não existe mais o que contar, esse é o final
% \end{comment}

% =======
% \chapter{Como foi feito este livro}

% \begin{flushright}
% \textsc{Íris Morais Araújo}\\
% \textsc{Karin Vivanco}
% \end{flushright}

% \noindent{}Este livro tem uma longa história. Em 1992, a linguista Luciana Storto
% iniciou sua pesquisa sobre a língua karitiana. Para poder estudar o
% idioma, ela gravou, neste e nos cinco anos seguintes, histórias
% tradicionais do povo Karitiana --- mitos de origem, rituais e narrativas
% históricas. As histórias foram narradas por Pereira Karitiana, Barabadá
% Karitiana, Garcia Karitiana, Antonio Paulo Karitiana, Cizino Karitiana,
% Joana Karitiana e Nazaré Karitiana, alguns dos homens e mulheres mais
% velhos de então, tidos como conhecedores da arte verbal. Com o
% importante apoio de interlocutores indígenas mais jovens, os atuais
% professores Nelson Karitiana, João Karitiana, Luiz Karitiana e Inácio
% Karitiana, bem como de vários outros falantes da língua, foram feitas as
% primeiras transcrições e traduções do material.

% Os linguistas trabalham transcrevendo e traduzindo as narrativas
% sentença a sentença. A passagem da fala para a escrita é o primeiro
% desafio colocado, já que em qualquer língua existem diferenças entre
% como se fala e como se escreve. Para chegar à escrita da fala, a maneira
% escolhida pela linguista foi ouvir cada sentença conjuntamente com os
% jovens karitiana com os quais trabalhou na transcrição e tradução,
% pausar o áudio, e ir decidindo o que permaneceria no texto transcrito e
% o que seria deixado de fora da transcrição. Este foi um modo de manter o
% conteúdo da narrativa e sua estrutura prosódica e artística, sem incluir
% os erros, hesitações e repetições não intencionais, naturais da fala
% ocorridas enquanto o falante busca na memória pelo próximo assunto a ser
% narrado.

% Neste livro, as frases numeradas incluíram mais de uma sentença quando
% foram pronunciadas com uma única entoação. Os linguistas fazem esses
% registros com muitos detalhes. Para eles, é importante saber como
% funciona cada parte de uma única palavra, chamada de \textit{morfema,} e
% cada palavra em uma frase. Para que essas informações estejam
% disponíveis para outros estudiosos, as sentenças são registradas em três
% linhas: (1) a linha do original na língua indígena, com um hífen
% separando cada morfema dentro das palavras; (2) a linha chamada
% \textit{glosa}, na qual se faz uma tradução para o português do
% significado de cada morfema; e (3) a linha contendo uma tradução
% aproximada da sentença inteira para o português; neste caso, o linguista
% por vezes precisa fazer escolhas entre uma tradução literal da sentença
% e uma tradução mais natural:

% \begin{enumerate}
% \item I-a-oky padni Gokyp (original)

% \item 3ªpessoa-passiva-morrer não Sol (glosa)

% \item ``Não se mata o sol/O Sol não pode ser morto.''

% (tradução literal/tradução escolhida)
% \end{enumerate}

% As transcrições apresentadas neste livro foram editadas para uma leitura
% confortável, mas se buscou preservar certas características da narrativa
% oral, como a repetição poética (repetição da sentença anterior com uma
% modificação, a fim de criar um efeito poético na forma ou no
% significado) e estruturas sintáticas comuns na língua karitiana: um
% exemplo são as inversões na ordem de palavras (por exemplo, ``Caça, o
% Osiip desnorteia'' em vez de ``O Osiip desnorteia a caça'').

% Nem todas as narrativas gravadas por Luciana na década de 1990 chegaram
% a ser transcritas e traduzidas, mas todas as que o foram e que não fazem
% parte deste volume serão publicadas futuramente em outro volume desta
% coleção.

% \chapter{Como pronunciar as palavras da língua karitiana}

% Neste livro, foi adotada a ortografia elaborada pela linguista Luciana
% Storto, que coordenou um programa de alfabetização da língua karitiana
% desenvolvido junto à comunidade na década de 1990 e aprovado por ela em
% 1996, quando as convenções ortográficas foram registradas no material de
% apoio ao aprendizado da ortografia karitiana, que tem sido usado desde
% então no ensino de sua língua materna. Atualmente, o grupo vem
% discutindo a reformulação de algumas dessas convenções ortográficas.

% \section{VOGAIS}

% \begin{itemize}
% \item[a] como ``a'' em ``até''

% \item[e] como ``e'' em ``mesa''

% \item[i] como ``i'' em ``idoso''

% \item[o] como ``o'' em ``hoje''

% \item[y] trata-se de um som intermediário entre ``i'' e ``u'', que não possui um
% equivalente no português do Brasil. Para pronunciá-lo, se pode falar um
% ``i'' e, gradualmente, mover a língua em direção a um ``u''. Quando a
% língua estiver em uma posição entre ``i'' e ``u'', esta será a pronúncia
% do ``y''.
% \end{itemize}

% \section{CONSOANTES}

% \begin{itemize}
% \item[b] como ``b'' em ``boto''

% \item[d] como ``d'' em ``dedo''

% \item[g] como ``g'' em ``gato''

% \item[h] como ``r'' em ``rato''

% \item[j] no início da palavra, se pronuncia ``dj''; no meio ``i'' como em ``saia''

% \item[k] como ``c'' em ``casa''

% \item[m] como ``m'' em ``mulher''

% \item[n] como ``n'' em ``nariz''

% \item[p] como ``p'' em ``pé''

% \item[r] como ``r'' em ``arara''

% \item[s] como ``s'' em ``sapo''

% \item[t] como ``t'' em ``tatu''

% \item[w] como ``u'' em ``água''

% \item[x] como ``tch'' em ``tchau''

% \item[`] uma pausa, como quando dizemos ``ã-ã'' com o sentido de ``não''.
% Corresponde a uma breve pausa entre as duas sílabas, que equivale a uma
% obstrução, na região das cordas vocais, do fluxo de ar que vem do
% pulmão, chamada de consoante oclusiva glotal no alfabeto fonético.
% \end{itemize}

% \chapter{Quem são os Karitiana}

% Os Karitiana são um grupo indígena ainda pouco conhecido no Brasil. Eles
% vivem no atual estado de Rondônia, considerado o lugar de origem da
% língua-mãe de todas as línguas Tupi. Os Karitiana falam a língua de
% mesmo nome, que é a única remanescente da família linguística Arikém, o
% que lhes confere uma importância central para os estudos comparativos
% das línguas tupi e, consequentemente, das línguas indígenas como um
% todo.

% Os Karitiana se aproximaram dos não indígenas durante o ciclo da
% borracha. Tanto a memória do grupo como os documentos não indígenas dão
% destaque para esses vínculos de trabalho caracterizados pela violência
% dos patrões. Nesse período, os não indígenas disseminaram entre o grupo
% diversas doenças, como a gripe e o sarampo. Por isso, em meados do
% século \textsc{xx} os Karitiana sofreram um grande declínio populacional,
% chegando a apenas 64 pessoas na década de 1970. Para que continuassem a
% existir, dois grupos locais decidiram viver juntos, casando-se entre si,
% e procuraram o Serviço de Proteção aos Índios, para que seus direitos
% como povo indígena fossem garantidos.

% No censo realizado pelo linguista Ivan Rocha em 2017, os Karitiana
% contavam com 397 pessoas.

% Atualmente, eles habitam sete aldeias, sendo cinco na Terra Indígena
% Karitiana, demarcada em 1986, e duas fora dela, em áreas que são parte
% do seu território tradicional. Algumas famílias também moram nas cidades
% rondonienses de Porto Velho, a capital, e em Cacoal.

% Além de trabalharem em atividades agrícolas, no manejo dos recursos
% florestais e na produção de artesanato, os Karitiana também são
% profissionais das áreas de saúde e educação. O grupo luta historicamente
% pela garantia de seus direitos, como a ampliação da terra indígena e o
% fortalecimento da educação e da saúde indígena.
% >>>>>>> b2f8e27e079774a74c1d5690901385d2d2807095



