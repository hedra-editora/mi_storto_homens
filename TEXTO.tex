\chapter{Como foi feito este livro}

\begin{flushright}
\textsc{Íris Morais Araújo}\\
\textsc{Karin Vivanco}
\end{flushright}

\noindent{}Este livro tem uma longa história. Em 1992, a linguista Luciana Storto
iniciou sua pesquisa sobre a língua karitiana. Para poder estudar o
idioma, ela gravou, neste e nos cinco anos seguintes, histórias
tradicionais do povo Karitiana --- mitos de origem, rituais e narrativas
históricas. As histórias foram narradas por Pereira Karitiana, Barabadá
Karitiana, Garcia Karitiana, Antonio Paulo Karitiana, Cizino Karitiana,
Joana Karitiana e Nazaré Karitiana, alguns dos homens e mulheres mais
velhos de então, tidos como conhecedores da arte verbal. Com o
importante apoio de interlocutores indígenas mais jovens, os atuais
professores Nelson Karitiana, João Karitiana, Luiz Karitiana e Inácio
Karitiana, bem como de vários outros falantes da língua, foram feitas as
primeiras transcrições e traduções do material.

Os linguistas trabalham transcrevendo e traduzindo as narrativas
sentença a sentença. A passagem da fala para a escrita é o primeiro
desafio colocado, já que em qualquer língua existem diferenças entre
como se fala e como se escreve. Para chegar à escrita da fala, a maneira
escolhida pela linguista foi ouvir cada sentença conjuntamente com os
jovens karitiana com os quais trabalhou na transcrição e tradução,
pausar o áudio, e ir decidindo o que permaneceria no texto transcrito e
o que seria deixado de fora da transcrição. Este foi um modo de manter o
conteúdo da narrativa e sua estrutura prosódica e artística, sem incluir
os erros, hesitações e repetições não intencionais, naturais da fala
ocorridas enquanto o falante busca na memória pelo próximo assunto a ser
narrado.

Neste livro, as frases numeradas incluíram mais de uma sentença quando
foram pronunciadas com uma única entoação. Os linguistas fazem esses
registros com muitos detalhes. Para eles, é importante saber como
funciona cada parte de uma única palavra, chamada de \emph{morfema,} e
cada palavra em uma frase. Para que essas informações estejam
disponíveis para outros estudiosos, as sentenças são registradas em três
linhas: (1) a linha do original na língua indígena, com um hífen
separando cada morfema dentro das palavras; (2) a linha chamada
\emph{glosa}, na qual se faz uma tradução para o português do
significado de cada morfema; e (3) a linha contendo uma tradução
aproximada da sentença inteira para o português; neste caso, o linguista
por vezes precisa fazer escolhas entre uma tradução literal da sentença
e uma tradução mais natural:

\begin{enumerate}
\item I-a-oky padni Gokyp (original)

\item 3ªpessoa-passiva-morrer não Sol (glosa)

\item ``Não se mata o sol/O Sol não pode ser morto.''

(tradução literal/tradução escolhida)
\end{enumerate}

As transcrições apresentadas neste livro foram editadas para uma leitura
confortável, mas se buscou preservar certas características da narrativa
oral, como a repetição poética (repetição da sentença anterior com uma
modificação, a fim de criar um efeito poético na forma ou no
significado) e estruturas sintáticas comuns na língua karitiana: um
exemplo são as inversões na ordem de palavras (por exemplo, ``Caça, o
Osiip desnorteia'' em vez de ``O Osiip desnorteia a caça'').

Nem todas as narrativas gravadas por Luciana na década de 1990 chegaram
a ser transcritas e traduzidas, mas todas as que o foram e que não fazem
parte deste volume serão publicadas futuramente em outro volume desta
coleção.

\chapter{Como pronunciar as palavras da língua karitiana}

Neste livro, foi adotada a ortografia elaborada pela linguista Luciana
Storto, que coordenou um programa de alfabetização da língua karitiana
desenvolvido junto à comunidade na década de 1990 e aprovado por ela em
1996, quando as convenções ortográficas foram registradas no material de
apoio ao aprendizado da ortografia karitiana, que tem sido usado desde
então no ensino de sua língua materna. Atualmente, o grupo vem
discutindo a reformulação de algumas dessas convenções ortográficas.

\section{VOGAIS}

\begin{itemize}
\item[a] como ``a'' em ``até''

\item[e] como ``e'' em ``mesa''

\item[i] como ``i'' em ``idoso''

\item[o] como ``o'' em ``hoje''

\item[y] semelhante a ``a'' em ``antes''
\end{itemize}

Trata-se de um som intermediário entre ``i'' e ``u'', que não possui um
equivalente no português do Brasil. Para pronunciá-lo, se pode falar um
``i'' e, gradualmente, mover a língua em direção a um ``u''. Quando a
língua estiver em uma posição entre ``i'' e ``u'', esta será a pronúncia
do ``y''.

\section{CONSOANTES}

\begin{itemize}
\item[b] como ``b'' em ``boto''

\item[d] como ``d'' em ``dedo''

\item[g] como ``g'' em ``gato''

\item[h] como ``r'' em ``rato''

\item[j] no início da palavra, se pronuncia ``dj''; no meio ``i'' como em ``saia''

\item[k] como ``c'' em ``casa''

\item[m] como ``m'' em ``mulher''

\item[n] como ``n'' em ``nariz''

\item[p] como ``p'' em ``pé''

\item[r] como ``r'' em ``arara''

\item[s] como ``s'' em ``sapo''

\item[t] como ``t'' em ``tatu''

\item[w] como ``u'' em ``água''

\item[x] como ``tch'' em ``tchau''

\item[`] uma pausa, como quando dizemos ``ã-ã'' com o sentido de ``não''.
Corresponde a uma breve pausa entre as duas sílabas, que equivale a uma
obstrução, na região das cordas vocais, do fluxo de ar que vem do
pulmão, chamada de consoante oclusiva glotal no alfabeto fonético.
\end{itemize}

\chapter{Quem são os Karitiana}

Os Karitiana são um grupo indígena ainda pouco conhecido no Brasil. Eles
vivem no atual estado de Rondônia, considerado o lugar de origem da
língua-mãe de todas as línguas Tupi. Os Karitiana falam a língua de
mesmo nome, que é a única remanescente da família linguística Arikém, o
que lhes confere uma importância central para os estudos comparativos
das línguas Tupi e, consequentemente, das línguas indígenas como um
todo.

Os Karitiana se aproximaram dos não indígenas durante o ciclo da
borracha. Tanto a memória do grupo como os documentos não indígenas dão
destaque para esses vínculos de trabalho caracterizados pela violência
dos patrões. Nesse período, os não indígenas disseminaram entre o grupo
diversas doenças, como a gripe e o sarampo. Por isso, em meados do
século \textsc{xx} os Karitiana sofreram um grande declínio populacional,
chegando a apenas 64 pessoas na década de 1970. Para que continuassem a
existir, dois grupos locais decidiram viver juntos, casando-se entre si,
e procuraram o Serviço de Proteção aos Índios, para que seus direitos
como povo indígena fossem garantidos.

No censo realizado pelo linguista Ivan Rocha em 2017, os Karitiana
contavam com 397 pessoas.

Atualmente, eles habitam sete aldeias, sendo cinco na Terra Indígena
Karitiana, demarcada em 1986, e duas fora dela, em áreas que são parte
do seu território tradicional. Algumas famílias também moram nas cidades
rondonienses de Porto Velho, a capital, e em Cacoal.

Além de trabalharem em atividades agrícolas, no manejo dos recursos
florestais e na produção de artesanato, os Karitiana também são
profissionais das áreas de saúde e educação. O grupo luta historicamente
pela garantia de seus direitos, como a ampliação da terra indígena e o
fortalecimento da educação e da saúde indígena.



