\newcommand{\linha}[2]{\ifdef{#2}{\linhalayout{#1}{#2}}{}}

\begingroup\tiny
\parindent=0cm
\thispagestyle{empty}

\textbf{copyright}\quad 					 {Hedra \the\year}\\ \textit{Direitos cedidos à Casa de Letras Eireli}\\

\textbf{narração}\quad			 	     	 {Cizino, Garcia e Barabadá Karitiana}\\
\textbf{organização e tradução\,©}\quad	 	 {Luciana Storto}\\
\textbf{coorganização}\quad			 	 {Íris Morais Araújo, Karin Vivanco e Ivan Rocha}\\
\textbf{coordenação da coleção}\quad		 {Luísa Valentini}\\

\textbf{edição}\quad			 			{Suzana Salama}\\
\textbf{revisão}\quad			 		 	{Renier Silva}\\
\textbf{capa}\quad			 			{Lucas Kröeff}\\
\textbf{assistência editorial}\quad		 	{Paulo Pompermaier}\\

\textbf{\textsc{isbn}}\quad			 		{978-65-6011-163-9}\smallskip

\vfill

\begin{minipage}{7cm}
\textbf{Dados Internacionais de Catalogação na Publicação (\textsc{cip})\\
(Câmara Brasileira do Livro, \textsc{sp}, Brasil)}

\textbf{\hrule}\smallskip

Storto, Luciana\\

\textit{Não havia mais homens}. Luciana Storto\\ 
1.\,ed. São Paulo, \textsc{sp}: Casa de Letras, 2025.\\

\textsc{isbn} 978-65-6011-163-9\\

1.\,Conto. 2.\,Literatura brasileira. \textsc{i}.\,Storto, Luciana. \textsc{ii}.Título.\\

 \hfill \textsc{cdd}: 869.93

\textbf{\hrule}\smallskip

\textbf{Elaborado por Janaina Ramos (\textsc{crb} 8/\,9166)}\\

\textbf{Índices para catálogo sistemático:}\\
\textsc{i.}\,Conto: Literatura brasileira


\end{minipage}

\vfill

\textit{Grafia atualizada segundo o Acordo Ortográfico da Língua\\
Portuguesa de 1990, em vigor no Brasil desde 2009.}\\

\textit{Direitos reservados em língua\\
portuguesa somente para o Brasil}\\

\textsc{casa de letras eireli}\\
Rua Fradique Coutinho, 1139\\
05416--001 São Paulo \textsc{sp} Brasil\\
Telefone +55 11 3914 7790\\\smallskip
comercial@casadeletras.com.br\\
www.casadeletras.com.br\\
\bigskip

Foi feito o depósito legal.

\endgroup
\pagebreak