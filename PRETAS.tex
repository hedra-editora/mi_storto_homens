% \textbf{Não havia mais homens} \textls[-10]{contém quatro narrativas de tradição oral dos Karitiana, contadas na primeira parte da década de 1990 por três narradores que são reconhecidas lideranças do grupo, conhecedores da mitologia e história do povo. Os mitos do sol e da lua foram contados pelo cacique Garcia, o ritual de iniciação masculino \textit{Osiip} por Cizino, que é o último pajé Karitiana, e a narrativa final, que retrata o reencontro ente as duas últimas aldeias da etnia, de onde foi tirado o título do livro \textit{Não havia mais homens}, foi contada por Barabadá. As narrativas foram gravadas pela organizadora, e depois transcritas e traduzidas com a ajuda de falantes da língua. Elas registram mitos e eventos importantes para o povo e devem ser lidos como parte de sua literatura.}

\textbf{Não havia mais homens} \textls[-10]{contém três narrativas de tradição oral dos Karitiana, contadas na primeira parte da década de 1990 por três narradores que são reconhecidas lideranças do grupo, conhecedores da mitologia e história do povo. O mito do sol forai contado pelo cacique Garcia, o ritual de iniciação masculino \textit{Osiip} por Cizino, que é o último pajé Karitiana, e a narrativa final, que retrata o reencontro ente as duas últimas aldeias da etnia, de onde foi tirado o título do livro \textit{Não havia mais homens}, foi contada por Barabadá. As narrativas foram gravadas pela organizadora, e depois transcritas e traduzidas com a ajuda de falantes da língua. Elas registram mitos e eventos importantes para o povo e devem ser lidos como parte de sua literatura.}

\textbf{Barabadá Karitiana} \textls[25]{foi um pajé Karitiana e pertenceu ao grupo Joari, também conhecido como Capivari. Ele narrou o encontro, vivenciado por ele, entre dois grupos de Karitiana que viviam em aldeias separadas: os Joari (ou Capivari) e os Karitiana. Estes grupos passaram a viver juntos na Terra Indígena Karitiana.}

\textbf{Cizino Karitiana} \textls[15]{é o atual pajé e cacique Karitiana. Ele narrou o ritual de iniciação masculina, chamado \textit{Osiip}, pelo qual passou várias vezes.}

\textbf{Garcia Karitiana} \textls[20]{foi um cacique Karitiana. Ele narrou o mito de origem do sol.}

\textbf{Luciana Storto} \textls[15]{é doutora em Linguística pelo Massachusetts Institute of Technology e professora do Departamento de Linguística da Universidade de São Paulo (\textsc{usp}). Estuda a língua karitiana desde 1992.}

\textls[18]{\textbf{Mundo Indígena} reúne materiais produzidos com pensadores de diferentes povos indígenas e pessoas que pesquisam, trabalham ou lutam pela garantia de seus direitos. Os livros foram feitos para serem utilizados pelas comunidades envolvidas na sua produção, e por isso uma parte significativa das obras é bilíngue. Esperamos divulgar a imensa diversidade linguística dos povos indígenas no Brasil, que compreende mais de 150 línguas pertencentes a mais de trinta famílias linguísticas.}



