\textbf{Não havia mais homens} 

\textbf{Luciana Storto} é doutora em Linguística pelo Massachusetts Institute of
Technology e professora do Departamento de Linguística da Universidade
de São Paulo (\textsc{usp}). Estuda a língua karitiana desde 1992.

% \textbf{Íris Morais Araújo} é doutora em Antropologia Social pela Universidade de
% São Paulo e pesquisadora do Centro de Pesquisa em Etnologia Indígena da
% Universidade Estadual de Campinas.

% \textbf{Karin Vivanco} é doutora em Linguística pela Universidade de São Paulo,
% foi bolsista de pós-doutorado do Departamento de Linguística da
% Universidade Estadual de Campinas e atualmente
% é professora da \textsc{ufrgs}.

\textbf{Coleção Mundo Indígena} reúne materiais produzidos com pensadores de diferentes povos indígenas e pessoas que pesquisam, trabalham ou lutam pela garantia de seus direitos. Os livros foram feitos para serem utilizados pelas comunidades envolvidas na sua produção, e por isso uma parte significativa das obras é bilíngue. Esperamos divulgar a imensa diversidade linguística dos povos indígenas no Brasil, que compreende mais de 150 línguas pertencentes a mais de trinta famílias linguísticas.

%(\textsc{fapesp} nº 2019/11661-4)
